\subsubsubsection{Построение диаграммы состояний автомата \(M\)}

Диаграмма состояний конечного автомата --- неупорядоченный ориентированный помеченный граф, вершины которого помечены именами состояний автомата и в котором есть дуга из вершины \(A\) к вершине \(B\) и если есть такой символ \(t\in\Sigma\), для которого существует функция перехода вида \(\delta(A,t)=B\) во множестве \(\delta\) конечного автомата \(M\). Кроме того, эта дуга помечается списком, состоящих из всех \(t\in\Sigma\), для которых есть функция перехода \(\delta(A, t) = B\).
Посторим димграммы состояний для КА \(M_1 = (Q_1, \Sigma, \delta_1, q_1, F_1)\) и \(M_2 = (Q_2, \Sigma, \delta_2, q_2, F_2)\).
\begin{figure}[h!]
	\centering
	\begin{tikzpicture}[
			->,
			>=stealth',
			node distance=2.5cm,
			every state/.style={thick, fill=gray!10},
			initial text={Начало}
		]

		% Состояния
		\node[state, initial] (H) {$H$};
		\node[state, right of=H] (S9) {$S_9$};
		\node[state, above right of=S9] (S15) {$S_{15}$};
		\node[state, right of=S15] (S11) {$S_{11}$};
		\node[state, accepting, below right of=S11] (S16) {$S_{16}$};

		% Переходы
		\path
		(H) edge node[above] {a, b} (S9)

		(S9) edge[bend left=20] node[above] {a, b} (S15)
		(S9) edge node[below] {a, b} (S16)

		(S15) edge[bend left=20] node[below] {a, b} (S9)
		(S15) edge node[above] {b, c} (S11)

		(S11) edge[bend left=20] node[above] {b, c} (S16)

		(S16) edge[bend left=20] node[below] {b, c} (S11);
	\end{tikzpicture}
	\caption{Диаграмма состояний недетерменированного конечного автомата \(M_1\)}
\end{figure}
\begin{figure}[h!]
	\centering
	\begin{tikzpicture}[
			->, % Стрелки от начала к концу
			>=stealth', % Стиль стрелок
			node distance=3cm, % Расстояние между узлами
			every state/.style={thick, fill=gray!10}, % Стиль состояний
			initial text={Начало} % Убираем текст "start" у начального состояния
		]

		% Состояния
		\node[state, initial] (S15) {$S_{15}$}; % Жирный кружок (начальное)
		\node[state, above right of=S15] (S10) {$S_{10}$};
		\node[state, below right of=S10] (S16) {$S_{16}$};
		\node[state, right of=S16] (S12) {$S_{12}$};
		\node[state, accepting, above right of=S12] (F) {$F$}; % Двойной кружок (конечное)

		% Переходы
		\path
		(S15) edge[bend left=20] node[above] {a, b} (S10)
		(S10) edge[bend left=20] node[below] {a, b} (S15)
		(S10) edge node[right] {a, b} (S16)
		(S10) edge node[above] {a, b} (F)
		(S16) edge[bend left=20] node[above] {b, c} (S12)
		(S12) edge[bend left=20] node[below] {b, c} (S16)
		(S12) edge node[above] {b, c} (F);
	\end{tikzpicture}
	\caption{Диаграмма состояний недетерменированного конечного автомата \(M_2\)}
\end{figure}

На этих диаграммах и далее выделенные состояний являются заключительными.
