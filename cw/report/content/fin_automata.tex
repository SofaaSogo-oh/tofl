\subsubsection{Построение конечного автомата для приведенной грамматики}
\begin{enumerate}
	\item Приведение к автоматному виду
	      Все правила в заданной грамматике имеют вид
	      \begin{align*}
		      P'_{19} \subset \cbr{A\to Bx|x \colon A, B \in \aleph, x \in \Sigma}
	      \end{align*}
	      для леволинейной грамматики, а для праволинейной
	      \begin{align*}
		      P''_{20} \subset \cbr{A\to xB|x \colon A, B \in \aleph, x \in \Sigma}
	      \end{align*}
	      А это в свою очередь значит, по построению, что правила данных грамматик \(G'_{19}\) и \(G''_{20}\) удовлетворяют определению автоматной грамматики, а, значит, изменение данных грамматик не производится.
	\item Построение конечных автоматов \(M_1 = (Q_1, \Sigma, \delta_1, q_1, F_1)\) и \(M_2 = (Q_2, \Sigma, \delta_2, q_2, F_2)\) для автоматных грамматик \(G'_{19}\) и \(G''_{20}\).
	      \subsubsubsection{Построение конечных автоматов \(M_1 = (Q_1, \Sigma, \delta_1, q_1, F_1)\) и \(M_2 = (Q_2, \Sigma, \delta_2, q_2, F_2)\) для автоматных грамматик \(G'_{20}\) и \(G''_{19}\).}
\begin{itemize}
	\item Построение автомата \(M_1 = (Q_1, \Sigma, \delta_1, q_1, F_1)\) для леволинейной грамматики производится следующим образом:
	      \begin{itemize}
		      \item Множество состояний состоит из именуемых нетерминалы состояний;
		      \item Добавляется новое состояние --- начальное (на наименование действуют соглашения по наименованию нетерминалов грамматик)
	      \end{itemize}
	      Таким образом
	      \begin{align*}
		      Q_1 = \aleph'_{20} \cup \cbr{H} = \cbr{H, S_9, S_{11}, S_{15}, S_{16}}
	      \end{align*}
	      Начальное состояние:
	      \begin{align*}
		      q_1 \equiv H
	      \end{align*}
	      Множество заключительных состояний содержит целевой символ исходной грамматики
	      \begin{align*}
		      F = \cbr{S_{16}}
	      \end{align*}
	      Множество переходов:
	      \begin{align*}
		      \begin{array}{lll}
			      \delta_1(S_{15}, a) = \cbr{S_9}    & \delta_1(S_{15}, b) = \cbr{S_9, S_{11}}   & \delta_1(S_{15}, c) = \cbr{S_{11}} \\
			      \delta_1(S_{16}, b) = \cbr{S_{11}} & \delta_1(S_{16}, c) = \cbr{S_{11}}                                             \\
			      \delta_1(S_{9}, a) = \cbr{S_{15}}  & \delta_1(S_{9}, b) = \cbr{S_{15}, S_{16}}                                      \\
			      \delta_1(S_{11}, b) = \cbr{S_{16}} & \delta_1(S_{11}, c) = \cbr{S_{16}}                                             \\
			      \delta_1(H, a) = \cbr{S_{9}}       & \delta_1(H, b) = \cbr{S_{9}, S_{11}}      & \delta_1(H, c) = \cbr{S_{16}}
		      \end{array}
	      \end{align*}
	\item Построение автомата \(M_2 = (Q_2, \Sigma, \delta_2, q_2, F_2)\) для леволинейной грамматики производится следующим образом:
	      \begin{itemize}
		      \item Множество состояний состоит из именуемых нетерминалы состояний;
		      \item Добавляется новое состояние --- заключительное (на наименование действуют соглашения по наименованию нетерминалов грамматик)
	      \end{itemize}
	      Таким образом
	      \begin{align*}
		      Q_2 = \aleph''_{20} \cup \cbr{F} = \cbr{F, S_{10}, S_{12}, S_{15}, S_{16}}
	      \end{align*}
	      Начальное состояние --- состояние, соответствующее целевому символу исходной грамматики:
	      \begin{align*}
		      q_2 \equiv S_{15}
	      \end{align*}
	      Множество заключительных состояний будет содержать новое состояние
	      \begin{align*}
		      F_2 = \cbr{F}
	      \end{align*}
	      Множество переходов:
	      \begin{align*}
		      \begin{array}{lll}
			      \delta_2(S_{10}, a) = \cbr{S_{15}, S_{16}} & \delta_2(S_{10}, b) = \cbr{S_{15}, S_{16}}                                      \\
			      \delta_2(S_{12}, b) = \cbr{S_{16}, F}      & \delta_2(S_{12}, c) = \cbr{S_{16}, F}                                           \\
			      \delta_2(S_{15}, a) = \cbr{S_{10}}         & \delta_2(S_{15}, b) = \cbr{S_{10}, S_{12}} & \delta_2(S_{15}, c) = \cbr{S_{12}} \\
			      \delta_2(S_{16}, b) = \cbr{S_{12}}         & \delta_2(S_{16}, c) = \cbr{S_{12}}
		      \end{array}
	      \end{align*}
\end{itemize}
На этом построение конечных автоматов по автоматным грамматикам заканчивается

	\item Построение диаграммы состояний автомата \(M\)
	      \subsubsubsection{Построение диаграммы состояний автомата \(M\)}

Диаграмма состояний конечного автомата --- неупорядоченный ориентированный помеченный граф, вершины которого помечены именами состояний автомата и в котором есть дуга из вершины \(A\) к вершине \(B\) и если есть такой символ \(t\in\Sigma\), для которого существует функция перехода вида \(\delta(A,t)=B\) во множестве \(\delta\) конечного автомата \(M\). Кроме того, эта дуга помечается списком, состоящих из всех \(t\in\Sigma\), для которых есть функция перехода \(\delta(A, t) = B\).
Посторим димграммы состояний для КА \(M_1 = (Q_1, \Sigma, \delta_1, q_1, F_1)\) и \(M_2 = (Q_2, \Sigma, \delta_2, q_2, F_2)\).
\begin{figure}[h!]
	\centering
	\begin{tikzpicture}[
			->,
			>=stealth',
			node distance=2.5cm,
			every state/.style={thick, fill=gray!10},
			initial text={Начало}
		]

		\node[state, initial] (H) {$H$};
		\node[state, above of=H] (S9) {$S_9$};
		\node[state, right of=S9] (S15) {$S_{15}$};
		\node[state, right of=H] (S11) {$S_{11}$};
		\node[state, accepting, right of=S11] (S16) {$S_{16}$};
		\path
		(H) edge node[left] {a,b} (S9)
		(H) edge node[above] {b,c} (S11)

		(S9) edge[bend left=20] node[above] {a,b} (S15)

		(S11) edge[bend left=20] node[above] {b,c} (S16)

		(S15) edge[bend left=20] node[below] {a,b} (S9)
		(S15) edge node[right] {b,c} (S11)

		(S16) edge[bend left=20] node[below] {b,c} (S11)
		;
	\end{tikzpicture}
	\caption{Диаграмма состояний недетерменированного конечного автомата \(M_1\)}
\end{figure}
\begin{figure}[h!]
	\centering
	\begin{tikzpicture}[
			->, % Стрелки от начала к концу
			>=stealth', % Стиль стрелок
			node distance=3cm, % Расстояние между узлами
			every state/.style={thick, fill=gray!10}, % Стиль состояний
			initial text={Начало} % Убираем текст "start" у начального состояния
		]
		\node[state, initial] (S15) {$S_{15}$};
		\node[state, above of=S15] (S10) {$S_{10}$};
		\node[state, right of=S15] (S12) {$S_{12}$};
		\node[state, right of=S10] (S16) {$S_{16}$};
		\node[state, accepting, right of=S12] (F) {$F$};

		\path
		(S15) edge[bend left=20] node[left] {a,b} (S10)
		edge node[below] {b,c} (S12)
		(S10) edge[bend left=20] node[right] {a,b} (S15)
		edge node[above] {a,b} (S16)
		(S12) edge node[below] {b,c} (F)
		edge[bend left=20] node[left] {b,c} (S16)
		(S16) edge[bend left=20] node[right] {b,c} (S12)
		;

	\end{tikzpicture}
	\caption{Диаграмма состояний недетерменированного конечного автомата \(M_2\)}
\end{figure}

На этих диаграммах и далее выделенные состояний являются заключительными.

\end{enumerate}

