\subsubsection{Построение регулярного выражения \(p_3\) По минимальному КА \(M\)}
Система уравнений с регулярными коэффициентами в правосторонней форме записи
\begin{align*}
  \left\{
  \begin{array}{l}
    H = \varnothing + \varnothing H + aA + bB + cC + \varnothing F_1 + \varnothing F_2 \\
    A = \varnothing + (a+b) H + \varnothing A + \varnothing B + \varnothing C + \varnothing F_1 + \varnothing F_2 \\
    B = \varnothing + aH + \varnothing A + \varnothing B + \varnothing C + b F_1 + c F_2 \\
    C = \varnothing + \varnothing H + \varnothing A + \varnothing C + \varnothing F_1 + (b+c) F_2 \\
    F_1 = \varepsilon + \varnothing H + a A + b B + c C + \varnothing F_1 + \varnothing F_2 \\
    F_2 = \varepsilon + \varnothing H + \varnothing A + \varnothing B + (b+c)C + \varnothing F_1 + \varnothing F_2
  \end{array}
  \right.
\end{align*}

\begin{align*}
	\left\{
	\begin{array}{l}
		H = aA + bB + cC                 \\
		A = aH + bH                      \\
		B = aH + bF_1 + cF_2             \\
		C = bF_2 + cF_2                  \\
		F_1 = aA + bB + cC + \varepsilon \\
		F_2 = bC + cC + \varepsilon
	\end{array}
	\right.
\end{align*}

Так как СУРК совпадает с СУРКом в правосторонней форме записи, то решение будет аналогичным и ответ будет совпадать
\begin{align*}
	p_3           & = ((a+b)(a+b))^*((b+c)(b+c))^+
\end{align*}
