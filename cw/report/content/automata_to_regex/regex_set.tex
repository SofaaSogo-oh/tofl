\subsubsection{Построение регулярных множеств \(L_1\), \(L_2\), \(L_3\) по регулярным выражениям \(p_1\), \(p_2\), \(p_3\)}
Полученные выражения \(p_1\), \(p_2\), \(p_3\) эквивалентны, следовательно, множества, которые они описывают, также эквивалентны. Построение язывка для регулярного выражения:
\begin{align*}
	p   & = ((a+b)(a+b))^* ((b+c)(b+c))^+                                                   \\
	L_1 & = \cbr{((a,b)^2)^k\cdot((b,c)^2)^m| \forall k \geq 0, m > 0, k, m \in \mathbb{Z}}
\end{align*}
\subsubsection{Сравнение исходного языка \(L\) и полученного}
Для исходного языка
\begin{align*}
	L & = \cbr{((a,b)^2)^k\cdot((b,c)^2)^m| \forall k \geq 0, m > 0, k, m \in \mathbb{Z}}
\end{align*}
Ранее было получено регулярное выражение вида:
\begin{align*}
	p = ((a+b)(a+b))^*((b+c)(b+c))^+
\end{align*}
Полученный и исходный языки эквивалентны.
