\subsubsubsection{ Удаление недостижимых символов грамматик \(G'_{19}\) и \(G''_{18}\)}
\begin{itemize}
	\item Для леволинейной грамматики \(G'_{19}\)
	      \begin{align*}
		      C_{0} & = \cbr{S_{16}}                                                                                       \\
		      C_{1} & = \cbr{S_9, S_{11}, a, b, c} \cup C_{0} = \cbr{S_9, S_{11}, S_{16}, a, b, c}                         \\
		      C_{2} & = \cbr{S_9, S_{11}, S_{15}, S_{16}, a, b, c} \cup C_{1} = \cbr{S_9, S_{11}, S_{15}, S_{16}, a, b, c} \\
		      C_{3} & = \cbr{S_9, S_{11}, S_{15}, S_{16}, a, b, c} \cup C_{2} = \cbr{S_9, S_{11}, S_{15}, S_{16}, a, b, c}
	      \end{align*}
	      Строим результирующую грамматику \(G'_{20}\) без недостижимых символов
	      \begin{align*}
		      \aleph'_{20} & = \aleph'_{19} \cap C_{3} = \cbr{S_9, S_{11}, S_{15}, S_{16}}                                                                \\
		      \Sigma'_{20} & = \Sigma'_{19} \cap C_{3} = \cbr{a, b, c}                                                                                    \\
		      P'_{20}      & = \cbr{(A \to \alpha) | \forall (A\to\alpha)\in P'_{18},A\in \aleph'_{19}, \alpha \in (\Sigma'_{19} \cup  \aleph'_{19})^*} = \\
		                   & = \cbr{\begin{array}{ll}
				                            S_9 \to S_{15}a|a|S_{15}b|b & S_{11} \to S_{16}b|S_{15}b|S_{16}c|S_{15}c|b|c \\
				                            S_{15} \to S_9a|S_9b        & S_{16} \to S_{11}b|S_{11}c
			                            \end{array}}                                          \\
		      S'_{20}      & \equiv S_{16}
	      \end{align*}
	      Таким образом, результирующая грамматика \(G'_{20}\) примет вид
	      \begin{align*}
		      G'_{20} = \grammatics{S_9, S_{11}, S_{15}, S_{16}}{\cbr{a, b, c}}{%
		      S_9 \to S_{15}a|a|S_{15}b|b & S_{11} \to S_{16}b|S_{15}b|S_{16}c|S_{15}c|b|c \\
		      S_{15} \to S_9a|S_9b        & S_{16} \to S_{11}b|S_{11}c
		      }{S_{16}}
	      \end{align*}
	\item Для праволинейной грамматики \(G''_{18}\)
	      \begin{align*}
		      C_{0} & = \cbr{S_{15}}                                                                                             \\
		      C_{1} & = \cbr{S_{10}, S_{12}, a, b, c} \cup C_{0} = \cbr{S_{10}, S_{12}, S_{15}, a, b, c}                         \\
		      C_{2} & = \cbr{S_{10}, S_{12}, S_{15}, S_{16}, a, b, c} \cup C_{1} = \cbr{S_{10}, S_{12}, S_{15}, S_{16}, a, b, c} \\
		      C_{3} & = \cbr{S_{10}, S_{12}, S_{15}, S_{16}, a, b, c} \cup C_{2} = \cbr{S_{10}, S_{12}, S_{15}, S_{16}, a, b, c}
	      \end{align*}
	      Строим результирующую грамматику \(G''_{19}\) без недостижимых символов
	      \begin{align*}
		      \aleph''_{19} & = \aleph''_{19} \cap C_{4} = \cbr{S_{10}, S_{12}, S_{15}, S_{16}}                                                                \\
		      \Sigma''_{19} & = \Sigma''_{19} \cap C_{4} = \cbr{a, b, c}                                                                                       \\
		      P''_{19}      & = \cbr{(A \to \alpha) | \forall (A\to\alpha)\in P''_{19},A\in \aleph''_{20}, \alpha \in (\Sigma''_{20} \cup  \aleph''_{20})^*} = \\
		                    & = \cbr{\begin{array}{ll}
				                             S_{10} \to aS_{15}|aS_{16}|bS_{15}|bS_{16} & S_{12} \to bS_{16}|b|cS_{16}|c \\
				                             S_{15} \to aS_{10}|bS_{10}|bS_{12}|cS_{12} & S_{16} \to bS_{12}|cS_{12}
			                             \end{array}}                                               \\
		      S''_{19}      & \equiv S_{15}
	      \end{align*}
	      Таким образом, результирующая грамматика \(G''_{19}\) примет вид
	      \begin{align*}
		      G''_{19} = \grammatics{S_{10}, S_{12}, S_{15}, S_{16}}{\cbr{a, b, c}}{%
		      S_{10} \to aS_{15}|aS_{16}|bS_{15}|bS_{16} & S_{12} \to bS_{16}|b|cS_{16}|c \\
		      S_{15} \to aS_{10}|bS_{10}|bS_{12}|cS_{12} & S_{16} \to bS_{12}|cS_{12}
		      }{S_{15}}
	      \end{align*}
\end{itemize}
