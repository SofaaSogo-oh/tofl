\subsection{Построение КА по регулярному выражению}
\subsubsection{Построение КА \(M_3\)}
Выполним построение конечных автоматов для выражения \cref{eq:regex}. Очередность построения конечных автоматов будет определяться таки же образом, как и в случае построения грамматик по регулярному выражению \cref{eq:regex-desc}.

Воспользуемся рекурсивным определением регулярного выражения для построения последовательности конечных автоматов для каждого элементарного регулярного выражения, входящих в состав выражения \cref{eq:regex-desc}. Собственно последний КА и будет являться искомым.

Построим КА для указанных выражений. Каждый КА будем нумеровать по номеру выражения, для которого строится данный  КА. Кроме того нумерация состояний КА будет определяться следующим образом: номер каждого состояния будет начинаться с номера конечного автомата.

Для выражения \(a\) конечный автомат примет вид
\begin{align*}
	M_1 = \rbr{\cbr{q_{10}, q_{11}}, \Sigma, \delta_1, q_{10}, \cbr{q_{11}}},
\end{align*}
где множество переходов \(\delta_1\) автомата будет содержать переходы вида
\begin{align*}
	\delta_1(q_{10}, a) = \cbr{q_{11}}
\end{align*}
Граф переходов построенного КА \(M_1\) примет вид
\begin{figure}[h!]
	\centering
	\begin{tikzpicture}[
			->,
			>=stealth',
			node distance=2.5cm,
			every state/.style={thick, fill=gray!10},
			initial text={Начало}
		]

		% Состояния
		\node[state, initial] (q10) {$q_{10}$};
		\node[state, accepting, right of=q10] (q11) {$q_{11}$};

		% Переходы
		\path
		(q10) edge node[above] {a} (q11);
	\end{tikzpicture}
	\caption{Диаграмма состояний НКА \(M_1\)}
\end{figure}

Для выражения \(b\) конечный автомат примет вид
\begin{align*}
	M_2 = \rbr{\cbr{q_{20}, q_{21}}, \Sigma, \delta_2, q_{20}, \cbr{q_{21}}},
\end{align*}
где множество переходов \(\delta_2\) автомата будет содержать переходы вида
\begin{align*}
	\delta_2(q_{20}, b) = \cbr{q_{21}}
\end{align*}
Граф переходов построенного КА \(M_2\) примет вид
\begin{figure}[h!]
	\centering
	\begin{tikzpicture}[
			->,
			>=stealth',
			node distance=2.5cm,
			every state/.style={thick, fill=gray!10},
			initial text={Начало}
		]

		% Состояния
		\node[state, initial] (q20) {$q_{20}$};
		\node[state, accepting, right of=q20] (q21) {$q_{21}$};

		% Переходы
		\path
		(q20) edge node[above] {b} (q21);
	\end{tikzpicture}
	\caption{Диаграмма состояний НКА \(M_2\)}
\end{figure}

Для выражения \(a\) конечный автомат примет вид
\begin{align*}
	M_3 = \rbr{\cbr{q_{30}, q_{31}}, \Sigma, \delta_3, q_{30}, \cbr{q_{31}}},
\end{align*}
где множество переходов \(\delta_3\) автомата будет содержать переходы вида
\begin{align*}
	\delta_3(q_{30}, a) = \cbr{q_{31}}
\end{align*}
Граф переходов построенного КА \(M_3\) примет вид
\begin{figure}[h!]
	\centering
	\begin{tikzpicture}[
			->,
			>=stealth',
			node distance=2.5cm,
			every state/.style={thick, fill=gray!10},
			initial text={Начало}
		]

		% Состояния
		\node[state, initial] (q30) {$q_{30}$};
		\node[state, accepting, right of=q30] (q31) {$q_{31}$};

		% Переходы
		\path
		(q30) edge node[above] {a} (q31);
	\end{tikzpicture}
	\caption{Диаграмма состояний НКА \(M_3\)}
\end{figure}

Для выражения \(b\) конечный автомат примет вид
\begin{align*}
	M_4 = \rbr{\cbr{q_{40}, q_{41}}, \Sigma, \delta_4, q_{40}, \cbr{q_{41}}},
\end{align*}
где множество переходов \(\delta_4\) автомата будет содержать переходы вида
\begin{align*}
	\delta_4(q_{40}, b) = \cbr{q_{41}}
\end{align*}
Граф переходов построенного КА \(M_4\) примет вид
\begin{figure}[h!]
	\centering
	\begin{tikzpicture}[
			->,
			>=stealth',
			node distance=2.5cm,
			every state/.style={thick, fill=gray!10},
			initial text={Начало}
		]

		% Состояния
		\node[state, initial] (q40) {$q_{40}$};
		\node[state, accepting, right of=q40] (q41) {$q_{41}$};

		% Переходы
		\path
		(q40) edge node[above] {b} (q41);
	\end{tikzpicture}
	\caption{Диаграмма состояний НКА \(M_4\)}
\end{figure}

Для выражения \(b\) конечный автомат примет вид
\begin{align*}
	M_5 = \rbr{\cbr{q_{50}, q_{51}}, \Sigma, \delta_5, q_{50}, \cbr{q_{51}}},
\end{align*}
где множество переходов \(\delta_5\) автомата будет содержать переходы вида
\begin{align*}
	\delta_5(q_{50}, b) = \cbr{q_{51}}
\end{align*}
Граф переходов построенного КА \(M_5\) примет вид
\begin{figure}[h!]
	\centering
	\begin{tikzpicture}[
			->,
			>=stealth',
			node distance=2.5cm,
			every state/.style={thick, fill=gray!10},
			initial text={Начало}
		]

		% Состояния
		\node[state, initial] (q50) {$q_{50}$};
		\node[state, accepting, right of=q50] (q51) {$q_{51}$};

		% Переходы
		\path
		(q50) edge node[above] {b} (q51);
	\end{tikzpicture}
	\caption{Диаграмма состояний НКА \(M_5\)}
\end{figure}

Для выражения \(c\) конечный автомат примет вид
\begin{align*}
	M_6 = \rbr{\cbr{q_{60}, q_{61}}, \Sigma, \delta_6, q_{60}, \cbr{q_{61}}},
\end{align*}
где множество переходов \(\delta_5\) автомата будет содержать переходы вида
\begin{align*}
	\delta_6(q_{60}, c) = \cbr{q_{61}}
\end{align*}
Граф переходов построенного КА \(M_6\) примет вид
\begin{figure}[h!]
	\centering
	\begin{tikzpicture}[
			->,
			>=stealth',
			node distance=2.5cm,
			every state/.style={thick, fill=gray!10},
			initial text={Начало}
		]

		% Состояния
		\node[state, initial] (q60) {$q_{60}$};
		\node[state, accepting, right of=q60] (q61) {$q_{61}$};

		% Переходы
		\path
		(q60) edge node[above] {c} (q61);
	\end{tikzpicture}
	\caption{Диаграмма состояний НКА \(M_6\)}
\end{figure}

\newpage

Для выражения \(b\) конечный автомат примет вид
\begin{align*}
	M_7 = \rbr{\cbr{q_{70}, q_{71}}, \Sigma, \delta_7, q_{70}, \cbr{q_{71}}},
\end{align*}
где множество переходов \(\delta_7\) автомата будет содержать переходы вида
\begin{align*}
	\delta_7(q_{70}, b) = \cbr{q_{71}}
\end{align*}
Граф переходов построенного КА \(M_7\) примет вид
\begin{figure}[h!]
	\centering
	\begin{tikzpicture}[
			->,
			>=stealth',
			node distance=2.5cm,
			every state/.style={thick, fill=gray!10},
			initial text={Начало}
		]

		% Состояния
		\node[state, initial] (q70) {$q_{70}$};
		\node[state, accepting, right of=q70] (q71) {$q_{71}$};

		% Переходы
		\path
		(q70) edge node[above] {b} (q71);
	\end{tikzpicture}
	\caption{Диаграмма состояний НКА \(M_7\)}
\end{figure}


Для выражения \(c\) конечный автомат примет вид
\begin{align*}
	M_8 = \rbr{\cbr{q_{80}, q_{81}}, \Sigma, \delta_8, q_{80}, \cbr{q_{81}}},
\end{align*}
где множество переходов \(\delta_5\) автомата будет содержать переходы вида
\begin{align*}
	\delta_8(q_{80}, c) = \cbr{q_{81}}
\end{align*}
Граф переходов построенного КА \(M_8\) примет вид
\begin{figure}[h!]
	\centering
	\begin{tikzpicture}[
			->,
			>=stealth',
			node distance=2.5cm,
			every state/.style={thick, fill=gray!10},
			initial text={Начало}
		]

		% Состояния
		\node[state, initial] (q80) {$q_{80}$};
		\node[state, accepting, right of=q80] (q81) {$q_{81}$};

		% Переходы
		\path
		(q80) edge node[above] {c} (q81);
	\end{tikzpicture}
	\caption{Диаграмма состояний НКА \(M_8\)}
\end{figure}

\newpage
Для выражения \(a+b\) строим КА \(M_9 = (Q_9, \Sigma, \delta_9, q_{90}, F_9)\) следующим образом:
\begin{enumerate}
	\item Множество состояний автомата \(M_9\) получается путем объединений множества состояний автоматов \(M_1\) и \(M_2\) и нового состояния \(q_{90}\)
	      \begin{align*}
		      Q_9 = Q_1 \cup Q_2 \cup \cbr{q_{90}} = \cbr{q_{10}, q_{11}, q_{20}, q_{21}, q_{90}}
	      \end{align*}
	\item \(q_{90}\) --- начальное состояние;
	\item Конечные состояния определяются как объединение конечных состояний \(M_1\) и \(M_2\)
	      \begin{align*}
		      F_{9} = F_{1} \cup F_{2} = \cbr{q_{11}, q_{21}}
	      \end{align*}
	\item Множество переходов \(\delta_9\) строится:
	      \begin{align*}
		      \begin{array}{ll}
			      \delta_9(q_{90}, a) = \cbr{q_{11}} & \delta_9(q_{90}, b) = \cbr{q_{21}} \\
			      \delta_9(q_{10}, a) = \cbr{q_{11}}                                      \\
			      \delta_9(q_{20}, b) = \cbr{q_{21}}
		      \end{array}
	      \end{align*}
	      Граф переходов построенного КА \(M_9\) примет вид
	      \begin{figure}[h!]
		      \centering
		      \begin{tikzpicture}[
				      ->,
				      >=stealth',
				      node distance=2.5cm,
				      every state/.style={thick, fill=gray!10},
				      initial text={Начало}
			      ]

			      % Состояния
			      \node[state, initial] (q90) {$q_{90}$};

			      \node[state, above  of=q90] (q10) {$q_{10}$};
			      \node[state, accepting, right of=q10] (q11) {$q_{11}$};

			      \node[state, below  of=q90] (q20) {$q_{20}$};
			      \node[state, accepting, right of=q20] (q21) {$q_{21}$};

			      % Переходы
			      \path
			      (q10) edge node[above] {a} (q11)
			      (q20) edge node[below] {b} (q21)
			      (q90) edge node[above] {a} (q11)
			      (q90) edge node[below] {b} (q21)
			      ;
		      \end{tikzpicture}
		      \caption{Диаграмма состояний НКА \(M_9\)}
	      \end{figure}
\end{enumerate}
\newpage
Для выражения \(a+b\) строим КА \(M_{10} = (Q_{10}, \Sigma, \delta_{10}, q_{100}, F_{10})\) следующим образом:
\begin{enumerate}
	\item Множество состояний автомата \(M_{10}\) получается путем объединений множества состояний автоматов \(M_1\) и \(M_2\) и нового состояния \(q_{100}\)
	      \begin{align*}
		      Q_{10} = Q_3 \cup Q_4 \cup \cbr{q_{100}} = \cbr{q_{30}, q_{31}, q_{40}, q_{41}, q_{100}}
	      \end{align*}
	\item \(q_{100}\) --- начальное состояние;
	\item Конечные состояния определяются как объединение конечных состояний \(M_3\) и \(M_4\)
	      \begin{align*}
		      F_{10} = F_{3} \cup F_{4} = \cbr{q_{31}, q_{41}}
	      \end{align*}
	\item Множество переходов \(\delta\) строится:
	      \begin{align*}
		      \begin{array}{ll}
			      \delta_{10}(q_{100}, a) = \cbr{q_{31}} & \delta_{10}(q_{100}, b) = \cbr{q_{41}} \\
			      \delta_{10}(q_{30}, a) = \cbr{q_{31}}                                           \\
			      \delta_{10}(q_{40}, b) = \cbr{q_{41}}
		      \end{array}
	      \end{align*}
	      Граф переходов построенного КА \(M_{10}\) примет вид
	      \begin{figure}[h!]
		      \centering
		      \begin{tikzpicture}[
				      ->,
				      >=stealth',
				      node distance=2.5cm,
				      every state/.style={thick, fill=gray!10},
				      initial text={Начало}
			      ]

			      % Состояния
			      \node[state, initial] (q100) {$q_{100}$};

			      \node[state, above  of=q100] (q30) {$q_{30}$};
			      \node[state, accepting, right of=q30] (q31) {$q_{31}$};

			      \node[state, below  of=q100] (q40) {$q_{40}$};
			      \node[state, accepting, right of=q40] (q41) {$q_{41}$};

			      % Переходы
			      \path
			      (q30) edge node[above] {a} (q31)
			      (q40) edge node[below] {b} (q41)
			      (q100) edge node[above] {a} (q31)
			      (q100) edge node[below] {b} (q41)
			      ;
		      \end{tikzpicture}
		      \caption{Диаграмма состояний НКА \(M_{10}\)}
	      \end{figure}
\end{enumerate}
\newpage
Для выражения \(b+c\) строим КА \(M_{11} = (Q_{11}, \Sigma, \delta_{11}, q_{110}, F_{11})\) следующим образом:
\begin{enumerate}
	\item Множество состояний автомата \(M_{11}\) получается путем объединений множества состояний автоматов \(M_1\) и \(M_2\) и нового состояния \(q_{110}\)
	      \begin{align*}
		      Q_{11} = Q_1 \cup Q_2 \cup \cbr{q_{110}} = \cbr{q_{50}, q_{51}, q_{60}, q_{61}, q_{110}}
	      \end{align*}
	\item \(q_{110}\) --- начальное состояние;
	\item Конечные состояния определяются как объединение конечных состояний \(M_5\) и \(M_6\)
	      \begin{align*}
		      F_{11} = F_{5} \cup F_{6} = \cbr{q_{51}, q_{61}}
	      \end{align*}
	\item Множество переходов \(\delta\) строится:
	      \begin{align*}
		      \begin{array}{ll}
			      \delta_{11}(q_{110}, b) = \cbr{q_{51}} & \delta_{11}(q_{110}, c) = \cbr{q_{61}} \\
			      \delta_{11}(q_{50}, b) = \cbr{q_{51}}                                           \\
			      \delta_{11}(q_{60}, c) = \cbr{q_{61}}
		      \end{array}
	      \end{align*}
	      Граф переходов построенного КА \(M_{11}\) примет вид
	      \begin{figure}[h!]
		      \centering
		      \begin{tikzpicture}[
				      ->,
				      >=stealth',
				      node distance=2.5cm,
				      every state/.style={thick, fill=gray!10},
				      initial text={Начало}
			      ]

			      % Состояния
			      \node[state, initial] (q110) {$q_{110}$};

			      \node[state, above  of=q110] (q50) {$q_{50}$};
			      \node[state, accepting, right of=q50] (q51) {$q_{51}$};

			      \node[state, below  of=q110] (q60) {$q_{60}$};
			      \node[state, accepting, right of=q60] (q61) {$q_{61}$};

			      % Переходы
			      \path
			      (q50) edge node[above] {b} (q51)
			      (q60) edge node[below] {c} (q61)
			      (q110) edge node[above] {b} (q51)
			      (q110) edge node[below] {c} (q61)
			      ;
		      \end{tikzpicture}
		      \caption{Диаграмма состояний НКА \(M_{11}\)}
	      \end{figure}
\end{enumerate}

\newpage
Для выражения \(b+c\) строим КА \(M_{12} = (Q_{12}, \Sigma, \delta_{12}, q_{120}, F_{12})\) следующим образом:
\begin{enumerate}
	\item Множество состояний автомата \(M_{12}\) получается путем объединений множества состояний автоматов \(M_1\) и \(M_2\) и нового состояния \(q_{120}\)
	      \begin{align*}
		      Q_{12} = Q_1 \cup Q_2 \cup \cbr{q_{120}} = \cbr{q_{70}, q_{71}, q_{80}, q_{81}, q_{120}}
	      \end{align*}
	\item \(q_{120}\) --- начальное состояние;
	\item Конечные состояния определяются как объединение конечных состояний \(M_7\) и \(M_8\)
	      \begin{align*}
		      F_{12} = F_{7} \cup F_{8} = \cbr{q_{71}, q_{81}}
	      \end{align*}
	\item Множество переходов \(\delta\) строится:
	      \begin{align*}
		      \begin{array}{ll}
			      \delta_{12}(q_{120}, b) = \cbr{q_{71}} & \delta_{12}(q_{120}, c) = \cbr{q_{81}} \\
			      \delta_{12}(q_{70}, b) = \cbr{q_{71}}                                           \\
			      \delta_{12}(q_{80}, c) = \cbr{q_{81}}
		      \end{array}
	      \end{align*}
	      Граф переходов построенного КА \(M_{12}\) примет вид
	      \begin{figure}[h!]
		      \centering
		      \begin{tikzpicture}[
				      ->,
				      >=stealth',
				      node distance=2.5cm,
				      every state/.style={thick, fill=gray!10},
				      initial text={Начало}
			      ]

			      % Состояния
			      \node[state, initial] (q120) {$q_{120}$};

			      \node[state, above  of=q120] (q70) {$q_{70}$};
			      \node[state, accepting, right of=q70] (q71) {$q_{71}$};

			      \node[state, below  of=q120] (q80) {$q_{80}$};
			      \node[state, accepting, right of=q80] (q81) {$q_{81}$};

			      % Переходы
			      \path
			      (q70) edge node[above] {b} (q71)
			      (q80) edge node[below] {c} (q81)
			      (q120) edge node[above] {b} (q71)
			      (q120) edge node[below] {c} (q81)
			      ;
		      \end{tikzpicture}
		      \caption{Диаграмма состояний НКА \(M_{12}\)}
	      \end{figure}
\end{enumerate}

\newpage
Для выражения \((a+b)(a+b)\) строим КА \(M_{13} = (Q_{13}, \Sigma, \delta_{13}, q_{130}, F_{13})\):
\begin{enumerate}
	\item множество состояний автомата \(M_{13}\) получается путём объединения множеств состояний исходных автоматов
	      \begin{align*}
		      Q_{13} = Q_{9} \cup Q_{10} = \cbr{q_{10}, q_{11}, q_{20}, q_{21}, q_{90},q_{30}, q_{31}, q_{40}, q_{41}, q_{100} };
	      \end{align*}
	\item начальным состоянием результирующего автомата \(M_{13}\) будет начальное состояние автомата \(M_9\)
	      \begin{align*}
		      q_{130} \equiv q_{90};
	      \end{align*}
	\item множество заключительных состояний \(F_{13}\) будет содержать только множество заключительных состояний автомата \(M_{10}\)
	      \begin{align*}
		      F_{13} = F_{10} = \cbr{q_{31}, q_{41}}
	      \end{align*}
	\item множество переходов \(\delta_{13}\) автомата \(M_{13}\) будет содержать переходы автомата \(M_{9}\) кроме переходов из заключительных состояний
	      \begin{align*}
		      \begin{array}{ll}
			      \delta_{13}(q_{90}, a)  = \delta_9(q_{90}, a) = \cbr{q_{11}} & \delta_{13}(q_{90}, b)  =  \delta_9(q_{90}, b) = \cbr{q_{21}} \\
			      \delta_{13}(q_{10}, a)  = \delta_9(q_{10}, a) = \cbr{q_{11}} & \delta_{13}(q_{20}, b)  = \delta_9(q_{20}, b) = \cbr{q_{21}},
		      \end{array}
	      \end{align*}
	      а также добавляются переходы из заключительных состояний первого автомата в состояния второго, в которые имеются переходы из начальных состояний второго автомата
	      \begin{align*}
		      \begin{array}{ll}
			      \delta_{13}(q_{11}, a)  = \varnothing \cup \cbr{q_{31}} = \cbr{q_{31}} & \delta_{13}(q_{11}, b)  = \varnothing \cup \cbr{q_{41}} = \cbr{q_{41}}  \\
			      \delta_{13}(q_{21}, a)  = \varnothing \cup \cbr{q_{31}} = \cbr{q_{31}} & \delta_{13}(q_{21}, b)  = \varnothing \cup \cbr{q_{41}} = \cbr{q_{41}}.
		      \end{array}
	      \end{align*}
	      Кроме этого добавляются все состояния автомата \(M_{10}\)
	      \begin{align*}
		      \begin{array}{ll}
			      \delta_{13}(q_{100}, a)=      \delta_{10}(q_{100}, a) = \cbr{q_{31}} & \delta_{13}(q_{100}, b) =\delta_{10}(q_{100}, b) = \cbr{q_{41}} \\
			      \delta_{13}(q_{30}, a) =      \delta_{10}(q_{30}, a) = \cbr{q_{31}}  & \delta_{13}(q_{40}, b)  =\delta_{10}(q_{40}, b) = \cbr{q_{41}}
		      \end{array}
	      \end{align*}
\end{enumerate}
Граф переходов построенного КА \(M_{13}\) примет вид:
\begin{figure}[h!]
	\centering
	\begin{tikzpicture}[
			->,
			>=stealth',
			node distance=2.0cm,
			every state/.style={thick, fill=gray!10},
			initial text={Начало}
		]

		% Состояния
		\node[state, initial] (q90) {$q_{90}$};

		\node[state, above  of=q90] (q10) {$q_{10}$};
		\node[state,  right of=q10] (q11) {$q_{11}$};

		\node[state, below  of=q90] (q20) {$q_{20}$};
		\node[state,  right of=q20] (q21) {$q_{21}$};

		\node[state, accepting, right of=q11] (q31) {$q_{31}$};
		\node[state, accepting, right of=q21] (q41) {$q_{41}$};
		\node[state, right of=q31] (q30) {$q_{30}$};
		\node[state, right of=q41] (q40) {$q_{40}$};
		\node[state, below of=q30] (q100) {$q_{100}$};

		% Переходы
		\path
		(q10) edge node[above] {a} (q11)
		(q20) edge node[below] {b} (q21)
		(q90) edge node[above] {a} (q11)
		(q90) edge node[below] {b} (q21)

		(q30) edge node[above] {a} (q31)
		(q40) edge node[below] {b} (q41)
		(q100) edge node[above] {a} (q31)
		(q100) edge node[below] {b} (q41)

		(q11) edge node[above] {a} (q31)
		(q21) edge[bend left=20] node[above] {a} (q31)

		(q11) edge[bend right=20] node[below] {b} (q41)
		(q21) edge node[below] {b} (q41)
		;
	\end{tikzpicture}
	\caption{Диаграмма состояний НКА \(M_{13}\)}
\end{figure}

\newpage
Для выражения \((b+c)(b+c)\) строим КА \(M_{14} = (Q_{14}, \Sigma, \delta_{14}, q_{140}, F_{14})\):
\begin{enumerate}
	\item множество состояний автомата \(M_{14}\) получается путём объединения множеств состояний исходных автоматов
	      \begin{align*}
		      Q_{14} = Q_{11} \cup Q_{12} = \cbr{q_{50}, q_{51}, q_{60}, q_{61}, q_{110}, q_{70}, q_{71}, q_{80}, q_{81}, q_{120}};
	      \end{align*}
	\item начальным состоянием результирующего автомата \(M_{14}\) будет начальное состояние автомата \(M_{11}\)
	      \begin{align*}
		      q_{140} \equiv q_{110};
	      \end{align*}
	\item множество заключительных состояний \(F_{14}\) будет содержать только множество заключительных состояний автомата \(M_{12}\)
	      \begin{align*}
		      F_{14} = F_{12} = \cbr{q_{71}, q_{81}}
	      \end{align*}
	\item множество переходов \(\delta_{14}\) автомата \(M_{14}\) будет содержать переходы автомата \(M_{11}\) кроме переходов из заключительных состояний
	      \begin{align*}
		      \begin{array}{ll}
			      \delta_{14}(q_{110}, b) = \delta_{11}(q_{110}, b) = \cbr{q_{51}} & \delta_{14}(q_{110}, c) = \delta_{11}(q_{110}, c) = \cbr{q_{61}} \\
			      \delta_{14}(q_{50}, b)  = \delta_{11}(q_{50}, b) = \cbr{q_{51}}  & \delta_{14}(q_{60}, c)  = \delta_{11}(q_{60}, c) = \cbr{q_{61}}
		      \end{array}
	      \end{align*}
	      а также добавляются переходы из заключительных состояний первого автомата в состояния второго, в которые имеются переходы из начальных состояний второго автомата
	      \begin{align*}
		      \begin{array}{ll}
			      \delta_{14}(q_{51}, b)  = \varnothing \cup \cbr{q_{71}} = \cbr{q_{71}} & \delta_{14}(q_{51}, c)  = \varnothing \cup \cbr{q_{81}} = \cbr{q_{81}}  \\
			      \delta_{14}(q_{61}, b)  = \varnothing \cup \cbr{q_{71}} = \cbr{q_{71}} & \delta_{14}(q_{61}, c)  = \varnothing \cup \cbr{q_{81}} = \cbr{q_{81}}.
		      \end{array}
	      \end{align*}
	      Кроме этого добавляются все состояния автомата \(M_{12}\)
	      \begin{align*}
		      \begin{array}{ll}
			      \delta_{14}(q_{120}, b) =\delta_{12}(q_{120}, b) = \cbr{q_{71}} & \delta_{14}(q_{120}, c) =\delta_{12}(q_{120}, c) = \cbr{q_{81}} \\
			      \delta_{14}(q_{70}, b)  =\delta_{12}(q_{70}, b) = \cbr{q_{71}}  & \delta_{14}(q_{80}, c)  =\delta_{12}(q_{80}, c) = \cbr{q_{81}}
		      \end{array}
	      \end{align*}
\end{enumerate}
Граф переходов построенного КА \(M_{14}\) примет вид:
\begin{figure}[h!]
	\centering
	\begin{tikzpicture}[
			->,
			>=stealth',
			node distance=2.0cm,
			every state/.style={thick, fill=gray!10},
			initial text={Начало}
		]

		% Состояния
		\node[state, initial] (q110) {$q_{110}$};

		\node[state, above  of=q110] (q50) {$q_{50}$};
		\node[state,  right of=q50] (q51) {$q_{51}$};

		\node[state, below  of=q110] (q60) {$q_{60}$};
		\node[state,  right of=q60] (q61) {$q_{61}$};

		\node[state, accepting, right of=q51] (q71) {$q_{71}$};
		\node[state, accepting, right of=q61] (q81) {$q_{81}$};
		\node[state, right of=q71] (q70) {$q_{70}$};
		\node[state, right of=q81] (q80) {$q_{80}$};
		\node[state, above of=q80] (q120) {$q_{120}$};

		% Переходы
		\path
		(q50) edge node[above] {b} (q51)
		(q60) edge node[below] {c} (q61)
		(q110) edge node[above] {b} (q11)
		(q110) edge node[below] {c} (q21)

		(q70) edge node[above] {b} (q71)
		(q80) edge node[below] {c} (q81)
		(q120) edge node[above] {b} (q71)
		(q120) edge node[below] {c} (q81)

		(q51) edge node[above] {b} (q71)
		(q61) edge[bend left=20] node[above] {b} (q71)

		(q51) edge[bend right=20] node[below] {c} (q81)
		(q61) edge node[below] {c} (q81)
		;
	\end{tikzpicture}
	\caption{Диаграмма состояний НКА \(M_{14}\)}
\end{figure}

\newpage
Для выражения \(((a+b)(a+b))^+\) строим КА \(M_{15}=(Q_{15}, \Sigma, \delta_{15}, q_{150}, F_{15})\):
\begin{enumerate}
	\item множество состояний конечного автомтата \(M_{13}\) переносится с добавлением нового состояния \(q_{150}\), состояние \(q_{150}\) --- начальное
	      \begin{align*}
		      Q_{15} = Q_{13} \cup \cbr{ q_{150}} = \cbr{q_{10}, q_{11}, q_{20}, q_{21}, q_{90},q_{30}, q_{31}, q_{40}, q_{41}, q_{100}, q_{150} }.
	      \end{align*}
	\item множество результирующих состояний автомтата переносится без изменений
	      \begin{align*}
		      F_{15} = F_{13} = \cbr{q_{31}, q_{41}}
	      \end{align*}
	\item множество переходов  \(\delta_{15}\) сохраняет все те переходы из незаключительных состояний, что и в автомате \(M_{13}\)
	      \begin{align*}
		      \begin{array}{ll}
			      \delta_{15}(q_{90}, a)  = \delta_{13}(q_{90}, a) = \cbr{q_{11}} & \delta_{15}(q_{90}, b)  =  \delta_{13}(q_{90}, b) = \cbr{q_{21}} \\
			      \delta_{15}(q_{10}, a)  = \delta_{13}(q_{10}, a) = \cbr{q_{11}} & \delta_{15}(q_{20}, b)  = \delta_{13}(q_{20}, b) = \cbr{q_{21}}  \\
			      \delta_{15}(q_{100}, a)= \delta_{13}(q_{100}, a) = \cbr{q_{31}} & \delta_{15}(q_{100}, b) =\delta_{13}(q_{100}, b) = \cbr{q_{41}}  \\
			      \delta_{15}(q_{30}, a) = \delta_{13}(q_{30}, a) = \cbr{q_{31}}  & \delta_{15}(q_{40}, b)  =\delta_{13}(q_{40}, b) = \cbr{q_{41}}   \\
			      \delta_{15}(q_{11}, a) =\delta_{13}(q_{11}, a)  =  \cbr{q_{31}} & \delta_{15}(q_{11}, b)=\delta_{13}(q_{11}, b)  =  \cbr{q_{41}}   \\
			      \delta_{15}(q_{21}, a) =\delta_{13}(q_{21}, a)  =  \cbr{q_{31}} & \delta_{15}(q_{21}, b)=\delta_{13}(q_{21}, b)  =  \cbr{q_{41}},
		      \end{array}
	      \end{align*}
	      добавляются переходы из заключительных состояний автомата в состояния, в которые ведут переходы начального состояния автомата
	      \begin{align*}
		      \begin{array}{ll}
			      \delta_{15}(q_{31}, a) = \varnothing \cup \delta_{13}(q_{90}, a) = \cbr{q_{11}} & \delta_{15}(q_{31}, b) = \varnothing \cup \delta_{13}(q_{90}, b) = \cbr{q_{21}}  \\
			      \delta_{15}(q_{41}, a) = \varnothing \cup \delta_{13}(q_{90}, a) = \cbr{q_{11}} & \delta_{15}(q_{41}, b) = \varnothing \cup \delta_{13}(q_{90}, b) = \cbr{q_{21}},
		      \end{array}
	      \end{align*}
	      для нового начального состояния \(q_{150}\) переносятся все переходы из старого началльного состояния \(q_{90}\)
	      \begin{align*}
		      \begin{array}{ll}
			      \delta_{15}(q_{150}, a)  = \delta_{13}(q_{90}, a) = \cbr{q_{11}} & \delta_{15}(q_{150}, b)  =  \delta_{13}(q_{90}, b) = \cbr{q_{21}}.
		      \end{array}
	      \end{align*}
\end{enumerate}
Граф переходов построенного КА \(M_{15}\) примет вид:
\begin{figure}[h!]
	\centering
	\begin{tikzpicture}[
			->,
			>=stealth',
			node distance=2.0cm,
			every state/.style={thick, fill=gray!10},
			initial text={Начало}
		]

		% Состояния
		\node[state, initial] (q150) {$q_{150}$};

		\node[state, above  of=q150] (q10) {$q_{10}$};
		\node[state,  right of=q10] (q11) {$q_{11}$};

		\node[state, below  of=q150] (q20) {$q_{20}$};
		\node[state,  right of=q20] (q21) {$q_{21}$};

		\node[state, right of=q150] (q90) {$q_{90}$};

		\node[state, accepting, right= 3cm of q11] (q31) {$q_{31}$};
		\node[state, accepting, right= 3cm of q21] (q41) {$q_{41}$};
		\node[state, right of=q31] (q30) {$q_{30}$};
		\node[state, right of=q41] (q40) {$q_{40}$};
		\node[state, below of=q30] (q100) {$q_{100}$};

		% Переходы
		\path
		(q10) edge node[above] {a} (q11)
		(q20) edge node[below] {b} (q21)

		(q150) edge node[above] {a} (q11)
		(q150) edge node[below] {b} (q21)

		(q90) edge node[left] {a} (q11)
		(q90) edge node[left] {b} (q21)

		(q30) edge node[above] {a} (q31)
		(q40) edge node[below] {b} (q41)
		(q100) edge node[above] {a} (q31)
		(q100) edge node[below] {b} (q41)

		(q11) edge node[above] {a} (q31)
		(q21) edge[bend left=20] node[above] {a} (q31)

		(q11) edge[bend right=20] node[below] {b} (q41)
		(q21) edge node[below] {b} (q41)

		(q31) edge[bend right=20] node[above]{a} (q11)
		(q31) edge[bend left=20] node[below]{b} (q21)

		(q41) edge[bend left=20] node[below]{b} (q21)
		(q41) edge[bend right=20] node[above]{a} (q11)
		;
	\end{tikzpicture}
	\caption{Диаграмма состояний НКА \(M_{15}\)}
\end{figure}

\newpage
Для выражения \(((b+c)(b+c))^*\) строим КА \(M_{16} = (Q_{16}, \Sigma, \delta_{16}, q_{160}, F_{16})\):
\begin{enumerate}
	\item множество состояний конечного автомата \(M_{14}\) переносится с добавлением нового состояния \(q_{160}\), состояние \(q_{160}\) --- начальное
	      \begin{align*}
		      Q_{16} = Q_{14} \cup \cbr{q_{160}} =\cbr{q_{50}, q_{51}, q_{60}, q_{61}, q_{110}, q_{70}, q_{71}, q_{80}, q_{81}, q_{120}, q_{160}}.
	      \end{align*}
	\item множество результирующих состояний переносится с добавлением нового состояния \(q_{160}\)
	      \begin{align*}
		      F_{16} = F_{14} \cup \cbr{q_{160}} = \cbr{q_{71}, q_{81}, q_{160}}
	      \end{align*}
	\item множество переходов \(\delta_{16}\) сохраняет все переходы из незаключительных состояний, что и в автомате \(M_{14}\)
	      \begin{align*}
		      \begin{array}{ll}
			      \delta_{16}(q_{110}, b) = \delta_{14}(q_{110}, b) = \cbr{q_{51}} & \delta_{16}(q_{110}, c) = \delta_{14}(q_{110}, c) = \cbr{q_{61}} \\
			      \delta_{16}(q_{50}, b)  = \delta_{14}(q_{50}, b)  = \cbr{q_{51}} & \delta_{16}(q_{60}, c)  = \delta_{14}(q_{60}, c)  = \cbr{q_{61}} \\
			      \delta_{16}(q_{51}, b)  = \delta_{14}(q_{51}, b)  = \cbr{q_{71}} & \delta_{16}(q_{51}, c)  = \delta_{14}(q_{51}, c)  = \cbr{q_{81}} \\
			      \delta_{16}(q_{61}, b)  = \delta_{14}(q_{61}, b)  = \cbr{q_{71}} & \delta_{16}(q_{61}, c)  = \delta_{14}(q_{61}, c)  = \cbr{q_{81}} \\
			      \delta_{16}(q_{120}, b) = \delta_{14}(q_{120}, b) = \cbr{q_{71}} & \delta_{16}(q_{120}, c) = \delta_{14}(q_{120}, c) = \cbr{q_{81}} \\
			      \delta_{16}(q_{70}, b)  = \delta_{14}(q_{70}, b) = \cbr{q_{71}}  & \delta_{16}(q_{80}, c) =  \delta_{14}(q_{80}, c) = \cbr{q_{81}},
		      \end{array}
	      \end{align*}
	      добавляются переходы из заключительных состояний автомата в состояния, в которые ведут переходы начального состояния автомата
	      \begin{align*}
		      \begin{array}{ll}
			      \delta_{16}(q_{71}, b) = \varnothing \cup \delta_{14}(q_{110}, b) = \cbr{q_{51}} & \delta_{16}(q_{71}, c) = \varnothing \cup \delta_{14}(q_{110}, c) = \cbr{q_{61}}  \\
			      \delta_{16}(q_{81}, b) = \varnothing \cup \delta_{14}(q_{110}, b) = \cbr{q_{51}} & \delta_{16}(q_{81}, c) = \varnothing \cup \delta_{14}(q_{110}, c) = \cbr{q_{61}},
		      \end{array}
	      \end{align*}
	      для нового начального состояния \(q_{160}\) переносятся все переходы из старого начального состояния \(q_{110}\)
	      \begin{align*}
		      \begin{array}{ll}
			      \delta_{16}(q_{160}, b) = \delta_{16}(q_{110}, b) = \cbr{q_{51}} & \delta_{16}(q_{160}, c) = \delta_{16}(q_{110}, c) = \cbr{q_{61}}
		      \end{array}
	      \end{align*}
\end{enumerate}
Граф переходов построенного КА \(M_{16}\) примет вид:
\begin{figure}[h!]
	\centering
	\begin{tikzpicture}[
			->,
			>=stealth',
			node distance=2.0cm,
			every state/.style={thick, fill=gray!10},
			initial text={Начало}
		]

		% Состояния
		\node[state, initial, accepting] (q160) {$q_{160}$};
		\node[state, right of=q160] (q110) {$q_{110}$};

		\node[state, above  of=q160] (q50) {$q_{50}$};
		\node[state,  right of=q50] (q51) {$q_{51}$};

		\node[state, below  of=q160] (q60) {$q_{60}$};
		\node[state,  right of=q60] (q61) {$q_{61}$};

		\node[state, accepting, right = 3cm of q51] (q71) {$q_{71}$};
		\node[state, accepting, right = 3cm of q61] (q81) {$q_{81}$};
		\node[state, right of=q71] (q70) {$q_{70}$};
		\node[state, right of=q81] (q80) {$q_{80}$};
		\node[state, above of=q80] (q120) {$q_{120}$};

		% Переходы
		\path
		(q110) edge node[left] {b} (q51)
		(q110) edge node[left] {c} (q61)

		(q50) edge node[above] {b} (q51)
		(q60) edge node[below] {c} (q61)
		(q160) edge node[above] {b} (q11)
		(q160) edge node[below] {c} (q21)

		(q70) edge node[above] {b} (q71)
		(q80) edge node[below] {c} (q81)
		(q120) edge node[above] {b} (q71)
		(q120) edge node[below] {c} (q81)

		(q51) edge node[above] {b} (q71)
		(q61) edge[bend left=20] node[above] {b} (q71)

		(q51) edge[bend right=20] node[below] {c} (q81)
		(q61) edge node[below] {c} (q81)

		(q71) edge[bend right=20] node[above] {b} (q51)
		(q81) edge[bend left=20] node[below] {c} (q61)

		(q71) edge[bend left=20] node[below]{c} (q61)
		(q81) edge[bend right=20] node[above]{b} (q51)
		;
	\end{tikzpicture}
	\caption{Диаграмма состояний НКА \(M_{16}\)}
\end{figure}

\newpage
Для выражения \(((a+b)(a+b))^*((b+c)(b+c))^+\) строим КА \(M_{17}=(Q_{17}, \Sigma, \delta_{17}, q_{170}, F_{17})\):
\begin{enumerate}
	\item Множество состояний автомата \(M_{17}\) получается путём объединения множеств состояний исходных автоматов
	      \begin{align*}
		      Q_{17} = Q_{15} \cup Q_{16} =  \cbr{
			      \begin{array}{l}
				      q_{10}, q_{11}, q_{20}, q_{21}, q_{90},q_{30}, q_{31}, q_{40}, q_{41}, q_{100}, q_{150}, \\
				      q_{50}, q_{51}, q_{60}, q_{61}, q_{110}, q_{70}, q_{71}, q_{80}, q_{81}, q_{120}, q_{160}
			      \end{array}}
	      \end{align*}
	\item начальным состоянием результирующего автомата \(M_{17}\) будет начальное состояние автомата \(M_{15}\)
	      \begin{align*}
		      q_{170} \equiv q_{150}
	      \end{align*}
	\item множество заключительных состояний \(F_{17}\) будет содержать только множество заключительных состояний автомата \(M_{16}\)
	      \begin{align*}
		      F_{17} = F_{16} = \cbr{q_{71}, q_{81}}
	      \end{align*}
	\item множество переходов \(\delta_{17}\) автомата \(M_{17}\) будет содержать все переходы автомата \(M_{15}\) кроме переходов из заключительных состояний
	      \begin{align*}
		      \begin{array}{ll}
			      \delta_{17}(q_{90}, a)  = \delta_{15}(q_{90}, a) = \cbr{q_{11}} & \delta_{17}(q_{90}, b)  =  \delta_{15}(q_{90}, b) = \cbr{q_{21}} \\
			      \delta_{17}(q_{10}, a)  = \delta_{15}(q_{10}, a) = \cbr{q_{11}} & \delta_{17}(q_{20}, b)  = \delta_{15}(q_{20}, b) = \cbr{q_{21}}  \\
			      \delta_{17}(q_{100}, a)= \delta_{15}(q_{100}, a) = \cbr{q_{31}} & \delta_{17}(q_{100}, b) =\delta_{15}(q_{100}, b) = \cbr{q_{41}}  \\
			      \delta_{17}(q_{30}, a) = \delta_{15}(q_{30}, a) = \cbr{q_{31}}  & \delta_{17}(q_{40}, b)  =\delta_{15}(q_{40}, b) = \cbr{q_{41}}   \\
			      \delta_{17}(q_{11}, a) =\delta_{15}(q_{11}, a)  =  \cbr{q_{31}} & \delta_{17}(q_{11}, b)=\delta_{15}(q_{11}, b)  =  \cbr{q_{41}}   \\
			      \delta_{17}(q_{21}, a) =\delta_{15}(q_{21}, a)  =  \cbr{q_{31}} & \delta_{17}(q_{21}, b)=\delta_{15}(q_{21}, b)  =  \cbr{q_{41}},  \\
		      \end{array}
	      \end{align*}
	      а также добавляются переходы из заключительных состояний первого автомата в состояния второго, в которые имеются переходы из начальных состояний второго автомата
	      \begin{align*}
		      \begin{array}{lll}
			      \delta_{17}(q_{31}, a) = \cbr{q_{11}}  & \delta_{17}(q_{31}, b) = \cbr{q_{21}, q_{51}}  & \delta_{17}(q_{31}, c) = \cbr{q_{61}}  \\
			      \delta_{17}(q_{41}, a) = \cbr{q_{11}}  & \delta_{17}(q_{41}, b) = \cbr{q_{21}, q_{51}}  & \delta_{17}(q_{41}, c) = \cbr{q_{61}}  \\
			      \delta_{17}(q_{150}, a) = \cbr{q_{11}} & \delta_{17}(q_{150}, b) = \cbr{q_{21}, q_{51}} & \delta_{17}(q_{150}, c) = \cbr{q_{61}} \\
		      \end{array}
	      \end{align*}
	      Кроме этого добавляются все состояния автомата \(M_{16}\)
	      \begin{align*}
		      \begin{array}{ll}
			      \delta_{17}(q_{110}, b) = \delta_{16}(q_{110}, b) = \cbr{q_{51}} & \delta_{17}(q_{110}, c) = \delta_{16}(q_{110}, c) = \cbr{q_{61}} \\
			      \delta_{17}(q_{50}, b)  = \delta_{16}(q_{50}, b)  = \cbr{q_{51}} & \delta_{17}(q_{60}, c)  = \delta_{16}(q_{60}, c)  = \cbr{q_{61}} \\
			      \delta_{17}(q_{51}, b)  = \delta_{16}(q_{51}, b)  = \cbr{q_{71}} & \delta_{17}(q_{51}, c)  = \delta_{16}(q_{51}, c)  = \cbr{q_{81}} \\
			      \delta_{17}(q_{61}, b)  = \delta_{16}(q_{61}, b)  = \cbr{q_{71}} & \delta_{17}(q_{61}, c)  = \delta_{16}(q_{61}, c)  = \cbr{q_{81}} \\
			      \delta_{17}(q_{120}, b) = \delta_{16}(q_{120}, b) = \cbr{q_{71}} & \delta_{17}(q_{120}, c) = \delta_{16}(q_{120}, c) = \cbr{q_{81}} \\
			      \delta_{17}(q_{70}, b)  = \delta_{16}(q_{70}, b) = \cbr{q_{71}}  & \delta_{17}(q_{80}, c) =  \delta_{16}(q_{80}, c) = \cbr{q_{81}}  \\
			      \delta_{17}(q_{71}, b) = \delta_{16}(q_{71}, b) = \cbr{q_{51}}   & \delta_{17}(q_{71}, c) = \delta_{16}(q_{71}, c) =\cbr{q_{61}}    \\
			      \delta_{17}(q_{81}, b) = \delta_{16}(q_{81}, b) = \cbr{q_{51}}   & \delta_{17}(q_{81}, c) = \delta_{16}(q_{81}, c) =\cbr{q_{61}}    \\
			      \delta_{17}(q_{160}, b) = \delta_{16}(q_{160}, b) = \cbr{q_{51}} & \delta_{17}(q_{160}, c) = \delta_{16}(q_{160}, c) = \cbr{q_{61}}
		      \end{array}
	      \end{align*}
\end{enumerate}

\begin{figure}[h!]
	\centering
	\begin{tikzpicture}[
			->,
			>=stealth',
			node distance=2.0cm,
			every state/.style={thick, fill=gray!10},
			initial text={Начало}
		]

		% Состояния
		\node[state, initial] (q150) {$q_{150}$};

		\node[state, above  of=q150] (q10) {$q_{10}$};
		\node[state,  right of=q10] (q11) {$q_{11}$};

		\node[state, below  of=q150] (q20) {$q_{20}$};
		\node[state,  right of=q20] (q21) {$q_{21}$};

		\node[state, right of=q150] (q90) {$q_{90}$};

		\node[state, right= of q11] (q31) {$q_{31}$};
		\node[state, right= of q21] (q41) {$q_{41}$};
		\node[state, right of=q31] (q30) {$q_{30}$};
		\node[state, right of=q41] (q40) {$q_{40}$};
		\node[state, below of=q31] (q100) {$q_{100}$};

		% Переходы
		\path
		(q10) edge node[above] {a} (q11)
		(q20) edge node[below] {b} (q21)

		(q150) edge node[above] {a} (q11)
		(q150) edge node[below] {b} (q21)

		(q90) edge node[left] {a} (q11)
		(q90) edge node[left] {b} (q21)

		(q30) edge node[below] {a} (q31)
		(q40) edge node[above] {b} (q41)
		(q100) edge node[right] {a} (q31)
		(q100) edge node[right] {b} (q41)

		(q11) edge node[below] {a} (q31)
		(q21) edge[bend left=20] node[above] {a} (q31)

		(q11) edge[bend right=20] node[below] {b} (q41)
		(q21) edge node[above] {b} (q41)

		(q31) edge[bend right=20] node[above]{a} (q11)
		(q31) edge[bend left=20] node[below]{b} (q21)

		(q41) edge[bend left=20] node[below]{b} (q21)
		(q41) edge[bend right=20] node[above]{a} (q11)
		;

		\node[state, right of=q100] (q160) {$q_{160}$};
		\node[state, right of=q160] (q110) {$q_{110}$};

		\node[state, above of=q110] (q51) {$q_{51}$};
		\node[state, above of=q51] (q50) {$q_{50}$};

		\node[state, below of=q110] (q61) {$q_{61}$};
		\node[state, below of=q61] (q60) {$q_{60}$};

		\node[state, accepting, right = of q51] (q71) {$q_{71}$};
		\node[state, accepting, right = of q61] (q81) {$q_{81}$};
		\node[state, above of=q71] (q70) {$q_{70}$};
		\node[state, below of=q81] (q80) {$q_{80}$};
		\node[state, below of=q71] (q120) {$q_{120}$};

		% Переходы
		\path
		(q110) edge node[left] {b} (q51)
		(q110) edge node[left] {c} (q61)

		(q50) edge node[right] {b} (q51)
		(q60) edge node[right] {c} (q61)
		(q160) edge node[above] {b} (q51)
		(q160) edge node[below] {c} (q61)

		(q70) edge node[right] {b} (q71)
		(q80) edge node[right] {c} (q81)
		(q120) edge node[right] {b} (q71)
		(q120) edge node[right] {c} (q81)

		(q51) edge node[below] {b} (q71)
		(q61) edge[bend left=20] node[above] {b} (q71)

		(q51) edge[bend right=20] node[below] {c} (q81)
		(q61) edge node[above] {c} (q81)

		(q71) edge[bend right=20] node[above] {b} (q51)
		(q81) edge[bend left=20] node[below] {c} (q61)

		(q71) edge[bend left=20] node[below]{c} (q61)
		(q81) edge[bend right=20] node[above]{b} (q51)
		;

		\path
		(q41) edge[bend right=30] node[below]{c} (q61)
		(q31) edge[bend left=30] node[above]{b} (q51)
		(q41) edge[bend left=20] node[above]{b} (q51)
		(q31) edge[bend right=20] node[below]{c} (q61)

		(q150) edge[bend right=60] node[below]{c} (q61)
		(q150) edge[bend left=60] node[above]{b} (q51)
		;

	\end{tikzpicture}
	\caption{Диаграмма состояний НКА \(M_{17}\)}
\end{figure}

% \begin{figure}[h!]
% 	\centering
% 	\begin{tikzpicture}[
% 			->,
% 			>=stealth',
% 			node distance=2.0cm,
% 			every state/.style={thick, fill=gray!10},
% 			initial text={Начало}
% 		]
%
% 		% Состояния
% 		\node[state, initial] (q150) {$q_{150}$};
% 		\node[state,  right of=q10] (q11) {$q_{11}$};
% 		\node[state,  right of=q20] (q21) {$q_{21}$};
% 		\node[state, right= of q11] (q31) {$q_{31}$};
% 		\node[state, right= of q21] (q41) {$q_{41}$};
% 		% Переходы
% 		\path
% 		(q150) edge node[below] {b} (q21)
% 		(q11) edge[bend right=20] node[below] {b} (q41)
% 		(q21) edge node[above] {b} (q41)
% 		(q31) edge[bend left=20] node[below]{b} (q21)
% 		(q41) edge[bend left=20] node[below]{b} (q21)
% 		;
% 		\node[state, above of=q110] (q51) {$q_{51}$};
% 		\node[state, below of=q110] (q61) {$q_{61}$};
% 		\node[state, accepting, right = of q51] (q71) {$q_{71}$};
% 		\node[state, accepting, right = of q61] (q81) {$q_{81}$};
% 		% Переходы
% 		\path
% 		(q51) edge node[below] {b} (q71)
% 		(q61) edge[bend left=20] node[above] {b} (q71)
% 		(q71) edge[bend right=20] node[above] {b} (q51)
% 		(q81) edge[bend right=20] node[above]{b} (q51)
% 		;
%
% 		\path
% 		(q31) edge[bend left=30] node[above]{b} (q51)
% 		(q41) edge[bend left=20] node[above]{b} (q51)
% 		(q150) edge[bend left=60] node[above]{b} (q51)
% 		;
%
% 	\end{tikzpicture}
% 	\caption{Диаграмма состояний НКА \(M_{17}\)}
% \end{figure}


% S -> UUTTTb -> UUTTAb -> UUTAAAb -> UUAATAAb -> UUAAAATAb -> UUAAAAAATb -> UUAAAAAAAb -> 
% -> UAAUAAAAb -> UAAAAUAb -> UAAAAb -> AAUAb -> AAb -> aab
