\newpage
Для выражения \(a+b\) строим КА \(M_9 = (Q_9, \Sigma, \delta_9, q_{90}, F_9)\) следующим образом:
\begin{enumerate}
	\item Множество состояний автомата \(M_9\) получается путем объединений множества состояний автоматов \(M_1\) и \(M_2\) и нового состояния \(q_{90}\)
	      \begin{align*}
		      Q_9 = Q_1 \cup Q_2 \cup \cbr{q_{90}} = \cbr{q_{10}, q_{11}, q_{20}, q_{21}, q_{90}}
	      \end{align*}
	\item \(q_{90}\) --- начальное состояние;
	\item Конечные состояния определяются как объединение конечных состояний \(M_1\) и \(M_2\)
	      \begin{align*}
		      F_{9} = F_{1} \cup F_{2} = \cbr{q_{11}, q_{21}}
	      \end{align*}
	\item Множество переходов \(\delta_9\) строится:
	      \begin{align*}
		      \begin{array}{ll}
			      \delta_9(q_{90}, a) = \cbr{q_{11}} & \delta_9(q_{90}, b) = \cbr{q_{21}} \\
			      \delta_9(q_{10}, a) = \cbr{q_{11}}                                      \\
			      \delta_9(q_{20}, b) = \cbr{q_{21}}
		      \end{array}
	      \end{align*}
	      Граф переходов построенного КА \(M_9\) примет вид
	      \begin{figure}[h!]
		      \centering
		      \begin{tikzpicture}[
				      ->,
				      >=stealth',
				      node distance=2.5cm,
				      every state/.style={thick, fill=gray!10},
				      initial text={Начало}
			      ]

			      % Состояния
			      \node[state, initial] (q90) {$q_{90}$};

			      \node[state, above  of=q90] (q10) {$q_{10}$};
			      \node[state, accepting, right of=q10] (q11) {$q_{11}$};

			      \node[state, below  of=q90] (q20) {$q_{20}$};
			      \node[state, accepting, right of=q20] (q21) {$q_{21}$};

			      % Переходы
			      \path
			      (q10) edge node[above] {a} (q11)
			      (q20) edge node[below] {b} (q21)
			      (q90) edge node[above] {a} (q11)
			      (q90) edge node[below] {b} (q21)
			      ;
		      \end{tikzpicture}
		      \caption{Диаграмма состояний НКА \(M_9\)}
	      \end{figure}
\end{enumerate}
\newpage
Для выражения \(a+b\) строим КА \(M_{10} = (Q_{10}, \Sigma, \delta_{10}, q_{100}, F_{10})\) следующим образом:
\begin{enumerate}
	\item Множество состояний автомата \(M_{10}\) получается путем объединений множества состояний автоматов \(M_1\) и \(M_2\) и нового состояния \(q_{100}\)
	      \begin{align*}
		      Q_{10} = Q_3 \cup Q_4 \cup \cbr{q_{100}} = \cbr{q_{30}, q_{31}, q_{40}, q_{41}, q_{100}}
	      \end{align*}
	\item \(q_{100}\) --- начальное состояние;
	\item Конечные состояния определяются как объединение конечных состояний \(M_3\) и \(M_4\)
	      \begin{align*}
		      F_{10} = F_{3} \cup F_{4} = \cbr{q_{31}, q_{41}}
	      \end{align*}
	\item Множество переходов \(\delta\) строится:
	      \begin{align*}
		      \begin{array}{ll}
			      \delta_{10}(q_{100}, a) = \cbr{q_{31}} & \delta_{10}(q_{100}, b) = \cbr{q_{41}} \\
			      \delta_{10}(q_{30}, a) = \cbr{q_{31}}                                           \\
			      \delta_{10}(q_{40}, b) = \cbr{q_{41}}
		      \end{array}
	      \end{align*}
	      Граф переходов построенного КА \(M_{10}\) примет вид
	      \begin{figure}[h!]
		      \centering
		      \begin{tikzpicture}[
				      ->,
				      >=stealth',
				      node distance=2.5cm,
				      every state/.style={thick, fill=gray!10},
				      initial text={Начало}
			      ]

			      % Состояния
			      \node[state, initial] (q100) {$q_{100}$};

			      \node[state, above  of=q100] (q30) {$q_{30}$};
			      \node[state, accepting, right of=q30] (q31) {$q_{31}$};

			      \node[state, below  of=q100] (q40) {$q_{40}$};
			      \node[state, accepting, right of=q40] (q41) {$q_{41}$};

			      % Переходы
			      \path
			      (q30) edge node[above] {a} (q31)
			      (q40) edge node[below] {b} (q41)
			      (q100) edge node[above] {a} (q31)
			      (q100) edge node[below] {b} (q41)
			      ;
		      \end{tikzpicture}
		      \caption{Диаграмма состояний НКА \(M_{10}\)}
	      \end{figure}
\end{enumerate}
\newpage
Для выражения \(b+c\) строим КА \(M_{11} = (Q_{11}, \Sigma, \delta_{11}, q_{110}, F_{11})\) следующим образом:
\begin{enumerate}
	\item Множество состояний автомата \(M_{11}\) получается путем объединений множества состояний автоматов \(M_1\) и \(M_2\) и нового состояния \(q_{110}\)
	      \begin{align*}
		      Q_{11} = Q_1 \cup Q_2 \cup \cbr{q_{110}} = \cbr{q_{50}, q_{51}, q_{60}, q_{61}, q_{110}}
	      \end{align*}
	\item \(q_{110}\) --- начальное состояние;
	\item Конечные состояния определяются как объединение конечных состояний \(M_5\) и \(M_6\)
	      \begin{align*}
		      F_{11} = F_{5} \cup F_{6} = \cbr{q_{51}, q_{61}}
	      \end{align*}
	\item Множество переходов \(\delta\) строится:
	      \begin{align*}
		      \begin{array}{ll}
			      \delta_{11}(q_{110}, b) = \cbr{q_{51}} & \delta_{11}(q_{110}, c) = \cbr{q_{61}} \\
			      \delta_{11}(q_{50}, b) = \cbr{q_{51}}                                           \\
			      \delta_{11}(q_{60}, c) = \cbr{q_{61}}
		      \end{array}
	      \end{align*}
	      Граф переходов построенного КА \(M_{11}\) примет вид
	      \begin{figure}[h!]
		      \centering
		      \begin{tikzpicture}[
				      ->,
				      >=stealth',
				      node distance=2.5cm,
				      every state/.style={thick, fill=gray!10},
				      initial text={Начало}
			      ]

			      % Состояния
			      \node[state, initial] (q110) {$q_{110}$};

			      \node[state, above  of=q110] (q50) {$q_{50}$};
			      \node[state, accepting, right of=q50] (q51) {$q_{51}$};

			      \node[state, below  of=q110] (q60) {$q_{60}$};
			      \node[state, accepting, right of=q60] (q61) {$q_{61}$};

			      % Переходы
			      \path
			      (q50) edge node[above] {b} (q51)
			      (q60) edge node[below] {c} (q61)
			      (q110) edge node[above] {b} (q51)
			      (q110) edge node[below] {c} (q61)
			      ;
		      \end{tikzpicture}
		      \caption{Диаграмма состояний НКА \(M_{11}\)}
	      \end{figure}
\end{enumerate}

\newpage
Для выражения \(b+c\) строим КА \(M_{12} = (Q_{12}, \Sigma, \delta_{12}, q_{120}, F_{12})\) следующим образом:
\begin{enumerate}
	\item Множество состояний автомата \(M_{12}\) получается путем объединений множества состояний автоматов \(M_1\) и \(M_2\) и нового состояния \(q_{120}\)
	      \begin{align*}
		      Q_{12} = Q_1 \cup Q_2 \cup \cbr{q_{120}} = \cbr{q_{70}, q_{71}, q_{80}, q_{81}, q_{120}}
	      \end{align*}
	\item \(q_{120}\) --- начальное состояние;
	\item Конечные состояния определяются как объединение конечных состояний \(M_7\) и \(M_8\)
	      \begin{align*}
		      F_{12} = F_{7} \cup F_{8} = \cbr{q_{71}, q_{81}}
	      \end{align*}
	\item Множество переходов \(\delta\) строится:
	      \begin{align*}
		      \begin{array}{ll}
			      \delta_{12}(q_{120}, b) = \cbr{q_{71}} & \delta_{12}(q_{120}, c) = \cbr{q_{81}} \\
			      \delta_{12}(q_{70}, b) = \cbr{q_{71}}                                           \\
			      \delta_{12}(q_{80}, c) = \cbr{q_{81}}
		      \end{array}
	      \end{align*}
	      Граф переходов построенного КА \(M_{12}\) примет вид
	      \begin{figure}[h!]
		      \centering
		      \begin{tikzpicture}[
				      ->,
				      >=stealth',
				      node distance=2.5cm,
				      every state/.style={thick, fill=gray!10},
				      initial text={Начало}
			      ]

			      % Состояния
			      \node[state, initial] (q120) {$q_{120}$};

			      \node[state, above  of=q120] (q70) {$q_{70}$};
			      \node[state, accepting, right of=q70] (q71) {$q_{71}$};

			      \node[state, below  of=q120] (q80) {$q_{80}$};
			      \node[state, accepting, right of=q80] (q81) {$q_{81}$};

			      % Переходы
			      \path
			      (q70) edge node[above] {b} (q71)
			      (q80) edge node[below] {c} (q81)
			      (q120) edge node[above] {b} (q71)
			      (q120) edge node[below] {c} (q81)
			      ;
		      \end{tikzpicture}
		      \caption{Диаграмма состояний НКА \(M_{12}\)}
	      \end{figure}
\end{enumerate}
