\newpage
Для выражения \(((a+b)(a+b))^*((b+c)(b+c))^+\) строим КА \(M_{17}=(Q_{17}, \Sigma, \delta_{17}, q_{170}, F_{17})\):
\begin{enumerate}
	\item Множество состояний автомата \(M_{17}\) получается путём объединения множеств состояний исходных автоматов
	      \begin{align*}
		      Q_{17} = Q_{15} \cup Q_{16} =  \cbr{
			      \begin{array}{l}
				      q_{10}, q_{11}, q_{20}, q_{21}, q_{90},q_{30}, q_{31}, q_{40}, q_{41}, q_{100}, q_{150}, \\
				      q_{50}, q_{51}, q_{60}, q_{61}, q_{110}, q_{70}, q_{71}, q_{80}, q_{81}, q_{120}, q_{160}
			      \end{array}}
	      \end{align*}
	\item начальным состоянием результирующего автомата \(M_{17}\) будет начальное состояние автомата \(M_{15}\)
	      \begin{align*}
		      q_{170} \equiv q_{150}
	      \end{align*}
	\item множество заключительных состояний \(F_{17}\) будет содержать только множество заключительных состояний автомата \(M_{16}\)
	      \begin{align*}
		      F_{17} = F_{16} = \cbr{q_{71}, q_{81}}
	      \end{align*}
	\item множество переходов \(\delta_{17}\) автомата \(M_{17}\) будет содержать все переходы автомата \(M_{15}\) кроме переходов из заключительных состояний
	      \begin{align*}
		      \begin{array}{ll}
			      \delta_{17}(q_{90}, a)  = \delta_{15}(q_{90}, a) = \cbr{q_{11}} & \delta_{17}(q_{90}, b)  =  \delta_{15}(q_{90}, b) = \cbr{q_{21}} \\
			      \delta_{17}(q_{10}, a)  = \delta_{15}(q_{10}, a) = \cbr{q_{11}} & \delta_{17}(q_{20}, b)  = \delta_{15}(q_{20}, b) = \cbr{q_{21}}  \\
			      \delta_{17}(q_{100}, a)= \delta_{15}(q_{100}, a) = \cbr{q_{31}} & \delta_{17}(q_{100}, b) =\delta_{15}(q_{100}, b) = \cbr{q_{41}}  \\
			      \delta_{17}(q_{30}, a) = \delta_{15}(q_{30}, a) = \cbr{q_{31}}  & \delta_{17}(q_{40}, b)  =\delta_{15}(q_{40}, b) = \cbr{q_{41}}   \\
			      \delta_{17}(q_{11}, a) =\delta_{15}(q_{11}, a)  =  \cbr{q_{31}} & \delta_{17}(q_{11}, b)=\delta_{15}(q_{11}, b)  =  \cbr{q_{41}}   \\
			      \delta_{17}(q_{21}, a) =\delta_{15}(q_{21}, a)  =  \cbr{q_{31}} & \delta_{17}(q_{21}, b)=\delta_{15}(q_{21}, b)  =  \cbr{q_{41}},  \\
		      \end{array}
	      \end{align*}
	      а также добавляются переходы из заключительных состояний первого автомата в состояния второго, в которые имеются переходы из начальных состояний второго автомата
	      \begin{align*}
		      \begin{array}{lll}
			      \delta_{17}(q_{31}, a) = \cbr{q_{11}}  & \delta_{17}(q_{31}, b) = \cbr{q_{21}, q_{51}}  & \delta_{17}(q_{31}, c) = \cbr{q_{61}}  \\
			      \delta_{17}(q_{41}, a) = \cbr{q_{11}}  & \delta_{17}(q_{41}, b) = \cbr{q_{21}, q_{51}}  & \delta_{17}(q_{41}, c) = \cbr{q_{61}}  \\
			      \delta_{17}(q_{150}, a) = \cbr{q_{11}} & \delta_{17}(q_{150}, b) = \cbr{q_{21}, q_{51}} & \delta_{17}(q_{150}, c) = \cbr{q_{61}} \\
		      \end{array}
	      \end{align*}
	      Кроме этого добавляются все состояния автомата \(M_{16}\)
	      \begin{align*}
		      \begin{array}{ll}
			      \delta_{17}(q_{110}, b) = \delta_{16}(q_{110}, b) = \cbr{q_{51}} & \delta_{17}(q_{110}, c) = \delta_{16}(q_{110}, c) = \cbr{q_{61}} \\
			      \delta_{17}(q_{50}, b)  = \delta_{16}(q_{50}, b)  = \cbr{q_{51}} & \delta_{17}(q_{60}, c)  = \delta_{16}(q_{60}, c)  = \cbr{q_{61}} \\
			      \delta_{17}(q_{51}, b)  = \delta_{16}(q_{51}, b)  = \cbr{q_{71}} & \delta_{17}(q_{51}, c)  = \delta_{16}(q_{51}, c)  = \cbr{q_{81}} \\
			      \delta_{17}(q_{61}, b)  = \delta_{16}(q_{61}, b)  = \cbr{q_{71}} & \delta_{17}(q_{61}, c)  = \delta_{16}(q_{61}, c)  = \cbr{q_{81}} \\
			      \delta_{17}(q_{120}, b) = \delta_{16}(q_{120}, b) = \cbr{q_{71}} & \delta_{17}(q_{120}, c) = \delta_{16}(q_{120}, c) = \cbr{q_{81}} \\
			      \delta_{17}(q_{70}, b)  = \delta_{16}(q_{70}, b) = \cbr{q_{71}}  & \delta_{17}(q_{80}, c) =  \delta_{16}(q_{80}, c) = \cbr{q_{81}}  \\
			      \delta_{17}(q_{71}, b) = \delta_{16}(q_{71}, b) = \cbr{q_{51}}   & \delta_{17}(q_{71}, c) = \delta_{16}(q_{71}, c) =\cbr{q_{61}}    \\
			      \delta_{17}(q_{81}, b) = \delta_{16}(q_{81}, b) = \cbr{q_{51}}   & \delta_{17}(q_{81}, c) = \delta_{16}(q_{81}, c) =\cbr{q_{61}}    \\
			      \delta_{17}(q_{160}, b) = \delta_{16}(q_{160}, b) = \cbr{q_{51}} & \delta_{17}(q_{160}, c) = \delta_{16}(q_{160}, c) = \cbr{q_{61}}
		      \end{array}
	      \end{align*}
\end{enumerate}

\begin{figure}[h!]
	\centering
	\begin{tikzpicture}[
			->,
			>=stealth',
			node distance=2.0cm,
			every state/.style={thick, fill=gray!10},
			initial text={Начало}
		]

		% Состояния
		\node[state, initial] (q150) {$q_{150}$};

		\node[state, above  of=q150] (q10) {$q_{10}$};
		\node[state,  right of=q10] (q11) {$q_{11}$};

		\node[state, below  of=q150] (q20) {$q_{20}$};
		\node[state,  right of=q20] (q21) {$q_{21}$};

		\node[state, right of=q150] (q90) {$q_{90}$};

		\node[state, right= of q11] (q31) {$q_{31}$};
		\node[state, right= of q21] (q41) {$q_{41}$};
		\node[state, right of=q31] (q30) {$q_{30}$};
		\node[state, right of=q41] (q40) {$q_{40}$};
		\node[state, below of=q31] (q100) {$q_{100}$};

		% Переходы
		\path
		(q10) edge node[above] {a} (q11)
		(q20) edge node[below] {b} (q21)

		(q150) edge node[above] {a} (q11)
		(q150) edge node[below] {b} (q21)

		(q90) edge node[left] {a} (q11)
		(q90) edge node[left] {b} (q21)

		(q30) edge node[below] {a} (q31)
		(q40) edge node[above] {b} (q41)
		(q100) edge node[right] {a} (q31)
		(q100) edge node[right] {b} (q41)

		(q11) edge node[below] {a} (q31)
		(q21) edge[bend left=20] node[above] {a} (q31)

		(q11) edge[bend right=20] node[below] {b} (q41)
		(q21) edge node[above] {b} (q41)

		(q31) edge[bend right=20] node[above]{a} (q11)
		(q31) edge[bend left=20] node[below]{b} (q21)

		(q41) edge[bend left=20] node[below]{b} (q21)
		(q41) edge[bend right=20] node[above]{a} (q11)
		;

		\node[state, right of=q100] (q160) {$q_{160}$};
		\node[state, right of=q160] (q110) {$q_{110}$};

		\node[state, above of=q110] (q51) {$q_{51}$};
		\node[state, above of=q51] (q50) {$q_{50}$};

		\node[state, below of=q110] (q61) {$q_{61}$};
		\node[state, below of=q61] (q60) {$q_{60}$};

		\node[state, accepting, right = of q51] (q71) {$q_{71}$};
		\node[state, accepting, right = of q61] (q81) {$q_{81}$};
		\node[state, above of=q71] (q70) {$q_{70}$};
		\node[state, below of=q81] (q80) {$q_{80}$};
		\node[state, below of=q71] (q120) {$q_{120}$};

		% Переходы
		\path
		(q110) edge node[left] {b} (q51)
		(q110) edge node[left] {c} (q61)

		(q50) edge node[right] {b} (q51)
		(q60) edge node[right] {c} (q61)
		(q160) edge node[above] {b} (q51)
		(q160) edge node[below] {c} (q61)

		(q70) edge node[right] {b} (q71)
		(q80) edge node[right] {c} (q81)
		(q120) edge node[right] {b} (q71)
		(q120) edge node[right] {c} (q81)

		(q51) edge node[below] {b} (q71)
		(q61) edge[bend left=20] node[above] {b} (q71)

		(q51) edge[bend right=20] node[below] {c} (q81)
		(q61) edge node[above] {c} (q81)

		(q71) edge[bend right=20] node[above] {b} (q51)
		(q81) edge[bend left=20] node[below] {c} (q61)

		(q71) edge[bend left=20] node[below]{c} (q61)
		(q81) edge[bend right=20] node[above]{b} (q51)
		;

		\path
		(q41) edge[bend right=30] node[below]{c} (q61)
		(q31) edge[bend left=30] node[above]{b} (q51)
		(q41) edge[bend left=20] node[above]{b} (q51)
		(q31) edge[bend right=20] node[below]{c} (q61)

		(q150) edge[bend right=60] node[below]{c} (q61)
		(q150) edge[bend left=60] node[above]{b} (q51)
		;

	\end{tikzpicture}
	\caption{Диаграмма состояний НКА \(M_{17}\)}
\end{figure}
