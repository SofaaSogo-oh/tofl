\newpage
Для выражения \(((a+b)(a+b))^+((b+c)(b+c))^*\) строим КА \(M_{17}=(Q_{17}, \Sigma, \delta_{17}, q_{170}, F_{17})\):
\begin{enumerate}
	\item Множество состояний автомата \(M_{17}\) получается путём объединения множеств состояний исходных автоматов
	      \begin{align*}
		      Q_{17} = Q_{15} \cup Q_{16} =  \cbr{
			      \begin{array}{l}
				      q_{10}, q_{11}, q_{20}, q_{21}, q_{90},q_{30}, q_{31}, q_{40}, q_{41}, q_{100}, q_{150}, \\
				      q_{50}, q_{51}, q_{60}, q_{61}, q_{110}, q_{70}, q_{71}, q_{80}, q_{81}, q_{120}, q_{160}
			      \end{array}}
	      \end{align*}
	\item начальным состоянием результирующего автомата \(M_{17}\) будет начальное состояние автомата
\end{enumerate}
