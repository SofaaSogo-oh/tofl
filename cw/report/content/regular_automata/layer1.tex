Для выражения \(a\) конечный автомат примет вид
\begin{align*}
	M_1 = \rbr{\cbr{q_{10}, q_{11}}, \Sigma, \delta_1, q_{10}, \cbr{q_{11}}},
\end{align*}
где множество переходов \(\delta_1\) автомата будет содержать переходы вида
\begin{align*}
	\delta_1(q_{10}, a) = \cbr{q_{11}}
\end{align*}
Граф переходов построенного КА \(M_1\) примет вид
\begin{figure}[h!]
	\centering
	\begin{tikzpicture}[
			->,
			>=stealth',
			node distance=2.5cm,
			every state/.style={thick, fill=gray!10},
			initial text={Начало}
		]

		% Состояния
		\node[state, initial] (q10) {$q_{10}$};
		\node[state, accepting, right of=q10] (q11) {$q_{11}$};

		% Переходы
		\path
		(q10) edge node[above] {a} (q11);
	\end{tikzpicture}
	\caption{Диаграмма состояний НКА \(M_1\)}
\end{figure}

Для выражения \(b\) конечный автомат примет вид
\begin{align*}
	M_2 = \rbr{\cbr{q_{20}, q_{21}}, \Sigma, \delta_2, q_{20}, \cbr{q_{21}}},
\end{align*}
где множество переходов \(\delta_2\) автомата будет содержать переходы вида
\begin{align*}
	\delta_2(q_{20}, b) = \cbr{q_{21}}
\end{align*}
Граф переходов построенного КА \(M_2\) примет вид
\begin{figure}[h!]
	\centering
	\begin{tikzpicture}[
			->,
			>=stealth',
			node distance=2.5cm,
			every state/.style={thick, fill=gray!10},
			initial text={Начало}
		]

		% Состояния
		\node[state, initial] (q20) {$q_{20}$};
		\node[state, accepting, right of=q20] (q21) {$q_{21}$};

		% Переходы
		\path
		(q20) edge node[above] {b} (q21);
	\end{tikzpicture}
	\caption{Диаграмма состояний НКА \(M_2\)}
\end{figure}

Для выражения \(a\) конечный автомат примет вид
\begin{align*}
	M_3 = \rbr{\cbr{q_{30}, q_{31}}, \Sigma, \delta_3, q_{30}, \cbr{q_{31}}},
\end{align*}
где множество переходов \(\delta_3\) автомата будет содержать переходы вида
\begin{align*}
	\delta_3(q_{30}, a) = \cbr{q_{31}}
\end{align*}
Граф переходов построенного КА \(M_3\) примет вид
\begin{figure}[h!]
	\centering
	\begin{tikzpicture}[
			->,
			>=stealth',
			node distance=2.5cm,
			every state/.style={thick, fill=gray!10},
			initial text={Начало}
		]

		% Состояния
		\node[state, initial] (q30) {$q_{30}$};
		\node[state, accepting, right of=q30] (q31) {$q_{31}$};

		% Переходы
		\path
		(q30) edge node[above] {a} (q31);
	\end{tikzpicture}
	\caption{Диаграмма состояний НКА \(M_3\)}
\end{figure}

Для выражения \(b\) конечный автомат примет вид
\begin{align*}
	M_4 = \rbr{\cbr{q_{40}, q_{41}}, \Sigma, \delta_4, q_{40}, \cbr{q_{41}}},
\end{align*}
где множество переходов \(\delta_4\) автомата будет содержать переходы вида
\begin{align*}
	\delta_4(q_{40}, b) = \cbr{q_{41}}
\end{align*}
Граф переходов построенного КА \(M_4\) примет вид
\begin{figure}[h!]
	\centering
	\begin{tikzpicture}[
			->,
			>=stealth',
			node distance=2.5cm,
			every state/.style={thick, fill=gray!10},
			initial text={Начало}
		]

		% Состояния
		\node[state, initial] (q40) {$q_{40}$};
		\node[state, accepting, right of=q40] (q41) {$q_{41}$};

		% Переходы
		\path
		(q40) edge node[above] {b} (q41);
	\end{tikzpicture}
	\caption{Диаграмма состояний НКА \(M_4\)}
\end{figure}

Для выражения \(b\) конечный автомат примет вид
\begin{align*}
	M_5 = \rbr{\cbr{q_{50}, q_{51}}, \Sigma, \delta_5, q_{50}, \cbr{q_{51}}},
\end{align*}
где множество переходов \(\delta_5\) автомата будет содержать переходы вида
\begin{align*}
	\delta_5(q_{50}, b) = \cbr{q_{51}}
\end{align*}
Граф переходов построенного КА \(M_5\) примет вид
\begin{figure}[h!]
	\centering
	\begin{tikzpicture}[
			->,
			>=stealth',
			node distance=2.5cm,
			every state/.style={thick, fill=gray!10},
			initial text={Начало}
		]

		% Состояния
		\node[state, initial] (q50) {$q_{50}$};
		\node[state, accepting, right of=q50] (q51) {$q_{51}$};

		% Переходы
		\path
		(q50) edge node[above] {b} (q51);
	\end{tikzpicture}
	\caption{Диаграмма состояний НКА \(M_5\)}
\end{figure}

Для выражения \(c\) конечный автомат примет вид
\begin{align*}
	M_6 = \rbr{\cbr{q_{60}, q_{61}}, \Sigma, \delta_6, q_{60}, \cbr{q_{61}}},
\end{align*}
где множество переходов \(\delta_5\) автомата будет содержать переходы вида
\begin{align*}
	\delta_6(q_{60}, c) = \cbr{q_{61}}
\end{align*}
Граф переходов построенного КА \(M_6\) примет вид
\begin{figure}[h!]
	\centering
	\begin{tikzpicture}[
			->,
			>=stealth',
			node distance=2.5cm,
			every state/.style={thick, fill=gray!10},
			initial text={Начало}
		]

		% Состояния
		\node[state, initial] (q60) {$q_{60}$};
		\node[state, accepting, right of=q60] (q61) {$q_{61}$};

		% Переходы
		\path
		(q60) edge node[above] {c} (q61);
	\end{tikzpicture}
	\caption{Диаграмма состояний НКА \(M_6\)}
\end{figure}

\newpage

Для выражения \(b\) конечный автомат примет вид
\begin{align*}
	M_7 = \rbr{\cbr{q_{70}, q_{71}}, \Sigma, \delta_7, q_{70}, \cbr{q_{71}}},
\end{align*}
где множество переходов \(\delta_7\) автомата будет содержать переходы вида
\begin{align*}
	\delta_7(q_{70}, b) = \cbr{q_{51}}
\end{align*}
Граф переходов построенного КА \(M_7\) примет вид
\begin{figure}[h!]
	\centering
	\begin{tikzpicture}[
			->,
			>=stealth',
			node distance=2.5cm,
			every state/.style={thick, fill=gray!10},
			initial text={Начало}
		]

		% Состояния
		\node[state, initial] (q70) {$q_{70}$};
		\node[state, accepting, right of=q70] (q71) {$q_{71}$};

		% Переходы
		\path
		(q70) edge node[above] {b} (q71);
	\end{tikzpicture}
	\caption{Диаграмма состояний НКА \(M_7\)}
\end{figure}


Для выражения \(c\) конечный автомат примет вид
\begin{align*}
	M_8 = \rbr{\cbr{q_{80}, q_{81}}, \Sigma, \delta_8, q_{80}, \cbr{q_{81}}},
\end{align*}
где множество переходов \(\delta_5\) автомата будет содержать переходы вида
\begin{align*}
	\delta_8(q_{80}, c) = \cbr{q_{81}}
\end{align*}
Граф переходов построенного КА \(M_8\) примет вид
\begin{figure}[h!]
	\centering
	\begin{tikzpicture}[
			->,
			>=stealth',
			node distance=2.5cm,
			every state/.style={thick, fill=gray!10},
			initial text={Начало}
		]

		% Состояния
		\node[state, initial] (q80) {$q_{80}$};
		\node[state, accepting, right of=q80] (q81) {$q_{81}$};

		% Переходы
		\path
		(q80) edge node[above] {c} (q81);
	\end{tikzpicture}
	\caption{Диаграмма состояний НКА \(M_8\)}
\end{figure}
