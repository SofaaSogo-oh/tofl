\begin{itemize}
	\item Построение автомата \(M_1 = (Q_1, \Sigma, \delta_1, q_1, F_1)\) для леволинейной грамматики производится следующим образом:
	      \begin{itemize}
		      \item Множество состояний состоит из именуемых нетерминалы состояний;
		      \item Добавляется новое состояние --- начальное (на наименование действуют соглашения по наименованию нетерминалов грамматик)
	      \end{itemize}
	      Таким образом
	      \begin{align*}
		      Q_1 = \aleph'_{19} \cup \cbr{H} = \cbr{H, S_9, S_{11}, S_{15}, S_{16}}
	      \end{align*}
	      Начальное состояние:
	      \begin{align*}
		      q_1 \equiv H
	      \end{align*}
	      Множество заключительных состояний содержит целевой символ исходной грамматики
	      \begin{align*}
		      F = \cbr{S_{16}}
	      \end{align*}
	      Множество переходов:
	      \begin{align*}
		      \begin{array}{lll}
			      \delta_1(S_{15}, a) = \cbr{S_9}           & \delta_1(S_{15}, b) = \cbr{S_9, S_{11}}   & \delta(S_{15}, c) = \cbr{S_{11}} \\
			      \delta_1(S_{16}, b) = \cbr{S_{11}}        & \delta_1(S_{16}, c) = \cbr{S_{11}}                                           \\
			      \delta_1(S_{9}, a) = \cbr{S_{15}, S_{16}} & \delta_1(S_{9}, b) = \cbr{S_{15}, S_{16}}                                    \\
			      \delta_1(S_{11}, b) = \cbr{S_{16}}        & \delta_1(S_{11}, c) = \cbr{S_{16}}                                           \\
			      \delta_1(H, a) = \cbr{S_{9}}              & \delta_1(H, b) = \cbr{S_{9}}
		      \end{array}
	      \end{align*}
	\item Построение автомата \(M_2 = (Q_2, \Sigma, \delta_2, q_2, F_2)\) для леволинейной грамматики производится следующим образом:
	      \begin{itemize}
		      \item Множество состояний состоит из именуемых нетерминалы состояний;
		      \item Добавляется новое состояние --- заключительное (на наименование действуют соглашения по наименованию нетерминалов грамматик)
	      \end{itemize}
	      Таким образом
	      \begin{align*}
		      Q_2 = \aleph''_{20} \cup \cbr{F} = \cbr{F, S_10, S_{12}, S_{15}, S_{16}}
	      \end{align*}
	      Начальное состояние --- состояние, соответствующее целевому символу исходной грамматики:
	      \begin{align*}
		      q_2 \equiv S_{15}
	      \end{align*}
	      Множество заключительных состояний будет содержать новое состояние
	      \begin{align*}
		      F_2 = \cbr{F}
	      \end{align*}
	      Множество переходов:
	      \begin{align*}
		      \begin{array}{lll}
			      \delta_2(S_{16}, b) = \cbr{S_{12}}            & \delta_2(S_{16}, c) = \cbr{S_{12}}            \\
			      \delta_2(S_{12}, b) = \cbr{S_{16}, F}         & \delta_2(S_{12}, c) = \cbr{S_{16}, F}         \\
			      \delta_2(S_{10}, a) = \cbr{S_{15}, S_{16}, F} & \delta_2(S_{10}, b) = \cbr{S_{15}, S_{16}, F} \\
			      \delta_2(S_{15}, a) = \cbr{S_{10}}            & \delta_2(S_{15}, b) = \cbr{S_{10}}
		      \end{array}
	      \end{align*}
\end{itemize}
На этом построение конечных автоматов по автоматным грамматикам заканчивается
