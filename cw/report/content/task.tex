Вариант 2.14
\begin{align}\label{language-for-analyze}
	L = \left\{ ((a,b)^2)^k \cdot ((b,c)^2)^m \colon \forall k > 0, m \geq 0,\, k,m \in \mathbb{Z} \right\} \\
\end{align}
\section{Определение типа языка L}
Язык \cref{language-for-analyze} является регулярным. Докажем это, пользуясь замкнутостью класса регулярных языков.
\begin{enumerate}
	\item Множества \(\{a\}, \{b\}, \{c\}\) являются регулярными по определению;
	\item Множества
	      \begin{align}
		      \{a\} \cup \{b\} =  \{a,b\} \\
		      \{b\} \cup \{c\} =  \{b,c\}
	      \end{align}
	      регулярны, так как объединение регулярных множеств --- регулярное множество
	\item Множества \begin{align}
		      S_1 = \{a, b\}\{a,b\} \\
		      S_2 = \{b, c\}\{b,c\}
	      \end{align}
	      регулярны , поскольку конкатенация регулярных множеств --- регулярное множество
	\item Множества
	      \begin{align}
		      S_1^+ = S_1 S_1^* \\
		      S_2^*
	      \end{align}
	      регулярны, посколько итерация регулярного множества --- регулярное множество и конкатенация регулярных множеств --- регулярное множество
	\item Конкатенация регулярных множеств --- регулярное множество, а потому:
	      \begin{align}
		      S_3 = S_1^+ \cdot S_2^*
	      \end{align}
	      есть регулярное множество.
\end{enumerate}
\section{Регулярный язык}
\subsection{Приведите искомого множества к регулярному виду}
Регулярное множество:
\begin{align}
	\{a, b\}\cdot\{a, b\}^*\cdot\{b, c\}^*
\end{align}
\subsection{Построение регулярного выражения для искомого регулярного множества}
\begin{align}
	p=((a+b)(a+b))^+((b+c)(b+c))^*
\end{align}
\subsection{Получение регулярной грамматики}
\subsubsection{Построение леволинейной и праволинейной грамматик}
\begin{align}
	% p=\ub{\ub{(\ub{\ub{a}_1+\ub{b}_2}_5)^+}_7\ub{(\ub{\ub{b}_3+\ub{c}_4}_6)^*}_8}_9
	% \ub{ % 9
	% 	\ub{ % 7
	% 		\rbr{%
	% 			\ub{\ub{a}_{1} + \ub{b}_{2}}_{5}%
	% 			\ub{\ub{a}_{3} + \ub{b}_{4}}_{5}%
	% 		}^{+}%
	% 	}_{7} % 7
	% 	\cdot
	% 	\ub{%
	% 		\rbr{%
	% 			\ub{\ub{b}_{3} + \ub{c}_{4}}_{6}%
	% 		}^{*}%
	% 	}_{8} % 8
	% }_{9}% 9
	% \ub{\rbr{%
	% 		\ub{a}_{1} + \ub_{b}_{2}%
	% 	}}_{9}%
	p=\ub{\ub{\rbr{\ub{
					\rbr{\ub{%
							\ub{a}_{1} + \ub{b}_{2}%
						}_{9}}
					\cdot
					\rbr{\ub{%
							\ub{a}_{3} + \ub{b}_{4}%
						}_{10}}}_{13}}^+}_{15}
		\cdot
		\ub{\rbr{\ub{\rbr{\ub{%
							\ub{b}_{5} + \ub{c}_{6}%
						}_{11}}
					\cdot
					\rbr{\ub{%
							\ub{b}_{7} + \ub{c}_{8}%
						}_{12}}}_{14}}^{*}}_{16}}_{17}
\end{align}
\begin{align*}
	G_1 = \grammatics{S_1}{\Sigma}{S_1 \to a}{S_1}, G_2 = \grammatics{S_2}{\Sigma}{S_2 \to b}{S_2} \\
	G_3 = \grammatics{S_3}{\Sigma}{S_3 \to a}{S_3}, G_4 = \grammatics{S_4}{\Sigma}{S_4 \to b}{S_4} \\
	%
	G_5 = \grammatics{S_5}{\Sigma}{S_5 \to b}{S_5}, G_6 = \grammatics{S_6}{\Sigma}{S_6 \to c}{S_6} \\
	G_7 = \grammatics{S_7}{\Sigma}{S_7 \to b}{S_7}, G_8 = \grammatics{S_8}{\Sigma}{S_8 \to c}{S_8} \\
	%
	G_9 = \grammatics{S_9, S_1, S_2}{\Sigma}{
	S_9 \to S_1|S_2                                                                                \\
	S_1 \to a                                                                                      \\
	S_2 \to b                                                                                      \\
	}{S_9},
	G_{10} = \grammatics{S_{10}, S_3, S_4}{\Sigma}{
	S_{10} \to S_3|S_4                                                                             \\
	S_3 \to a                                                                                      \\
	S_4 \to b                                                                                      \\
	}{S_{10}}                                                                                      \\
	%
	G_{11} = \grammatics{S_{11}, S_5, S_6}{\Sigma}{
	S_{11} \to S_5|S_6                                                                             \\
	S_5 \to b                                                                                      \\
	S_6 \to c                                                                                      \\
	}{S_{11}},
	G_{12} = \grammatics{S_{12}, S_7, S_8}{\Sigma}{
	S_{12} \to S_7|S_8                                                                             \\
	S_7 \to b                                                                                      \\
	S_8 \to c                                                                                      \\
	}{S_{12}}                                                                                      \\
	%
	G'_{13} = \grammatics{S_9, S_1, S_2, S_{10}, S_3, S_4}{\Sigma}{
	S_9 \to S_1|S_2                                                                                \\
	S_1 \to a, S_2 \to b                                                                           \\
	S_{10} \to S_3|S_4                                                                             \\
	S_3 \to S_9a                                                                                   \\
	S_4 \to S_9b                                                                                   \\
	}{S_{10}}
	G''_{13} = \grammatics{S_9, S_1, S_2, S_{10}, S_3, S_4}{\Sigma}{
	S_9 \to S_1|S_2                                                                                \\
	S_1 \to aS_{10}                                                                                \\
	S_2 \to bS_{10}                                                                                \\
	S_{10} \to S_3|S_4                                                                             \\
	S_3 \to a, S_4 \to b                                                                           \\
	}{S_9}                                                                                         \\
	%
\end{align*}
\begin{align*}
	G'_{14} = \grammatics{S_{11}, S_5, S_6, S_{12}, S_7, S_8}{\Sigma}{
	S_{11} \to S_5|S_6                                      \\
	S_5 \to b , S_6 \to c                                   \\
	S_{12} \to S_7|S_8                                      \\
	S_7 \to S_{11}b                                         \\
	S_8 \to S_{11}c                                         \\
	}{S_{12}},
	G''_{14} = \grammatics{S_{11}, S_5, S_6, S_{12}, S_7, S_8}{\Sigma}{
	S_{11} \to S_5|S_6                                      \\
	S_5 \to bS_{12}                                         \\
	S_6 \to cS_{12}                                         \\
	S_{12} \to S_7|S_8                                      \\
	S_7 \to b , S_8 \to c                                   \\
	}{S_{11}}                                               \\
	%
	G'_{15} = \grammatics{S_9, S_1, S_2, S_{10}, S_3, S_4, S_{15}}{\Sigma}{
	S_9 \to S_1|S_2                                         \\
	S_1 \to S_{15}a|a                                       \\
	S_2 \to S_{15}b|b                                       \\
	S_{10} \to S_3|S_4                                      \\
	S_3 \to S_9a                                            \\
	S_4 \to S_9b                                            \\
	S_{15} \to S_{10}                                       \\
	}{S_{15}},
	G''_{15} = \grammatics{S_9, S_1, S_2, S_{10}, S_3, S_4, S_{15}}{\Sigma}{
	S_9 \to S_1|S_2                                         \\
	S_1 \to aS_{10}                                         \\
	S_2 \to bS_{10}                                         \\
	S_{10} \to S_3|S_4                                      \\
	S_3 \to aS_{15}|a                                       \\
	S_4 \to bS_{15}|b                                       \\
	S_{15} \to S_{9}                                        \\
	}{S_{15}}                                               \\
	%
	G'_{16} = \grammatics{S_{11}, S_5, S_6, S_{12}, S_7, S_8, S_{16}}{\Sigma}{
	S_{11} \to S_5|S_6                                      \\
	S_5 \to S_{16}b|b                                       \\
	S_6 \to S_{16}c|c                                       \\
	S_{12} \to S_7|S_8                                      \\
	S_7 \to S_{11}b                                         \\
	S_8 \to S_{11}c                                         \\
	S_{16} \to S_{12}|\varepsilon                           \\
	}{S_{16}}
	G''_{16} = \grammatics{S_{11}, S_5, S_6, S_{12}, S_7, S_8, S_{16}}{\Sigma}{
	S_{11} \to S_5|S_6                                      \\
	S_5 \to bS_{12}                                         \\
	S_6 \to cS_{12}                                         \\
	S_{12} \to S_7|S_8                                      \\
	S_7 \to bS_{16}|b                                       \\
	S_8 \to cS_{16}|c                                       \\
	S_{16} \to S_{11}|\varepsilon                           \\
	}{S_{16}}                                               \\
	%
	G'_{17} = \grammatics{S_9, S_1, S_2, S_{10}, S_3, S_4, S_{15}, S_{11}, S_5, S_6, S_{12}, S_7, S_8, S_{16}}{\Sigma}{
	S_9 \to S_1|S_2         & S_{11} \to S_5|S_6            \\
	S_1 \to S_{15}a|a       & S_5 \to S_{16}b|S_{15}b       \\
	S_2 \to S_{15}b|b       & S_6 \to S_{16}c|S_{15}c       \\
	S_{10} \to S_3|S_4      & S_{12} \to S_7|S_8            \\
	S_3 \to S_9a            & S_7 \to S_{11}b               \\
	S_4 \to S_9b            & S_8 \to S_{11}c               \\
	S_{15} \to S_{10}       & S_{16} \to S_{12}|\varepsilon \\
	}{S_{16}}                                               \\
	G''_{17} = \grammatics{S_9, S_1, S_2, S_{10}, S_3, S_4, S_{15}, S_{11}, S_5, S_6, S_{12}, S_7, S_8, S_{16}}{\Sigma}{
	S_9 \to S_1|S_2         & S_{11} \to S_5|S_6            \\
	S_1 \to aS_{10}         & S_5 \to bS_{12}               \\
	S_2 \to bS_{10}         & S_6 \to cS_{12}               \\
	S_{10} \to S_3|S_4      & S_{12} \to S_7|S_8            \\
	S_3 \to aS_{15}|aS_{16} & S_7 \to bS_{16}|b             \\
	S_4 \to bS_{15}|bS_{16} & S_8 \to cS_{16}|c             \\
	S_{15} \to S_{9}        & S_{16} \to S_{11}|\varepsilon \\
	}{S_{15}}
\end{align*}
\subsubsubsection{ Проверка пустоты}
\begin{itemize}
	\item Для леволинейной грамматики \(G'_{17}\)
	      \begin{align*}
		      C_0    & = \varnothing                                                                                           \\
		      C_{2}  & = \cbr{S_9} \cup C_{1} = \cbr{S_1, S_2, S_9}                                                            \\
		      C_{3}  & = \cbr{S_3, S_4, S_9} \cup C_{2} = \cbr{S_1, S_2, S_3, S_4, S_9}                                        \\
		      C_{4}  & = \cbr{S_3, S_4, S_9, S_{10}} \cup C_{3} = \cbr{S_1, S_2, S_3, S_4, S_9, S_{10}}                        \\
		      C_{5}  & = \cbr{S_3, S_4, S_9, S_{10}, S_{15}} \cup C_{4} = \cbr{S_1, S_2, S_3, S_4, S_9, S_{10}, S_{15}}        \\
		      C_{6}  & = \cbr{S_1, S_2, S_3, S_4, S_5, S_6, S_9, S_{10}, S_{15}} \cup C_{5}                                    \\
		             & = \cbr{S_1, S_2, S_3, S_4, S_5, S_6, S_9, S_{10}, S_{15}}                                               \\
		      C_{7}  & = \cbr{S_1, S_2, S_3, S_4, S_5, S_6, S_9, S_{10}, S_{11}, S_{15}} \cup C_{6}                            \\
		             & = \cbr{S_1, S_2, S_3, S_4, S_5, S_6, S_9, S_{10}, S_{11}, S_{15}}                                       \\
		      C_{8}  & = \cbr{S_1, S_2, S_3, S_4, S_5, S_6, S_7, S_8, S_9, S_{10}, S_{11}, S_{15}} \cup C_{7}                  \\
		             & = \cbr{S_1, S_2, S_3, S_4, S_5, S_6, S_7, S_8, S_9, S_{10}, S_{11}, S_{15}}                             \\
		      C_{9}  & = \cbr{S_1, S_2, S_3, S_4, S_5, S_6, S_7, S_8, S_9, S_{10}, S_{11}, S_{12}, S_{15}} \cup C_{8}          \\
		             & = \cbr{S_1, S_2, S_3, S_4, S_5, S_6, S_7, S_8, S_9, S_{10}, S_{11}, S_{12}, S_{15}}                     \\
		      C_{10} & = \cbr{S_1, S_2, S_3, S_4, S_5, S_6, S_7, S_8, S_9, S_{10}, S_{11}, S_{12}, S_{15}, S_{16}} \cup C_{9}  \\
		             & = \cbr{S_1, S_2, S_3, S_4, S_5, S_6, S_7, S_8, S_9, S_{10}, S_{11}, S_{12}, S_{15}, S_{16}}             \\
		      C_{11} & = \cbr{S_1, S_2, S_3, S_4, S_5, S_6, S_7, S_8, S_9, S_{10}, S_{11}, S_{12}, S_{15}, S_{16}} \cup C_{10} \\
		             & = \cbr{S_1, S_2, S_3, S_4, S_5, S_6, S_7, S_8, S_9, S_{10}, S_{11}, S_{12}, S_{15}, S_{16}}
	      \end{align*}
	      Так как
	      \begin{align}
		      S = S_{16} \in C_{11} \Longrightarrow L(G'_{17}) \not= \varnothing
	      \end{align}
	\item Для праволинейной грамматики \(G''_{17}\)
	      \begin{align*}
		      C_{0} & = \varnothing                                                                                                                        \\
		      C_{1} & = \cbr{S_{7}, S_{8}, S_{16}} \cup C_{0} = \cbr{S_{7}, S_{8}, S_{16}}                                                                 \\
		      C_{2} & = \cbr{S_3, S_4, S_7, S_8, S_{12}, S_{16}} \cup C_{1} = \cbr{S_3, S_4, S_7, S_8, S_{12}, S_{16}}                                     \\
		      C_{3} & = \cbr{S_3, S_4, S_5, S_6, S_7, S_8, S_{10}, S_{12}, S_{16}} \cup C_{2} = \cbr{S_3, S_4, S_5, S_6, S_7, S_8, S_{10}, S_{12}, S_{16}} \\
		      C_{4} & = \cbr{S_1, S_2, S_3, S_4, S_5, S_6, S_7, S_8, S_{10}, S_{11}, S_{12}, S_{16}} \cup C_{3}                                            \\
		            & = \cbr{S_1, S_2, S_3, S_4, S_5, S_6, S_7, S_8, S_{10}, S_{11}, S_{12}, S_{16}}                                                       \\
		      C_{5} & = \cbr{S_1, S_2, S_3, S_4, S_5, S_6, S_7, S_8, S_9, S_{10}, S_{11}, S_{12}, S_{16}} \cup C_{4}                                       \\
		            & = \cbr{S_1, S_2, S_3, S_4, S_5, S_6, S_7, S_8, S_9, S_{10}, S_{11}, S_{12}, S_{16}}                                                  \\
		      C_{6} & = \cbr{S_1, S_2, S_3, S_4, S_5, S_6, S_7, S_8, S_9, S_{10}, S_{11}, S_{12}, S_{15}, S_{16}} \cup C_{5}                               \\
		            & = \cbr{S_1, S_2, S_3, S_4, S_5, S_6, S_7, S_8, S_9, S_{10}, S_{11}, S_{12}, S_{15}, S_{16}}                                          \\
		      C_{7} & = \cbr{S_1, S_2, S_3, S_4, S_5, S_6, S_7, S_8, S_9, S_{10}, S_{11}, S_{12}, S_{15}, S_{16}} \cup C_{6}                               \\
		            & = \cbr{S_1, S_2, S_3, S_4, S_5, S_6, S_7, S_8, S_9, S_{10}, S_{11}, S_{12}, S_{15}, S_{16}}
	      \end{align*}
	      Так как
	      \begin{align}
		      S = S_{15} \in C_6 \Longrightarrow L(G''_{17}) \not= \varnothing
	      \end{align}
\end{itemize}

