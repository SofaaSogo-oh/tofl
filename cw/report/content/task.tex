Вариант 2.14
\begin{align}\label{language-for-analyze}
	L = \left\{ ((a,b)^2)^k \cdot ((b,c)^2)^m \colon \forall k > 0, m \geq 0,\, k,m \in \mathbb{Z} \right\} \\
\end{align}
\section{Определение типа языка L}
Язык \cref{language-for-analyze} является регулярным. Докажем это, пользуясь замкнутостью класса регулярных языков.
\begin{enumerate}
	\item Множества \(\{a\}, \{b\}, \{c\}\) являются регулярными по определению;
	\item Множества
	      \begin{align}
		      \{a\} \cup \{b\} =  \{a,b\} \\
		      \{b\} \cup \{c\} =  \{b,c\}
	      \end{align}
	      регулярны, так как объединение регулярных множеств --- регулярное множество
	\item Множества \begin{align}
		      S_1 = \{a, b\}\{a,b\} \\
		      S_2 = \{b, c\}\{b,c\}
	      \end{align}
	      регулярны , поскольку конкатенация регулярных множеств --- регулярное множество
	\item Множества
	      \begin{align}
		      S_1^+ = S_1 S_1^* \\
		      S_2^*
	      \end{align}
	      регулярны, посколько итерация регулярного множества --- регулярное множество и конкатенация регулярных множеств --- регулярное множество
	\item Конкатенация регулярных множеств --- регулярное множество, а потому:
	      \begin{align}
		      S_3 = S_1^+ \cdot S_2^*
	      \end{align}
	      есть регулярное множество.
\end{enumerate}
\section{Регулярный язык}
\subsection{Приведите искомого множества к регулярному виду}
Регулярное множество:
\begin{align}
	\{a, b\}\cdot\{a, b\}^*\cdot\{b, c\}^*
\end{align}
\subsection{Построение регулярного выражения для искомого регулярного множества}
\begin{align}
	p=((a+b)(a+b))^+((b+c)(b+c))^*
\end{align}
\subsection{Получение регулярной грамматики}
\subsubsection{Построение леволинейной и праволинейной грамматик}
\begin{align}
	% p=\ub{\ub{(\ub{\ub{a}_1+\ub{b}_2}_5)^+}_7\ub{(\ub{\ub{b}_3+\ub{c}_4}_6)^*}_8}_9
	% \ub{ % 9
	% 	\ub{ % 7
	% 		\rbr{%
	% 			\ub{\ub{a}_{1} + \ub{b}_{2}}_{5}%
	% 			\ub{\ub{a}_{3} + \ub{b}_{4}}_{5}%
	% 		}^{+}%
	% 	}_{7} % 7
	% 	\cdot
	% 	\ub{%
	% 		\rbr{%
	% 			\ub{\ub{b}_{3} + \ub{c}_{4}}_{6}%
	% 		}^{*}%
	% 	}_{8} % 8
	% }_{9}% 9
	% \ub{\rbr{%
	% 		\ub{a}_{1} + \ub_{b}_{2}%
	% 	}}_{9}%
	\ub{\ub{\rbr{\ub{
					\rbr{\ub{%
							\ub{a}_{1} + \ub{b}_{2}%
						}_{9}}
					\cdot
					\rbr{\ub{%
							\ub{a}_{3} + \ub{b}_{4}%
						}_{10}}}_{13}}^+}_{15}
		\cdot
		\ub{\rbr{\ub{\rbr{\ub{%
							\ub{b}_{5} + \ub{c}_{6}%
						}_{11}}
					\cdot
					\rbr{\ub{%
							\ub{a}_{7} + \ub{b}_{8}%
						}_{12}}}_{14}}^{*}}_{16}}_{17}
\end{align}
\begin{align*}
	G_1 = \grammatics{S_1}{\Sigma}{S_1 \to a}{S_1}                      \\
	G_2 = \grammatics{S_2}{\Sigma}{S_2 \to b}{S_2}                      \\
	G_2 = \grammatics{S_2}{\Sigma}{S_2 \to b}{S_2}                      \\
	G_4 = \grammatics{S_4}{\Sigma}{S_4 \to c}{S_4}                      \\
	G_5 = \grammatics{S_1, S_2, S_5}{\Sigma}{S_5 \to S_1\vert S_2}{S_5} \\
	G_6 = \grammatics{S_3, S_4, S_6}{\Sigma}{S_6 \to S_3\vert S_4}{S_6} \\
\end{align*}
