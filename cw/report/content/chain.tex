\subsubsubsection{ Удаление цепных правил}
\begin{itemize}
	\item Строим последовательность множеств \(\aleph_i^X\) для леволинейной грамматики \(G'_{17}\)
	      \begin{align*}
		      \setconc{
		      \aleph^{S_0}_{0}  = \cbr{S_0}                                                   \\
		      \aleph^{S_0}_{1}  = \cbr{S_0}                                                   \\
		      }{\aleph^{S_0}  = \varnothing}
		      \setconc{
		      \aleph^{S_1}_{0}  = \cbr{S_1}                                                   \\
		      \aleph^{S_1}_{1}  = \cbr{S_1}                                                   \\
		      }{\aleph^{S_1}  = \varnothing}                                                  \\
		      \setconc{
		      \aleph^{S_2}_{0}  = \cbr{S_2}                                                   \\
		      \aleph^{S_2}_{1}  = \cbr{S_2}                                                   \\
		      }{\aleph^{S_2}  = \varnothing}
		      \setconc{
		      \aleph^{S_3}_{0}  = \cbr{S_3}                                                   \\
		      \aleph^{S_3}_{1}  = \cbr{S_3}                                                   \\
		      }{\aleph^{S_3}  = \varnothing}                                                  \\
		      \setconc{
		      \aleph^{S_4}_{0}  = \cbr{S_4}                                                   \\
		      \aleph^{S_4}_{1}  = \cbr{S_4}                                                   \\
		      }{\aleph^{S_4}  = \varnothing}
		      \setconc{
		      \aleph^{S_5}_{0}  = \cbr{S_5}                                                   \\
		      \aleph^{S_5}_{1}  = \cbr{S_5}                                                   \\
		      }{\aleph^{S_5}  = \varnothing}                                                  \\
		      \setconc{
		      \aleph^{S_6}_{0}  = \cbr{S_6}                                                   \\
		      \aleph^{S_6}_{1}  = \cbr{S_6}                                                   \\
		      }{\aleph^{S_6}  = \varnothing}
		      \setconc{
		      \aleph^{S_7}_{0}  = \cbr{S_7}                                                   \\
		      \aleph^{S_7}_{1}  = \cbr{S_7}                                                   \\
		      }{\aleph^{S_7}  = \varnothing}                                                  \\
		      \setconc{
		      \aleph^{S_8}_{0}  = \cbr{S_8}                                                   \\
		      \aleph^{S_8}_{1}  = \cbr{S_8}                                                   \\
		      }{\aleph^{S_8}  = \varnothing}                                                  \\
		      \setconc{
		      \aleph^{S_9}_{0}  = \cbr{S_9}                                                   \\
		      \aleph^{S_9}_{1}  = \cbr{S_1, S_2, S_9}                                         \\
		      \aleph^{S_9}_{2}  = \cbr{S_1, S_2, S_9}                                         \\
		      }{\aleph^{S_9}  = \cbr{S_1, S_2}}                                               \\
		      \setconc{
		      \aleph^{S_{10}}_{0}  = \cbr{S_{10}}                                             \\
		      \aleph^{S_{10}}_{1}  = \cbr{S_3, S_4, S_{10}}                                   \\
		      \aleph^{S_{10}}_{2}  = \cbr{S_3, S_4, S_{10}}                                   \\
		      }{\aleph^{S_{10}}  = \cbr{S_3, S_4}}                                            \\
		      \setconc{
		      \aleph^{S_{11}}_{0}  = \cbr{S_{11}}                                             \\
		      \aleph^{S_{11}}_{1}  = \cbr{S_5, S_6, S_{11}}                                   \\
		      \aleph^{S_{11}}_{2}  = \cbr{S_5, S_6, S_{11}}                                   \\
		      }{\aleph^{S_{11}}  = \cbr{S_5, S_6}}                                            \\
		      \setconc{
		      \aleph^{S_{12}}_{0}  = \cbr{S_{12}}                                             \\
		      \aleph^{S_{12}}_{1}  = \cbr{S_7, S_8, S_{12}}                                   \\
		      \aleph^{S_{12}}_{2}  = \cbr{S_7, S_8, S_{12}}                                   \\
		      }{\aleph^{S_{12}}  = \cbr{S_7, S_8}}                                            \\
		      \setconc{
		      \aleph^{S_{15}}_{0}  = \cbr{S_{15}}                                             \\
		      \aleph^{S_{15}}_{1}  = \cbr{S_{10}, S_{15}}                                     \\
		      \aleph^{S_{15}}_{2}  = \cbr{S_3, S_4, S_{10}, S_{15}}                           \\
		      \aleph^{S_{15}}_{3}  = \cbr{S_3, S_4, S_{10}, S_{15}}                           \\
		      }{\aleph^{S_{15}}  = \cbr{S_3, S_4, S_{10}}}                                    \\
		      \setconc{
		      \aleph^{S_{16}}_{0}  = \cbr{S_{16}}                                             \\
		      \aleph^{S_{16}}_{1}  = \cbr{S_{12}, S_{15}, S_{16}}                             \\
		      \aleph^{S_{16}}_{2}  = \cbr{S_7, S_8, S_{10}, S_{12}, S_{15}, S_{16}}           \\
		      \aleph^{S_{16}}_{3}  = \cbr{S_3, S_4, S_7, S_8, S_{10}, S_{12}, S_{15}, S_{16}} \\
		      \aleph^{S_{16}}_{4}  = \cbr{S_3, S_4, S_7, S_8, S_{10}, S_{12}, S_{15}, S_{16}} \\
		      }{\aleph^{S_{16}}  = \cbr{S_3, S_4, S_7, S_8, S_{10}, S_{12}, S_{15}}}          \\
	      \end{align*}
	      Множество правил \(P'_{18}\) содержит все правила грамматики \(G'_{17}\) кроме цепных:
	      \begin{align*}
		      P'_{18} = \cbr{
			      \begin{array}{ll}
				      S_1 \to S_{15}a|a & S_5 \to S_{16}b|S_{15}b \\
				      S_2 \to S_{15}b|b & S_6 \to S_{16}c|S_{15}c \\
				      S_3 \to S_9a      & S_7 \to S_{11}b         \\
				      S_4 \to S_9b      & S_8 \to S_{11}c         \\
			      \end{array}
		      }
	      \end{align*}
	      С добавлением новых правил, опираясь на соотношение вида
	      \begin{align*}
		      P'_{18} = P'_{18} \cup \cbr{(B \to \alpha) | \forall (A \to \alpha) \in P, A \in \aleph^B},
	      \end{align*}
	      то есть
	      \begin{align*}
		      P'_{18} = P'_{18} \cup \cbr{
			      \begin{array}{ll}
				      S_9 \to S_{15}a|a|S_{15}b|b                & S_{10} \to S_9a|S_9b                 \\
				      S_{11} \to S_{16}b|S_{15}b|S_{16}c|S_{15}c & S_{12} \to S_{11}b|S_{11}c           \\
				      S_{15} \to S_9a|S_9b                       & S_{16} \to S_{11}b|S_{11}c|S_9a|S_9b
			      \end{array}
		      }
	      \end{align*}
	      Таким образом, результирующая грамматика \(G'_{18}\) примет следующий вид
	      \begin{align*}
		      G'_{18} = \grammatics{S_9, S_1, S_2, S_{10}, S_3, S_4, S_{15}, S_{11}, S_5, S_6, S_{12}, S_7, S_8, S_{16}}{\Sigma}{
		      S_1 \to S_{15}a|a                          & S_5 \to S_{16}b|S_{15}b              \\
		      S_2 \to S_{15}b|b                          & S_6 \to S_{16}c|S_{15}c              \\
		      S_3 \to S_9a                               & S_7 \to S_{11}b                      \\
		      S_4 \to S_9b                               & S_8 \to S_{11}c                      \\
		      S_9 \to S_{15}a|a|S_{15}b|b                & S_{10} \to S_9a|S_9b                 \\
		      S_{11} \to S_{16}b|S_{15}b|S_{16}c|S_{15}c & S_{12} \to S_{11}b|S_{11}c           \\
		      S_{15} \to S_9a|S_9b                       & S_{16} \to S_{11}b|S_{11}c|S_9a|S_9b
		      }{S_{16}}                                                                         \\
	      \end{align*}
	\item Строим последовательность множеств \(\aleph_i^X\) для праволинейной грамматики \(G''_{18}\)
	      \begin{align*}
		      \setconc{
		      \aleph^{S_0}_{0}     = \cbr{S_0}                      \\
		      \aleph^{S_0}_{1}     = \cbr{S_0}                      \\
		      }{\aleph^{S_0}       = \varnothing}
		      \setconc{
		      \aleph^{S_1}_{0}     = \cbr{S_1}                      \\
		      \aleph^{S_1}_{1}     = \cbr{S_1}                      \\
		      }{\aleph^{S_1}       = \varnothing}                   \\
		      \setconc{
		      \aleph^{S_2}_{0}     = \cbr{S_2}                      \\
		      \aleph^{S_2}_{1}     = \cbr{S_2}                      \\
		      }{\aleph^{S_2}       = \varnothing}
		      \setconc{
		      \aleph^{S_3}_{0}     = \cbr{S_3}                      \\
		      \aleph^{S_3}_{1}     = \cbr{S_3}                      \\
		      }{\aleph^{S_3}       = \varnothing}                   \\
		      \setconc{
		      \aleph^{S_4}_{0}     = \cbr{S_4}                      \\
		      \aleph^{S_4}_{1}     = \cbr{S_4}                      \\
		      }{\aleph^{S_4}       = \varnothing}
		      \setconc{
		      \aleph^{S_5}_{0}     = \cbr{S_5}                      \\
		      \aleph^{S_5}_{1}     = \cbr{S_5}                      \\
		      }{\aleph^{S_5}       = \varnothing}                   \\
		      \setconc{
		      \aleph^{S_6}_{0}     = \cbr{S_6}                      \\
		      \aleph^{S_6}_{1}     = \cbr{S_6}                      \\
		      }{\aleph^{S_6}       = \varnothing}
		      \setconc{
		      \aleph^{S_7}_{0}     = \cbr{S_7}                      \\
		      \aleph^{S_7}_{1}     = \cbr{S_7}                      \\
		      }{\aleph^{S_7}       = \varnothing}                   \\
		      \setconc{
		      \aleph^{S_8}_{0}     = \cbr{S_8}                      \\
		      \aleph^{S_8}_{1}     = \cbr{S_8}                      \\
		      }{\aleph^{S_8}       = \varnothing}                   \\
		      \setconc{
		      \aleph^{S_9}_{0}     = \cbr{S_9}                      \\
		      \aleph^{S_9}_{1}     = \cbr{S_1, S_2, S_9}            \\
		      \aleph^{S_9}_{2}     = \cbr{S_1, S_2, S_9}            \\
		      }{\aleph^{S_9}       = \cbr{S_1, S_2}}                \\
		      \setconc{
		      \aleph^{S_{10}}_{0}  = \cbr{S_{10}}                   \\
		      \aleph^{S_{10}}_{1}  = \cbr{S_3, S_4, S_{10}}         \\
		      \aleph^{S_{10}}_{2}  = \cbr{S_3, S_4, S_{10}}         \\
		      }{\aleph^{S_{10}}    = \cbr{S_3, S_4}}                \\
		      \setconc{
		      \aleph^{S_{11}}_{0}  = \cbr{S_{11}}                   \\
		      \aleph^{S_{11}}_{1}  = \cbr{S_5, S_6, S_{11}}         \\
		      \aleph^{S_{11}}_{2}  = \cbr{S_5, S_6, S_{11}}         \\
		      }{\aleph^{S_{11}}    = \cbr{S_5, S_6}}                \\
		      \setconc{
		      \aleph^{S_{12}}_{0}  = \cbr{S_{12}}                   \\
		      \aleph^{S_{12}}_{1}  = \cbr{S_7, S_8, S_{12}}         \\
		      \aleph^{S_{12}}_{2}  = \cbr{S_7, S_8, S_{12}}         \\
		      }{\aleph^{S_{12}}    = \cbr{S_7, S_8}}                \\
		      \setconc{
		      \aleph^{S_{15}}_{0}  = \cbr{S_{15}}                   \\
		      \aleph^{S_{15}}_{1}  = \cbr{S_9, S_{15}}              \\
		      \aleph^{S_{15}}_{2}  = \cbr{S_1, S_2, S_9, S_{15}}    \\
		      \aleph^{S_{15}}_{3}  = \cbr{S_1, S_2, S_9, S_{15}}    \\
		      }{\aleph^{S_{15}}    = \cbr{S_1, S_2, S_9}}           \\
		      \setconc{
		      \aleph^{S_{16}}_{0}  = \cbr{S_{16}}                   \\
		      \aleph^{S_{16}}_{1}  = \cbr{S_{11}, S_{16}}           \\
		      \aleph^{S_{16}}_{2}  = \cbr{S_5, S_6, S_{11}, S_{16}} \\
		      \aleph^{S_{16}}_{3}  = \cbr{S_5, S_6, S_{11}, S_{16}} \\
		      }{\aleph^{S_{16}} = \cbr{S_5, S_6, S_{11}}}           \\
	      \end{align*}
	      Множество правил \(P''_{19}\) содержит все правила грамматики \(G''_{18}\) кроме цепных:
	      \begin{align*}
		      P''_{19} = \cbr{
			      \begin{array}{ll}
				      S_1 \to aS_{10}           & S_5 \to bS_{12}   \\
				      S_2 \to bS_{10}           & S_6 \to cS_{12}   \\
				      S_3 \to aS_{15}|aS_{16}|a & S_7 \to bS_{16}|b \\
				      S_4 \to bS_{15}|bS_{16}|b & S_8 \to cS_{16}|c \\
			      \end{array}
		      }
	      \end{align*}
	      С добавлением новых правил, опираясь на соотношение вида
	      \begin{align*}
		      P''_{19} = P''_{19} \cup \cbr{(B \to \alpha) | \forall (A \to \alpha) \in P, A \in \aleph^B},
	      \end{align*}
	      то есть
	      \begin{align*}
		      P''_{19} = P''_{19} \cup \cbr{
			      \begin{array}{ll}
				      S_9 \to aS_{10}|bS_{10}    & S_{10} \to aS_{15}|aS_{16}|a|bS_{15}|bS_{16}|b \\
				      S_{11} \to bS_{12}|cS_{12} & S_{12} \to bS_{16}|b|cS_{16}|c                 \\
				      S_{15} \to aS_{10}|bS_{10} & S_{16} \to bS_{12}|cS_{12}
			      \end{array}
		      }
	      \end{align*}
	      Таким образом, результирующая грамматика \(G'_{18}\) примет следующий вид
	      \begin{align*}
		      G''_{19} = \grammatics{S_9, S_1, S_2, S_{10}, S_3, S_4, S_{15}, S_{11}, S_5, S_6, S_{12}, S_7, S_8, S_{16}}{\Sigma}{
		      S_1 \to aS_{10}            & S_5 \to bS_{12}                                \\
		      S_2 \to bS_{10}            & S_6 \to cS_{12}                                \\
		      S_3 \to aS_{15}|aS_{16}|a  & S_7 \to bS_{16}|b                              \\
		      S_4 \to bS_{15}|bS_{16}|b  & S_8 \to cS_{16}|c                              \\
		      S_9 \to aS_{10}|bS_{10}    & S_{10} \to aS_{15}|aS_{16}|a|bS_{15}|bS_{16}|b \\
		      S_{11} \to bS_{12}|cS_{12} & S_{12} \to bS_{16}|b|cS_{16}|c                 \\
		      S_{15} \to aS_{10}|bS_{10} & S_{16} \to bS_{12}|cS_{12}
		      }{S_{15}}                                                                   \\
	      \end{align*}
\end{itemize}
Так как при удалении пустых правил и цепных правил лево- и праволинейной грамматик произошло их изменение, то необходимо повторить удаление бесполезных и недостижимых символов.
