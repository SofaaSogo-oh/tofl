\begin{itemize}
	\item Для леволинейной грамматики \(G'_{17}\)
	      \begin{align*}
		      C_{0} & = \varnothing                                  \\
		      C_{1} & = \varnothing \cup C_{0} = \varnothing = C_{0}
	      \end{align*}
	      Пустых правил нет, следовательно, грамматика \(G'_{17}\) не поменялась.
	\item Для праволинейной грамматики \(G''_{17}\)
	      \begin{align*}
		      C_{0} & = \cbr{S_{16}}                          \\
		      C_{1} & = \varnothing \cup C_{0} = \cbr{S_{16}}
	      \end{align*}
	      Итоговая грамматика \(G''_{18}\) без пустых правил и после добавления новых примет вид
	      \begin{align*}
		      G''_{18} = \grammatics{S_9, S_1, S_2, S_{10}, S_3, S_4, S_{15}, S_{11}, S_5, S_6, S_{12}, S_7, S_8, S_{16}}{\Sigma}{
		      S_9 \to S_1|S_2           & S_{11} \to S_5|S_6 \\
		      S_1 \to aS_{10}           & S_5 \to bS_{12}    \\
		      S_2 \to bS_{10}           & S_6 \to cS_{12}    \\
		      S_{10} \to S_3|S_4        & S_{12} \to S_7|S_8 \\
		      S_3 \to aS_{15}|aS_{16}|a & S_7 \to bS_{16}|b  \\
		      S_4 \to bS_{15}|bS_{16}|b & S_8 \to cS_{16}|c  \\
		      S_{15} \to S_{9}          & S_{16} \to S_{11}  \\
		      }{S_{15}}
	      \end{align*}
\end{itemize}
