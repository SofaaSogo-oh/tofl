\subsection{Построение детерменированных конечных автоматов для НКА \(M_1\), \(M_2\), \(M_3\)}

\begin{itemize}
	\item Ножество состояний \(Q'\) результирующего автомата ДКА состояит из всех подмножеств \(Q\) исходного автомата. Каждое состояние \(Q'\) обозначается как \([A_1,\dots,A_n]\), где \(A_i \in Q\). Тогда получаем число различных сочетаний
	      \begin{align*}
		      |Q'| = \sum_{k=1}^n C^k_n = 2^n - 1
	      \end{align*}
	\item Начальное состояние имеет вид (\(H\) --- начальное состояние автомата \(M\))
	      \begin{align*}
		      q'_0  \equiv [H]
	      \end{align*}
	\item Множество конечных состояний (конечные состояния исходного автомата \(F=\cbr{F_1, \dots, F_n}\)) имеет вид
	      \begin{align*}
		      F' = \cbr{[q_1, \dots, q_j,q_{j+1},\dots,q_{j+k} ]}    \\
		      \cbr{q_1, \dots, q_j} \subset \cbr{H, F_1, \dots, F_n} \\
		      \cbr{q_{j+1},\dots,q_j} \subset Q \setminus \cbr{H, F_1, \dots, F_n}
	      \end{align*}
\end{itemize}
\subsubsection{ДКА для \(M_1\)}
Переходы определяются как
\begin{align*}
	\begin{array}{ll}
		\gendelta{S_9}{a}{S_{15}}{\maqa}                               \\
		\gendelta{H}{a}{S_9}{\maqa,S_{15}}                             \\
		\gendelta{S_{15}}{a}{S_9}{\maqa}                               \\
		\gendelta{S_9S_{15}}{a}{S_9S_{15}}{\maqa,H}                    \\
		\gendelta{S_9H}{a}{S_9S_{15}}{\maqa}                           \\
		\gendelta{S_{11}}{c}{S_{16}}{\maqc}                            \\
		\gendelta{S_{15}}{c}{S_{11}}{\maqc, H, S_{16}}                 \\
		\gendelta{H}{c}{S_{11}}{\maqc, S_{16}}                         \\
		\gendelta{S_{16}}{c}{S_{11}}{\maqc}                            \\
		\gendelta{HS_{11}}{c}{S_{11}S_{16}}{\maqc,S_{15},S_{16}}       \\
		\gendelta{S_{15}S_{11}}{c}{S_{11}S_{16}}{\maqc,S_{16}}         \\
		\gendelta{S_{16}S_{11}}{c}{S_{11}S_{16}}{\maqc}                \\
		\gendelta{S_{11}}{b}{S_{16}}{}                                 \\
		\gendelta{S_{9}}{b}{S_{15}}{}                                  \\
		\gendelta{H}{b}{S_9S_{11}}{S_{15},S_{16}}                      \\
		\gendelta{S_{15}}{b}{S_9S_{11}}{S_{16}}                        \\
		\gendelta{S_{16}}{b}{S_{11}}{}                                 \\
		\gendelta{S_9H}{b}{S_9S_{11}S_{15}}{S_{15},S_{16}}             \\
		\gendelta{S_9S_{15}}{b}{S_9S_{11}S_{15}}{S_{16}}               \\
		\gendelta{S_{11}H}{b}{S_9S_{11}S_{16}}{S_{15},S_{16}}          \\
		\gendelta{S_{11}S_{15}}{b}{S_9S_{11}S_{16}}{S_{16}}            \\
		\gendelta{S_9S_{16}}{b}{S_{15}S_{11}}{}                        \\
		\gendelta{S_9S_{11}H}{b}{S_9S_{15}S_{11}S_{16}}{S_{16},S_{15}} \\
		\gendelta{S_9S_{11}S_{15}}{b}{S_9S_{15}S_{11}S_{16}}{S_{16}}   \\
		\gendelta{S_9S_{11}}{b}{S_{15}S_{16}}{}
	\end{array}
\end{align*}

\subsubsection{ДКА для \(M_2\)}
Переходы определяются как
\begin{align*}
	\begin{array}{ll}
		\gendelta{S_{10}}{a}{S_{15}S_{16}}{\mbqa, S_{16}}                     \\
		\gendelta{S_{15}}{a}{S_{10}}{\mbqa, S_{16}}                           \\
		\gendelta{S_{10}S_{15}}{a}{S_{10}S_{15}S_{16}}{\mbqa, S_{16}}         \\
		\gendelta{S_{15}}{c}{S_{12}}{\mbqc, S_{16}, F}                        \\
		\gendelta{S_{16}}{c}{S_{12}}{\mbqc, F}                                \\
		\gendelta{S_{12}}{c}{S_{16}F}{\mbqc, F}                               \\
		\gendelta{S_{12}S_{15}}{c}{S_{12}S_{16}F}{\mbqc, S_{16}, F}           \\
		\gendelta{S_{12}S_{16}}{c}{S_{12}S_{16}F}{\mbqc, F}                   \\
		\gendelta{S_{16}}{b}{S_{12}}{F}                                       \\
		\gendelta{S_{12}}{b}{FS_{16}}{F}                                      \\
		\gendelta{S_{10}}{b}{S_{15}S_{16}}{F}                                 \\
		\gendelta{S_{15}}{b}{S_{10}S_{12}}{S_{16},F}                          \\
		\gendelta{S_{15}S_{12}}{b}{S_{10}S_{12}S_{16}F}{S_{16},F}             \\
		\gendelta{S_{15}S_{10}}{b}{S_{10}S_{12}S_{15}S_{16}}{S_{16},F}        \\
		\gendelta{S_{10}S_{16}}{b}{S_{12}S_{15}S_{16}}{F}                     \\
		\gendelta{S_{12}S_{10}S_{15}}{b}{S_{10}S_{12}S_{15}S_{16}F}{S_{16},F} \\
		\gendelta{S_{12}S_{10}}{b}{S_{15}S_{16}F}{F}
	\end{array}
\end{align*}

\newpage
\subsubsection{ДКА для \(M_3\)}
\subsubsubsection{Удаление недостижимых символов}
Для удобства, сначала удалим недостижимые состояния:
\begin{align*}
	R = \cbr{q_{150}}, P_0 = \cbr{q_{150}}                                                                                                                                      \\
	P_1 = \cbr{q_{11},q_{21}, q_{51}, q_{61}}, R \setminus P_1 \neq \varnothing \Longrightarrow R = \cbr{q_{150}, q_{11},q_{21}, q_{51}, q_{61}}                                \\
	P_2 = \cbr{q_{31},q_{41}, q_{71}, q_{81}}, R \setminus P_2 \neq \varnothing \Longrightarrow R = \cbr{q_{150}, q_{11},q_{21}, q_{31}, q_{41} q_{51}, q_{61}, q_{71}, q_{81}} \\
	P_3 = \cbr{q_{51}, q_{61}}, R \setminus P_3 = \varnothing \Longrightarrow R = \cbr{q_{150}, q_{11},q_{21}, q_{31}, q_{41} q_{51}, q_{61}, q_{71}, q_{81}}                   \\
\end{align*}
Автомат после удаления недостижимых состояний
\begin{align*}
	M_3 = \rbr{\cbr{q_{150}, q_{11},q_{21}, q_{31}, q_{41} q_{51}, q_{61}, q_{71}, q_{81}}, \Sigma, \delta_3, q_{150}, \cbr{q_{71}, q_{81}}} \\
	\begin{array}{lll}
		\delta_{3}(q_{11}, a) =  \cbr{q_{31}} & \delta_{3}(q_{11}, b)=  \cbr{q_{41}}                                                  \\
		\delta_{3}(q_{21}, a) =  \cbr{q_{31}} & \delta_{3}(q_{21}, b)=  \cbr{q_{41}}                                                  \\
		\delta_{3}(q_{31}, a) = \cbr{q_{11}}  & \delta_{3}(q_{31}, b) = \cbr{q_{21}, q_{51}}  & \delta_{3}(q_{31}, c) = \cbr{q_{61}}  \\
		\delta_{3}(q_{41}, a) = \cbr{q_{11}}  & \delta_{3}(q_{41}, b) = \cbr{q_{21}, q_{51}}  & \delta_{3}(q_{41}, c) = \cbr{q_{61}}  \\
		\delta_{3}(q_{150}, a) = \cbr{q_{11}} & \delta_{3}(q_{150}, b) = \cbr{q_{21}, q_{51}} & \delta_{3}(q_{150}, c) = \cbr{q_{61}} \\
		\delta_{3}(q_{51}, b)  = \cbr{q_{71}} & \delta_{3}(q_{51}, c)  = \cbr{q_{81}}                                                 \\
		\delta_{3}(q_{61}, b)  = \cbr{q_{71}} & \delta_{3}(q_{61}, c)  = \cbr{q_{81}}                                                 \\
		\delta_{3}(q_{71}, b) = \cbr{q_{51}}  & \delta_{3}(q_{71}, c) =\cbr{q_{61}}                                                   \\
		\delta_{3}(q_{81}, b) = \cbr{q_{51}}  & \delta_{3}(q_{81}, c) =\cbr{q_{61}}                                                   \\
	\end{array}
\end{align*}

\begin{figure}[h!]
	\centering
	\begin{tikzpicture}[
			->,
			>=stealth',
			node distance=2.0cm,
			every state/.style={thick, fill=gray!10},
			initial text={Начало}
		]

		% Состояния
		\node[state, initial] (q150) {$q_{150}$};
		\node[state,  right of=q10] (q11) {$q_{11}$};
		\node[state,  right of=q20] (q21) {$q_{21}$};
		\node[state, right= of q11] (q31) {$q_{31}$};
		\node[state, right= of q21] (q41) {$q_{41}$};
		% Переходы
		\path
		(q150) edge node[above] {a} (q11)
		(q150) edge node[below] {b} (q21)
		(q11) edge node[below] {a} (q31)
		(q21) edge[bend left=20] node[above] {a} (q31)
		(q11) edge[bend right=20] node[below] {b} (q41)
		(q21) edge node[above] {b} (q41)
		(q31) edge[bend right=20] node[above]{a} (q11)
		(q31) edge[bend left=20] node[below]{b} (q21)
		(q41) edge[bend left=20] node[below]{b} (q21)
		(q41) edge[bend right=20] node[above]{a} (q11)
		;
		\node[state, above of=q110] (q51) {$q_{51}$};
		\node[state, below of=q110] (q61) {$q_{61}$};
		\node[state, accepting, right = of q51] (q71) {$q_{71}$};
		\node[state, accepting, right = of q61] (q81) {$q_{81}$};
		% Переходы
		\path
		(q51) edge node[below] {b} (q71)
		(q61) edge[bend left=20] node[above] {b} (q71)
		(q51) edge[bend right=20] node[below] {c} (q81)
		(q61) edge node[above] {c} (q81)
		(q71) edge[bend right=20] node[above] {b} (q51)
		(q81) edge[bend left=20] node[below] {c} (q61)
		(q71) edge[bend left=20] node[below]{c} (q61)
		(q81) edge[bend right=20] node[above]{b} (q51)
		;

		\path
		(q41) edge[bend right=30] node[below]{c} (q61)
		(q31) edge[bend left=30] node[above]{b} (q51)
		(q41) edge[bend left=20] node[above]{b} (q51)
		(q31) edge[bend right=20] node[below]{c} (q61)
		(q150) edge[bend right=60] node[below]{c} (q61)
		(q150) edge[bend left=60] node[above]{b} (q51)
		;

	\end{tikzpicture}
	\caption{Диаграмма состояний НКА \(M_{17}\)}
\end{figure}

\newpage
\subsubsubsection{Построение ДКА}
Переходы определяются как
\begin{align*}
	\begin{array}{ll}
		\gendelta{q_{150}}{a}{q_{11}}{\mcqa, q_{41}, q_{31}}                                \\
		\gendelta{q_{41}}{a}{q_{11}}{\mcqa, q_{31}}                                         \\
		\gendelta{q_{31}}{a}{q_{11}}{\mcqa}                                                 \\
		\gendelta{q_{11}}{a}{q_{31}}{\mcqa, q_{21}}                                         \\
		\gendelta{q_{21}}{a}{q_{31}}{\mcqa}                                                 \\
		\gendelta{q_{11}q_{150}}{a}{q_{11}q_{31}}{\mcqa,q_{31}, q_{41},q_{21}}              \\
		\gendelta{q_{11}q_{31}}{a}{q_{11}q_{31}}{\mcqa, q_{41},q_{21}}                      \\
		\gendelta{q_{11}q_{41}}{a}{q_{11}q_{31}}{\mcqa, q_{21}}                             \\
		\gendelta{q_{21}q_{150}}{a}{q_{11}q_{31}}{\mcqa,q_{31}, q_{41}}                     \\
		\gendelta{q_{21}q_{31}}{a}{q_{11}q_{31}}{\mcqa, q_{41}}                             \\
		\gendelta{q_{21}q_{41}}{a}{q_{11}q_{31}}{\mcqa}                                     \\
		\gendelta{q_{150}}{c}{q_{61}}{\mcqc,q_{31},q_{41},q_{71},q_{81}}                    \\
		\gendelta{q_{31}}{c}{q_{61}}{\mcqc,q_{41},q_{71},q_{81}}                            \\
		\gendelta{q_{41}}{c}{q_{61}}{\mcqc,q_{71},q_{81}}                                   \\
		\gendelta{q_{71}}{c}{q_{61}}{\mcqc,q_{81}}                                          \\
		\gendelta{q_{81}}{c}{q_{61}}{\mcqc}                                                 \\
		\gendelta{q_{51}}{c}{q_{81}}{\mcqc, q_{61}}                                         \\
		\gendelta{q_{61}}{c}{q_{81}}{\mcqc}                                                 \\
		\gendelta{q_{51}q_{150}}{c}{q_{61}q_{81}}{\mcqc,q_{31},q_{41},q_{71},q_{81},q_{61}} \\
		\gendelta{q_{51}q_{31}}{c}{q_{61}q_{81}}{\mcqc,q_{41},q_{71},q_{81},q_{61}}         \\
		\gendelta{q_{51}q_{41}}{c}{q_{61}q_{81}}{\mcqc,q_{71},q_{81},q_{61}}                \\
		\gendelta{q_{51}q_{71}}{c}{q_{61}q_{81}}{\mcqc,q_{81},q_{61}}                       \\
		\gendelta{q_{51}q_{81}}{c}{q_{61}q_{81}}{\mcqc,q_{61}}                              \\
		\gendelta{q_{61}q_{150}}{c}{q_{61}q_{81}}{\mcqc,q_{31},q_{41},q_{71},q_{81}}        \\
		\gendelta{q_{61}q_{31}}{c}{q_{61}q_{81}}{\mcqc,q_{41},q_{71},q_{81}}                \\
		\gendelta{q_{61}q_{41}}{c}{q_{61}q_{81}}{\mcqc,q_{71},q_{81}}                       \\
		\gendelta{q_{61}q_{71}}{c}{q_{61}q_{81}}{\mcqc,q_{81}}                              \\
		\gendelta{q_{61}q_{81}}{c}{q_{61}q_{81}}{\mcqc}                                     \\
		\gendelta{q_{11}}{b}{q_{41}}{q_{21}}                                                \\
		\gendelta{q_{21}}{b}{q_{41}}{}                                                      \\
		\gendelta{q_{51}}{b}{q_{71}}{q_{61}}                                                \\
		\gendelta{q_{61}}{b}{q_{71}}{}                                                      \\
		\gendelta{q_{150}}{b}{q_{21}q_{51}}{q_{31},q_{41},q_{81}, q_{71}}                   \\
		\gendelta{q_{31}}{b}{q_{21}q_{51}}{q_{41},q_{81}, q_{71}}                           \\
		\gendelta{q_{41}}{b}{q_{21}q_{51}}{q_{81}, q_{71}}                                  \\
		\gendelta{q_{81}}{b}{q_{51}}{q_{71}}                                                \\
		\gendelta{q_{71}}{b}{q_{51}}{}                                                      \\
	\end{array}
\end{align*}

\begin{align*}
	\begin{array}{ll}
		\gendelta{q_{71}q_{51}}{b}{q_{51}q_{71}}{q_{81},q_{61}}                                                \\
		\gendelta{q_{81}q_{51}}{b}{q_{51}q_{71}}{q_{61}}                                                       \\
		\gendelta{q_{71}q_{61}}{b}{q_{51}q_{71}}{q_{81}}                                                       \\
		\gendelta{q_{81}q_{61}}{b}{q_{51}q_{71}}{}                                                             \\
		\gendelta{q_{51}q_{11}}{b}{q_{41}q_{71}}{q_{61},q_{21}}                                                \\
		\gendelta{q_{61}q_{11}}{b}{q_{41}q_{71}}{q_{21}}                                                       \\
		\gendelta{q_{51}q_{21}}{b}{q_{41}q_{71}}{q_{61}}                                                       \\
		\gendelta{q_{61}q_{21}}{b}{q_{41}q_{71}}{}                                                             \\
		\gendelta{q_{71}q_{11}}{b}{q_{51}q_{41}}{ q_{81}, q_{21}}                                              \\
		\gendelta{q_{81}q_{11}}{b}{q_{51}q_{41}}{ q_{21}}                                                      \\
		\gendelta{q_{71}q_{21}}{b}{q_{51}q_{41}}{ q_{81}}                                                      \\
		\gendelta{q_{81}q_{21}}{b}{q_{51}q_{41}}{}                                                             \\
		\gendelta{q_{150}q_{11}}{b}{q_{21}q_{51}q_{41}}{q_{31},q_{41},q_{81},q_{71},q_{21}}                    \\
		\gendelta{q_{31}q_{11}}{b}{q_{21}q_{51}q_{41}}{q_{41},q_{81},q_{71},q_{21}}                            \\
		\gendelta{q_{41}q_{11}}{b}{q_{21}q_{51}q_{41}}{q_{81},q_{71},q_{21}}                                   \\
		\gendelta{q_{150}q_{21}}{b}{q_{21}q_{51}q_{41}}{q_{31},q_{41},q_{81},q_{71}}                           \\
		\gendelta{q_{31}q_{21}}{b}{q_{21}q_{51}q_{41}}{q_{41},q_{81},q_{71}}                                   \\
		\gendelta{q_{41}q_{21}}{b}{q_{21}q_{51}q_{41}}{q_{81},q_{71}}                                          \\
		\gendelta{q_{150}q_{51}}{b}{q_{21}q_{51}q_{71}}{q_{31},q_{41},q_{81},q_{71},q_{61}}                    \\
		\gendelta{q_{31}q_{51}}{b}{q_{21}q_{51}q_{71}}{q_{41},q_{81},q_{71},q_{61}}                            \\
		\gendelta{q_{41}q_{51}}{b}{q_{21}q_{51}q_{71}}{q_{81},q_{71},q_{61}}                                   \\
		\gendelta{q_{150}q_{61}}{b}{q_{21}q_{51}q_{71}}{q_{31},q_{41},q_{81},q_{71}}                           \\
		\gendelta{q_{31}q_{61}}{b}{q_{21}q_{51}q_{71}}{q_{41},q_{81},q_{71}}                                   \\
		\gendelta{q_{41}q_{61}}{b}{q_{21}q_{51}q_{71}}{q_{81},q_{71}}                                          \\
		\gendelta{q_{11}q_{71}q_{51}}{b}{q_{71}q_{51}q_{41}}{q_{21},q_{81},q_{61}}                             \\
		\gendelta{q_{21}q_{71}q_{51}}{b}{q_{71}q_{51}q_{41}}{q_{81},q_{61}}                                    \\
		\gendelta{q_{11}q_{81}q_{51}}{b}{q_{71}q_{51}q_{41}}{q_{21},q_{61}}                                    \\
		\gendelta{q_{21}q_{81}q_{51}}{b}{q_{71}q_{51}q_{41}}{q_{61}}                                           \\
		\gendelta{q_{11}q_{81}q_{61}}{b}{q_{71}q_{51}q_{41}}{q_{21}}                                           \\
		\gendelta{q_{21}q_{81}q_{61}}{b}{q_{71}q_{51}q_{41}}{}                                                 \\
		\gendelta{q_{150}q_{11}q_{51}}{b}{q_{21}q_{51}q_{41}q_{71}}{q_{31},q_{41},q_{81},q_{71},q_{61},q_{21}} \\
		\gendelta{q_{31}q_{11}q_{51}}{b}{q_{21}q_{51}q_{41}q_{71}}{q_{41},q_{81},q_{71},q_{61},q_{21}}         \\
		\gendelta{q_{41}q_{11}q_{51}}{b}{q_{21}q_{51}q_{41}q_{71}}{q_{81},q_{71},q_{61},q_{21}}                \\
		\gendelta{q_{150}q_{21}q_{51}}{b}{q_{21}q_{51}q_{41}q_{71}}{q_{31},q_{41},q_{81},q_{71},q_{61}}        \\
		\gendelta{q_{31}q_{21}q_{51}}{b}{q_{21}q_{51}q_{41}q_{71}}{q_{41},q_{81},q_{71},q_{61}}                \\
		\gendelta{q_{41}q_{21}q_{51}}{b}{q_{21}q_{51}q_{41}q_{71}}{q_{81},q_{71},q_{61}}                       \\
		\gendelta{q_{150}q_{11}q_{61}}{b}{q_{21}q_{51}q_{41}q_{71}}{q_{31},q_{41},q_{81},q_{71},q_{21}}        \\
		\gendelta{q_{31}q_{11}q_{61}}{b}{q_{21}q_{51}q_{41}q_{71}}{q_{41},q_{81},q_{71},q_{21}}                \\
		\gendelta{q_{41}q_{11}q_{61}}{b}{q_{21}q_{51}q_{41}q_{71}}{q_{81},q_{71},q_{21}}                       \\
		\gendelta{q_{150}q_{21}q_{61}}{b}{q_{21}q_{51}q_{41}q_{71}}{q_{31},q_{41},q_{81},q_{71}}               \\
		\gendelta{q_{31}q_{21}q_{61}}{b}{q_{21}q_{51}q_{41}q_{71}}{q_{41},q_{81},q_{71}}                       \\
		\gendelta{q_{41}q_{21}q_{61}}{b}{q_{21}q_{51}q_{41}q_{71}}{q_{81},q_{71}}                              \\
	\end{array}
\end{align*}

