\subsection{Определение детерменированности построенных автоматов \(M_1\), \(M_2\), \(M_3\)}
\begin{enumerate}
	\item Автомат \(M_1\) --- НКА, так как есть переходы
	      \begin{align*}
		      \delta_1(H,b) = \cbr{S_9, S_{11}} \\
		      \delta_1(S_{15},b) = \cbr{S_9, S_{11}}
	      \end{align*}
	\item Автомат \(M_2\) --- НКА, так как есть переходы
	      \begin{align*}
		      \delta_2(S_{15}, b) = \cbr{S_{10}, S_{12}} \\
		      \delta_2(S_{12}, b) = \cbr{S_{16}, F}
	      \end{align*}
	\item Автомат \(M_3 \equiv M_{17}\) --- НКА, так как есть переходы
	      \begin{align*}
		      \delta_3(q_{31}, b) = \cbr{q_{21}, q_{51}} \\
		      \delta_3(q_{41}, b) = \cbr{q_{21}, q_{51}} \\
		      \delta_3(q_{150}, b) = \cbr{q_{21}, q_{51}}
	      \end{align*}
\end{enumerate}
Все три представленных автомата являются недетерменированными.
