% !TEX program = xelatex
\documentclass[12pt]{article}

\usepackage[pagestyles]{titlesec}
\usepackage{amsmath}
\usepackage{amsthm}
\usepackage{amssymb}

\usepackage[russian]{babel}
\usepackage{hyperref}
\hypersetup{
  colorlinks=true,
  linkcolor=blue,
  filecolor=magenta,
}
\usepackage[russian]{cleveref}
\usepackage{mathtools}
\usepackage{titlesec}

\setcounter{secnumdepth}{5}
\setcounter{tocdepth}{5}

\makeatletter
\newcommand\subsubsubsection{\@startsection{paragraph}{4}{\z@}{-2.5ex\@plus -1ex \@minus -.25ex}{1.25ex \@plus .25ex}{\normalfont\normalsize\bfseries}}
\newcommand\subsubsubsubsection{\@startsection{subparagraph}{5}{\z@}{-2.5ex\@plus -1ex \@minus -.25ex}{1.25ex \@plus .25ex}{\normalfont\normalsize\bfseries}}
\makeatother

\usepackage{mathspec}
\usepackage{fontspec}
\usepackage{xunicode}
\usepackage{xltxtra}
\usepackage{xecyr}
\usepackage{minted}
\usepackage{multirow}
\usepackage{cleveref}

\usepackage{ifthen}
\usemintedstyle{pastie}
\setminted{fontsize=\small, linenos, breaklines}
\usepackage{hyphenat}

\hyphenation{
	равно-от-стоя-щей
	ко-эф-фи-ци-ен-ты
	со-стоя-щим }


\usepackage[left=3cm, right=1.5cm, vmargin=1.99cm]{geometry}

\setromanfont{Times New Roman}
\setsansfont{Liberation Sans}
\setmonofont{3270 Nerd Font}
\setmathfont{Latin Modern Roman}

\newcommand{\ulwt}[3]{\underline{\hspace*{#1}#2\hspace{#3}}}

\newpagestyle{mainPS}{
  \setfoot{}{}{\thepage}
}

\pagestyle{mainPS}

\newcommand{\iftxt}{\text{если}}

\newtheorem*{remark}{Примечание}

\newcommand{\RomanNum}[1]
  {\MakeUppercase{\romannumeral #1}}

\DeclareMathOperator{\diag}{diag}

% \newcounter{ubracenum}
% \setcounter{ubracenum}{1}
% \newcommand{\ubrace}[1]{%
%   \underbrace{#1}_{\arabic{ubracenum}}%
%   \stepcounter{ubracenum}%
% }

\newcommand{\ub}[1]{%
  \underbrace{#1}
}
\newcommand{\rbr}[1]{%
  \left(#1\right)
}
\newcommand{\sbr}[1]{%
  \left[#1\right]
}
\newcommand{\cbr}[1]{%
  \left\{#1\right\}
}
\newcommand{\grammatics}[4]{%
  \rbr{%
    \begin{array}{l}
      \cbr{#1},%
      #2,\\%
      \cbr{\begin{array}{ll}%
        #3%
      \end{array}},
      #4%
    \end{array}
  }%
}
\newcommand{\setconc}[2]{%
  \cbr{%
    \begin{array}{l}
      #1
    \end{array}
  }\Longrightarrow {#2}%
}

\def \ifempty#1{\def\temp{#1} \ifx\temp\empty }

\newcommand{\gendeltanonempty}[4]{%
  \delta'\rbr{\sbr{#1 q_1\dots q_k}, #2} = \sbr{#3},& \cbr{q_1,\dots, q_k} \subset %
  \cbr{%
    \begin{array}{l}%
      #4
    \end{array}%
  }%
}

\newcommand{\gendeltaempty}[4]{%
  \delta'\rbr{\sbr{#1}, #2} = \sbr{#3}%
}

\newcommand{\gendelta}[4]{%
  \ifempty{#4}%
    \gendeltaempty{#1}{#2}{#3}{#4}%
  \else%
    \gendeltanonempty{#1}{#2}{#3}{#4}%
  \fi%
}

\usepackage{tikz}
\usetikzlibrary{automata, positioning, arrows}

\newcommand{\maqa}{S_{11},S_{16}}
\newcommand{\maqb}{}
\newcommand{\maqc}{S_9}

\newcommand{\mbqa}{S_{12},F}
\newcommand{\mbqb}{}
\newcommand{\mbqc}{S_{10}}

\newcommand{\mcqa}{q_{51},q_{71},q_{61},q_{81}}
\newcommand{\mcqb}{}
\newcommand{\mcqc}{q_{11},q_{21}}

\begin{document}
% foo
\newpagestyle{titlePS}{
	\setfoot{}{Казань 2025}{}
}
\thispagestyle{titlePS}

\begin{center}
	\MakeUppercase{ Министерство науки и высшего образования российской федерации }\\
	федеральное государственное бюджетное образовательное учреждение высшего образования <<Казанский национальный исследовательский
	технический университет им. А.Н. Туполева-КАИ>>\\
	(КНИТУ-КАИ)
\end{center}

\(\underset{\text{(наименование института)}}{\underline{\text{Институт Компьютерных технологий и защиты информации}\hspace{2cm}}}\)

Кафедра \( \underset{\text{(наименование кафедры)}}{\underline{\text{Прикладной математики и информатики}\hspace{2cm}}} \)
\vspace{0pt plus2fill}
\begin{center}
	\textbf{\MakeUppercase{домашняя работа}}\\
	\textbf{по дисциплине} \\
	\ulwt{1em}{<<Теория формальных языков и методы трансляции>>}{1em}\\
	\ulwt{1em}{Вариант 2.14}{1em}
\end{center}
\vspace{0pt plus1fill}


\vspace{0pt plus2fill}
\hfill\parbox{9cm}{
	Выполнил студент группы \underline{4312} \\
	\( \underset{\text{(ФИО)}}{\underline{\hspace*{0em}\text{Д.Д.Наумихин}\hspace*{1em}}} \) \vspace{1em} \\
	%Руководитель практики от университета \vspace{1em} \\
	%Проверил \vspace{1em}\\
	%\( \underset{\text{(должность)}}{\underline{\text{доцент}}} \) \hfill \( \underset{\text{(подпись)}}{\underline{\text{\phantom{,}}\hspace{2cm}}} \) \hfill \( \underset{\text{(расшифровка подписи)}}{\underline{\text{Комиссарова Е.М.}}}\) \vspace{1em} \\
	%Отчёт защищён с оценкой: \hrulefill \vspace{1ex} \\
	%Дата защиты <<\underline{\hspace{2ex}}>> \hrulefill \hspace{1ex} 2024 г.
}

\vspace{0pt plus2fill}


\tableofcontents
Вариант 2.14
\begin{align}\label{language-for-analyze}
	L = \left\{ ((a,b)^2)^k \cdot ((b,c)^2)^m \colon \forall k > 0, m \geq 0,\, k,m \in \mathbb{Z} \right\} \\
\end{align}
\section{Определение типа языка L}
Язык \cref{language-for-analyze} является регулярным. Докажем это, пользуясь замкнутостью класса регулярных языков.
\begin{enumerate}
	\item Множества \(\{a\}, \{b\}, \{c\}\) являются регулярными по определению;
	\item Множества
	      \begin{align}
		      \{a\} \cup \{b\} =  \{a,b\} \\
		      \{b\} \cup \{c\} =  \{b,c\}
	      \end{align}
	      регулярны, так как объединение регулярных множеств --- регулярное множество
	\item Множества \begin{align}
		      S_1 = \{a, b\}\{a,b\} \\
		      S_2 = \{b, c\}\{b,c\}
	      \end{align}
	      регулярны , поскольку конкатенация регулярных множеств --- регулярное множество
	\item Множества
	      \begin{align}
		      S_1^+ = S_1 S_1^* \\
		      S_2^*
	      \end{align}
	      регулярны, посколько итерация регулярного множества --- регулярное множество и конкатенация регулярных множеств --- регулярное множество
	\item Конкатенация регулярных множеств --- регулярное множество, а потому:
	      \begin{align}
		      S_3 = S_1^+ \cdot S_2^*
	      \end{align}
	      есть регулярное множество.
\end{enumerate}
\section{Регулярный язык}
\subsection{Приведите искомого множества к регулярному виду}
Регулярное множество:
\begin{align}
	\{a, b\}\cdot\{a, b\}^*\cdot\{b, c\}^*
\end{align}
\subsection{Построение регулярного выражения для искомого регулярного множества}
\begin{align}
	p=((a+b)(a+b))^+((b+c)(b+c))^*
\end{align}
\subsection{Получение регулярной грамматики}
\subsubsection{Построение леволинейной и праволинейной грамматик}
\begin{align}
	% p=\ub{\ub{(\ub{\ub{a}_1+\ub{b}_2}_5)^+}_7\ub{(\ub{\ub{b}_3+\ub{c}_4}_6)^*}_8}_9
	% \ub{ % 9
	% 	\ub{ % 7
	% 		\rbr{%
	% 			\ub{\ub{a}_{1} + \ub{b}_{2}}_{5}%
	% 			\ub{\ub{a}_{3} + \ub{b}_{4}}_{5}%
	% 		}^{+}%
	% 	}_{7} % 7
	% 	\cdot
	% 	\ub{%
	% 		\rbr{%
	% 			\ub{\ub{b}_{3} + \ub{c}_{4}}_{6}%
	% 		}^{*}%
	% 	}_{8} % 8
	% }_{9}% 9
	% \ub{\rbr{%
	% 		\ub{a}_{1} + \ub_{b}_{2}%
	% 	}}_{9}%
	\ub{\ub{\rbr{\ub{
					\rbr{\ub{%
							\ub{a}_{1} + \ub{b}_{2}%
						}_{9}}
					\cdot
					\rbr{\ub{%
							\ub{a}_{3} + \ub{b}_{4}%
						}_{10}}}_{13}}^+}_{15}
		\cdot
		\ub{\rbr{\ub{\rbr{\ub{%
							\ub{b}_{5} + \ub{c}_{6}%
						}_{11}}
					\cdot
					\rbr{\ub{%
							\ub{a}_{7} + \ub{b}_{8}%
						}_{12}}}_{14}}^{*}}_{16}}_{17}
\end{align}
\begin{align*}
	G_1 = \grammatics{S_1}{\Sigma}{S_1 \to a}{S_1}                      \\
	G_2 = \grammatics{S_2}{\Sigma}{S_2 \to b}{S_2}                      \\
	G_2 = \grammatics{S_2}{\Sigma}{S_2 \to b}{S_2}                      \\
	G_4 = \grammatics{S_4}{\Sigma}{S_4 \to c}{S_4}                      \\
	G_5 = \grammatics{S_1, S_2, S_5}{\Sigma}{S_5 \to S_1\vert S_2}{S_5} \\
	G_6 = \grammatics{S_3, S_4, S_6}{\Sigma}{S_6 \to S_3\vert S_4}{S_6} \\
\end{align*}

\subsubsection{Приведение грамматики}
\begin{enumerate}
	\item Проверка пустоты
	      \subsubsection{Приведение грамматики}
\begin{enumerate}
	\item Проверка пустоты
	      \begin{itemize}
		      \item Для леволинейной грамматики \(G'_{17}\)
		            \begin{align*}
			            C_0   & = \varnothing                                                                                                \\
			            C_{1} & = \cbr{S_5, S_6, S_9} \cup C_{0} = \cbr{S_1, S_2, S_5, S_6, S_9, S_{16}}                                     \\
			            C_{2} & = \cbr{S_3, S_4, S_5, S_6, S_9, S_{11}} \cup C_{1} = \cbr{S_1, S_2, S_3, S_4, S_5, S_6, S_9, S_{11}, S_{16}} \\
			            C_{3} & = \cbr{S_3, S_4, S_5, S_6, S_7, S_8, S_9, S_{10}, S_{11}} \cup C_{2} =
			            \\ &= \cbr{S_1, S_2, S_3, S_4, S_5, S_6, S_7, S_8, S_9, S_{10}, S_{11}, S_{16}}                                      \\
			            C_{4} & = \cbr{S_3, S_4, S_5, S_6, S_7, S_8, S_9, S_{10}, S_{11}, S_{12}, S_{15}} \cup C_{3}  =                      \\
			                  & = \cbr{S_1, S_2, S_3, S_4, S_5, S_6, S_7, S_8, S_9, S_{10}, S_{11}, S_{12}, S_{15}, S_{16}}                  \\
			            C_{5} & = \cbr{S_1, S_2, S_3, S_4, S_5, S_6, S_7, S_8, S_9, S_{10}, S_{11}, S_{12}, S_{15}, S_{16}} \cup C_{4} =     \\
			                  & = \cbr{S_1, S_2, S_3, S_4, S_5, S_6, S_7, S_8, S_9, S_{10}, S_{11}, S_{12}, S_{15}, S_{16}} = C_4 = \aleph   \\
		            \end{align*}
		            Так как
		            \begin{align}
			            S = S_{16} \in C_5 \Longrightarrow L(G'_{17}) \not= \varnothing
		            \end{align}
		      \item Для праволинейной грамматики \(G''_{17}\)
		            \begin{align*}
			            C_0   & = \varnothing                                                                                                                \\
			            C_{1} & = \cbr{S_3, S_4, S_7, S_8, S_{12}} \cup C_{0} = \cbr{S_3, S_4, S_7, S_8, S_{12}, S_{16}}                                     \\
			            C_{2} & = \cbr{S_3, S_4, S_5, S_6, S_7, S_8, S_{10}, S_{12}} \cup C_{1} = \cbr{S_3, S_4, S_5, S_6, S_7, S_8, S_{10}, S_{12}, S_{16}} \\
			            C_{3} & = \cbr{S_1, S_2, S_3, S_4, S_5, S_6, S_7, S_8, S_{10}, S_{11}, S_{12}} \cup C_{2} =                                          \\
			                  & = \cbr{S_1, S_2, S_3, S_4, S_5, S_6, S_7, S_8, S_{10}, S_{11}, S_{12}, S_{16}}                                               \\
			            C_{4} & = \cbr{S_1, S_2, S_3, S_4, S_5, S_6, S_7, S_8, S_9, S_{10}, S_{11}, S_{12}, S_{16}} \cup C_{3} =                             \\
			                  & = \cbr{S_1, S_2, S_3, S_4, S_5, S_6, S_7, S_8, S_9, S_{10}, S_{11}, S_{12}, S_{16}}                                          \\
			            C_{5} & = \cbr{S_1, S_2, S_3, S_4, S_5, S_6, S_7, S_8, S_9, S_{10}, S_{11}, S_{12}, S_{15}, S_{16}} \cup C_{4} =                     \\
			                  & = \cbr{S_1, S_2, S_3, S_4, S_5, S_6, S_7, S_8, S_9, S_{10}, S_{11}, S_{12}, S_{15}, S_{16}}                                  \\
			            C_{6} & = \cbr{S_1, S_2, S_3, S_4, S_5, S_6, S_7, S_8, S_9, S_{10}, S_{11}, S_{12}, S_{15}, S_{16}} \cup C_{5} =                     \\
			                  & = \cbr{S_1, S_2, S_3, S_4, S_5, S_6, S_7, S_8, S_9, S_{10}, S_{11}, S_{12}, S_{15}, S_{16}} = C_5 = \aleph                   \\
		            \end{align*}
		            Так как
		            \begin{align}
			            S = S_{15} \in C_6 \Longrightarrow L(G''_{17}) \not= \varnothing
		            \end{align}
	      \end{itemize}
	\item Удаление бесполезных символов
	      \begin{itemize}
		      \item Для леволинейной грамматики \(G'_{17}\)
		            \begin{align*}
			            C_0   & = \varnothing                                                                                                \\
			            C_{1} & = \cbr{S_5, S_6, S_9} \cup C_{0} = \cbr{S_1, S_2, S_5, S_6, S_9, S_{16}}                                     \\
			            C_{2} & = \cbr{S_3, S_4, S_5, S_6, S_9, S_{11}} \cup C_{1} = \cbr{S_1, S_2, S_3, S_4, S_5, S_6, S_9, S_{11}, S_{16}} \\
			            C_{3} & = \cbr{S_3, S_4, S_5, S_6, S_7, S_8, S_9, S_{10}, S_{11}} \cup C_{2} =
			            \\ &= \cbr{S_1, S_2, S_3, S_4, S_5, S_6, S_7, S_8, S_9, S_{10}, S_{11}, S_{16}}                                      \\
			            C_{4} & = \cbr{S_3, S_4, S_5, S_6, S_7, S_8, S_9, S_{10}, S_{11}, S_{12}, S_{15}} \cup C_{3}  =                      \\
			                  & = \cbr{S_1, S_2, S_3, S_4, S_5, S_6, S_7, S_8, S_9, S_{10}, S_{11}, S_{12}, S_{15}, S_{16}}                  \\
			            C_{5} & = \cbr{S_1, S_2, S_3, S_4, S_5, S_6, S_7, S_8, S_9, S_{10}, S_{11}, S_{12}, S_{15}, S_{16}} \cup C_{4} =     \\
			                  & = \cbr{S_1, S_2, S_3, S_4, S_5, S_6, S_7, S_8, S_9, S_{10}, S_{11}, S_{12}, S_{15}, S_{16}} = C_4 = \aleph   \\
		            \end{align*}
		            Бесполезных символов нет, следовательно, грамматика \(G'_{17}\) не изменилась.
		      \item Для праволинейной грамматики \(G''_{17}\)
		            \begin{align*}
			            C_0   & = \varnothing                                                                                                                \\
			            C_{1} & = \cbr{S_3, S_4, S_7, S_8, S_{12}} \cup C_{0} = \cbr{S_3, S_4, S_7, S_8, S_{12}, S_{16}}                                     \\
			            C_{2} & = \cbr{S_3, S_4, S_5, S_6, S_7, S_8, S_{10}, S_{12}} \cup C_{1} = \cbr{S_3, S_4, S_5, S_6, S_7, S_8, S_{10}, S_{12}, S_{16}} \\
			            C_{3} & = \cbr{S_1, S_2, S_3, S_4, S_5, S_6, S_7, S_8, S_{10}, S_{11}, S_{12}} \cup C_{2} =                                          \\
			                  & = \cbr{S_1, S_2, S_3, S_4, S_5, S_6, S_7, S_8, S_{10}, S_{11}, S_{12}, S_{16}}                                               \\
			            C_{4} & = \cbr{S_1, S_2, S_3, S_4, S_5, S_6, S_7, S_8, S_9, S_{10}, S_{11}, S_{12}, S_{16}} \cup C_{3} =                             \\
			                  & = \cbr{S_1, S_2, S_3, S_4, S_5, S_6, S_7, S_8, S_9, S_{10}, S_{11}, S_{12}, S_{16}}                                          \\
			            C_{5} & = \cbr{S_1, S_2, S_3, S_4, S_5, S_6, S_7, S_8, S_9, S_{10}, S_{11}, S_{12}, S_{15}, S_{16}} \cup C_{4} =                     \\
			                  & = \cbr{S_1, S_2, S_3, S_4, S_5, S_6, S_7, S_8, S_9, S_{10}, S_{11}, S_{12}, S_{15}, S_{16}}                                  \\
			            C_{6} & = \cbr{S_1, S_2, S_3, S_4, S_5, S_6, S_7, S_8, S_9, S_{10}, S_{11}, S_{12}, S_{15}, S_{16}} \cup C_{5} =                     \\
			                  & = \cbr{S_1, S_2, S_3, S_4, S_5, S_6, S_7, S_8, S_9, S_{10}, S_{11}, S_{12}, S_{15}, S_{16}} = C_5 = \aleph                   \\
		            \end{align*}
		            Бесполезных символов нет, следовательно, грамматика \(G''_{17}\) не изменилась.
	      \end{itemize}
\end{enumerate}

	\item Удаление бесполезных символов
	      \subsubsubsection{ Удаление бесполезных символов}
\begin{itemize}
	\item Для леволинейной грамматики \(G'_{17}\)
	      \begin{align*}
		      C_0    & = \varnothing                                                                                           \\
		      C_{2}  & = \cbr{S_9} \cup C_{1} = \cbr{S_1, S_2, S_9}                                                            \\
		      C_{3}  & = \cbr{S_3, S_4, S_9} \cup C_{2} = \cbr{S_1, S_2, S_3, S_4, S_9}                                        \\
		      C_{4}  & = \cbr{S_3, S_4, S_9, S_{10}} \cup C_{3} = \cbr{S_1, S_2, S_3, S_4, S_9, S_{10}}                        \\
		      C_{5}  & = \cbr{S_3, S_4, S_9, S_{10}, S_{15}} \cup C_{4} = \cbr{S_1, S_2, S_3, S_4, S_9, S_{10}, S_{15}}        \\
		      C_{6}  & = \cbr{S_1, S_2, S_3, S_4, S_5, S_6, S_9, S_{10}, S_{15}} \cup C_{5}                                    \\
		             & = \cbr{S_1, S_2, S_3, S_4, S_5, S_6, S_9, S_{10}, S_{15}}                                               \\
		      C_{7}  & = \cbr{S_1, S_2, S_3, S_4, S_5, S_6, S_9, S_{10}, S_{11}, S_{15}} \cup C_{6}                            \\
		             & = \cbr{S_1, S_2, S_3, S_4, S_5, S_6, S_9, S_{10}, S_{11}, S_{15}}                                       \\
		      C_{8}  & = \cbr{S_1, S_2, S_3, S_4, S_5, S_6, S_7, S_8, S_9, S_{10}, S_{11}, S_{15}} \cup C_{7}                  \\
		             & = \cbr{S_1, S_2, S_3, S_4, S_5, S_6, S_7, S_8, S_9, S_{10}, S_{11}, S_{15}}                             \\
		      C_{9}  & = \cbr{S_1, S_2, S_3, S_4, S_5, S_6, S_7, S_8, S_9, S_{10}, S_{11}, S_{12}, S_{15}} \cup C_{8}          \\
		             & = \cbr{S_1, S_2, S_3, S_4, S_5, S_6, S_7, S_8, S_9, S_{10}, S_{11}, S_{12}, S_{15}}                     \\
		      C_{10} & = \cbr{S_1, S_2, S_3, S_4, S_5, S_6, S_7, S_8, S_9, S_{10}, S_{11}, S_{12}, S_{15}, S_{16}} \cup C_{9}  \\
		             & = \cbr{S_1, S_2, S_3, S_4, S_5, S_6, S_7, S_8, S_9, S_{10}, S_{11}, S_{12}, S_{15}, S_{16}}             \\
		      C_{11} & = \cbr{S_1, S_2, S_3, S_4, S_5, S_6, S_7, S_8, S_9, S_{10}, S_{11}, S_{12}, S_{15}, S_{16}} \cup C_{10} \\
		             & = \cbr{S_1, S_2, S_3, S_4, S_5, S_6, S_7, S_8, S_9, S_{10}, S_{11}, S_{12}, S_{15}, S_{16}} = \aleph    \\
	      \end{align*}
	      Бесполезных символов нет, следовательно, грамматика \(G'_{17}\) не изменилась.
	\item Для праволинейной грамматики \(G''_{17}\)
	      \begin{align*}
		      C_{0} & = \varnothing                                                                                                                        \\
		      C_{1} & = \cbr{S_{7}, S_{8}, S_{16}} \cup C_{0} = \cbr{S_{7}, S_{8}, S_{16}}                                                                 \\
		      C_{2} & = \cbr{S_3, S_4, S_7, S_8, S_{12}, S_{16}} \cup C_{1} = \cbr{S_3, S_4, S_7, S_8, S_{12}, S_{16}}                                     \\
		      C_{3} & = \cbr{S_3, S_4, S_5, S_6, S_7, S_8, S_{10}, S_{12}, S_{16}} \cup C_{2} = \cbr{S_3, S_4, S_5, S_6, S_7, S_8, S_{10}, S_{12}, S_{16}} \\
		      C_{4} & = \cbr{S_1, S_2, S_3, S_4, S_5, S_6, S_7, S_8, S_{10}, S_{11}, S_{12}, S_{16}} \cup C_{3}                                            \\
		            & = \cbr{S_1, S_2, S_3, S_4, S_5, S_6, S_7, S_8, S_{10}, S_{11}, S_{12}, S_{16}}                                                       \\
		      C_{5} & = \cbr{S_1, S_2, S_3, S_4, S_5, S_6, S_7, S_8, S_9, S_{10}, S_{11}, S_{12}, S_{16}} \cup C_{4}                                       \\
		            & = \cbr{S_1, S_2, S_3, S_4, S_5, S_6, S_7, S_8, S_9, S_{10}, S_{11}, S_{12}, S_{16}}                                                  \\
		      C_{6} & = \cbr{S_1, S_2, S_3, S_4, S_5, S_6, S_7, S_8, S_9, S_{10}, S_{11}, S_{12}, S_{15}, S_{16}} \cup C_{5}                               \\
		            & = \cbr{S_1, S_2, S_3, S_4, S_5, S_6, S_7, S_8, S_9, S_{10}, S_{11}, S_{12}, S_{15}, S_{16}}                                          \\
		      C_{7} & = \cbr{S_1, S_2, S_3, S_4, S_5, S_6, S_7, S_8, S_9, S_{10}, S_{11}, S_{12}, S_{15}, S_{16}} \cup C_{6}                               \\
		            & = \cbr{S_1, S_2, S_3, S_4, S_5, S_6, S_7, S_8, S_9, S_{10}, S_{11}, S_{12}, S_{15}, S_{16}} = C_{6} = \aleph                         \\
	      \end{align*}
	      Бесполезных символов нет, следовательно, грамматика \(G''_{17}\) не изменилась.
\end{itemize}

	\item Удаление недостижимых символов
	      \subsubsubsection{ Удаление недостижимых символов}
\begin{itemize}
	\item Для леволинейной грамматики \(G'_{17}\)
	      \begin{align*}
		      C_{0}  & = \cbr{S_{16}}                                                                                                            \\
		      C_{1}  & = \cbr{S_{12}, S_{16}} \cup C_{0} = \cbr{S_{12}, S_{16}}                                                                  \\
		      C_{2}  & = \cbr{S_7, S_8, S_{12}, S_{16}} \cup C_{1} = \cbr{S_7, S_8, S_{12}, S_{16}}                                              \\
		      C_{3}  & = \cbr{S_7, S_8, S_{11}, S_{12}, S_{16}} \cup C_{2} = \cbr{S_7, S_8, S_{11}, S_{12}, S_{16}}                              \\
		      C_{4}  & = \cbr{S_5, S_6, S_7, S_8, S_{11}, S_{12}, S_{16}} \cup C_{3} = \cbr{S_5, S_6, S_7, S_8, S_{11}, S_{12}, S_{16}}          \\
		      C_{5}  & = \cbr{S_5, S_6, S_7, S_8, S_{11}, S_{12}, S_{15}, S_{16}, b, c} \cup C_{4}                                               \\
		             & = \cbr{S_5, S_6, S_7, S_8, S_{11}, S_{12}, S_{15}, S_{16}, b, c}                                                          \\
		      C_{6}  & = \cbr{S_5, S_6, S_7, S_8, S_{10}, S_{11}, S_{12}, S_{15}, S_{16}, b, c} \cup C_{5}                                       \\
		             & = \cbr{S_5, S_6, S_7, S_8, S_{10}, S_{11}, S_{12}, S_{15}, S_{16}, b, c}                                                  \\
		      C_{7}  & = \cbr{S_3, S_4, S_5, S_6, S_7, S_8, S_{10}, S_{11}, S_{12}, S_{15}, S_{16}, b, c} \cup C_{6}                             \\
		             & = \cbr{S_3, S_4, S_5, S_6, S_7, S_8, S_{10}, S_{11}, S_{12}, S_{15}, S_{16}, b, c}                                        \\
		      C_{8}  & = \cbr{S_3, S_4, S_5, S_6, S_7, S_8, S_9, S_{10}, S_{11}, S_{12}, S_{15}, S_{16}, b, c} \cup C_{7}                        \\
		             & = \cbr{S_3, S_4, S_5, S_6, S_7, S_8, S_9, S_{10}, S_{11}, S_{12}, S_{15}, S_{16}, b, c}                                   \\
		      C_{9}  & = \cbr{S_1, S_2, S_3, S_4, S_5, S_6, S_7, S_8, S_9, S_{10}, S_{11}, S_{12}, S_{15}, S_{16}, b, c} \cup C_{8}              \\
		             & = \cbr{S_1, S_2, S_3, S_4, S_5, S_6, S_7, S_8, S_9, S_{10}, S_{11}, S_{12}, S_{15}, S_{16}, b, c}                         \\
		      C_{10} & = \cbr{S_1, S_2, S_3, S_4, S_5, S_6, S_7, S_8, S_9, S_{10}, S_{11}, S_{12}, S_{15}, S_{16}, a, b, c} \cup C_{9}           \\
		             & = \cbr{S_1, S_2, S_3, S_4, S_5, S_6, S_7, S_8, S_9, S_{10}, S_{11}, S_{12}, S_{15}, S_{16}, a, b, c}                      \\
		      C_{11} & = \cbr{S_1, S_2, S_3, S_4, S_5, S_6, S_7, S_8, S_9, S_{10}, S_{11}, S_{12}, S_{15}, S_{16}, a, b, c} \cup C_{10}          \\
		             & = \cbr{S_1, S_2, S_3, S_4, S_5, S_6, S_7, S_8, S_9, S_{10}, S_{11}, S_{12}, S_{15}, S_{16}, a, b, c} = \Sigma \cup \aleph
	      \end{align*}
	      Недостижимых символов нет, следовательно, грамматика \(G'_{17}\) не изменилась.
	\item Для праволинейной грамматики \(G''_{17}\)
	      \begin{align*}
		      C_{0}  & = \cbr{S_{15}}                                                                                                            \\
		      C_{1}  & = \cbr{S_9} \cup C_{0} = \cbr{S_9, S_{15}}                                                                                \\
		      C_{2}  & = \cbr{S_1, S_2, S_9} \cup C_{1} = \cbr{S_1, S_2, S_9, S_{15}}                                                            \\
		      C_{3}  & = \cbr{S_1, S_2, S_9, S_{10}} \cup C_{2} = \cbr{S_1, S_2, S_9, S_{10}, S_{15}}                                            \\
		      C_{4}  & = \cbr{S_1, S_2, S_3, S_4, S_9, S_{10}} \cup C_{3} = \cbr{S_1, S_2, S_3, S_4, S_9, S_{10}, S_{15}}                        \\
		      C_{5}  & = \cbr{S_1, S_2, S_3, S_4, S_9, S_{10}, S_{15}, S_{16}, a, b} \cup C_{4}                                                  \\
		             & = \cbr{S_1, S_2, S_3, S_4, S_9, S_{10}, S_{15}, S_{16}, a, b}                                                             \\
		      C_{6}  & = \cbr{S_1, S_2, S_3, S_4, S_9, S_{10}, S_{11}, S_{15}, S_{16}, a, b} \cup C_{5}                                          \\
		             & = \cbr{S_1, S_2, S_3, S_4, S_9, S_{10}, S_{11}, S_{15}, S_{16}, a, b}                                                     \\
		      C_{7}  & = \cbr{S_1, S_2, S_3, S_4, S_5, S_6, S_9, S_{10}, S_{11}, S_{15}, S_{16}, a, b} \cup C_{6}                                \\
		             & = \cbr{S_1, S_2, S_3, S_4, S_5, S_6, S_9, S_{10}, S_{11}, S_{15}, S_{16}, a, b}                                           \\
		      C_{8}  & = \cbr{S_1, S_2, S_3, S_4, S_5, S_6, S_9, S_{10}, S_{11}, S_{12}, S_{15}, S_{16}, a, b} \cup C_{7}                        \\
		             & = \cbr{S_1, S_2, S_3, S_4, S_5, S_6, S_9, S_{10}, S_{11}, S_{12}, S_{15}, S_{16}, a, b}                                   \\
		      C_{9}  & = \cbr{S_1, S_2, S_3, S_4, S_5, S_6, S_7, S_8, S_9, S_{10}, S_{11}, S_{12}, S_{15}, S_{16}, a, b} \cup C_{8}              \\
		             & = \cbr{S_1, S_2, S_3, S_4, S_5, S_6, S_7, S_8, S_9, S_{10}, S_{11}, S_{12}, S_{15}, S_{16}, a, b}                         \\
		      C_{10} & = \cbr{S_1, S_2, S_3, S_4, S_5, S_6, S_7, S_8, S_9, S_{10}, S_{11}, S_{12}, S_{15}, S_{16}, a, b, c} \cup C_{9}           \\
		             & = \cbr{S_1, S_2, S_3, S_4, S_5, S_6, S_7, S_8, S_9, S_{10}, S_{11}, S_{12}, S_{15}, S_{16}, a, b, c}                      \\
		      C_{11} & = \cbr{S_1, S_2, S_3, S_4, S_5, S_6, S_7, S_8, S_9, S_{10}, S_{11}, S_{12}, S_{15}, S_{16}, a, b, c} \cup C_{10}          \\
		             & = \cbr{S_1, S_2, S_3, S_4, S_5, S_6, S_7, S_8, S_9, S_{10}, S_{11}, S_{12}, S_{15}, S_{16}, a, b, c} = \Sigma \cup \aleph
	      \end{align*}
	      Недостижимых символов нет, следовательно, грамматика \(G''_{17}\) не изменилась.
\end{itemize}

	\item Удаление пустых правил
	      \begin{itemize}
	\item Для леволинейной грамматики \(G'_{17}\)
	      \begin{align*}
		      C_{0} & = \varnothing                                  \\
		      C_{1} & = \varnothing \cup C_{0} = \varnothing = C_{0}
	      \end{align*}
	      Пустых правил нет, следовательно, грамматика \(G'_{17}\) не поменялась.
	\item Для праволинейной грамматики \(G''_{17}\)
	      \begin{align*}
		      C_{0} & = \cbr{S_{16}}                          \\
		      C_{1} & = \varnothing \cup C_{0} = \cbr{S_{16}}
	      \end{align*}
	      Итоговая грамматика \(G''_{18}\) без пустых правил и после добавления новых примет вид
	      \begin{align*}
		      G''_{18} = \grammatics{S_9, S_1, S_2, S_{10}, S_3, S_4, S_{15}, S_{11}, S_5, S_6, S_{12}, S_7, S_8, S_{16}}{\Sigma}{
		      S_9 \to S_1|S_2           & S_{11} \to S_5|S_6 \\
		      S_1 \to aS_{10}           & S_5 \to bS_{12}    \\
		      S_2 \to bS_{10}           & S_6 \to cS_{12}    \\
		      S_{10} \to S_3|S_4        & S_{12} \to S_7|S_8 \\
		      S_3 \to aS_{15}|aS_{16}|a & S_7 \to bS_{16}|b  \\
		      S_4 \to bS_{15}|bS_{16}|b & S_8 \to cS_{16}|c  \\
		      S_{15} \to S_{9}          & S_{16} \to S_{11}  \\
		      }{S_{15}}
	      \end{align*}
\end{itemize}

	\item Удаление цепных правил
	      \subsubsubsection{ Удаление цепных правил}
\begin{itemize}
	\item Строим последовательность множеств \(\aleph_i^X\) для леволинейной грамматики \(G'_{18}\)
	      \begin{align*}
		      \setconc{
		      \aleph^{S_0}_{0}  = \cbr{S_0}                         \\
		      \aleph^{S_0}_{1}  = \cbr{S_0}                         \\
		      }{\aleph^{S_0}  = \varnothing}
		      \setconc{
		      \aleph^{S_1}_{0}  = \cbr{S_1}                         \\
		      \aleph^{S_1}_{1}  = \cbr{S_1}                         \\
		      }{\aleph^{S_1}  = \varnothing}                        \\
		      \setconc{
		      \aleph^{S_2}_{0}  = \cbr{S_2}                         \\
		      \aleph^{S_2}_{1}  = \cbr{S_2}                         \\
		      }{\aleph^{S_2}  = \varnothing}
		      \setconc{
		      \aleph^{S_3}_{0}  = \cbr{S_3}                         \\
		      \aleph^{S_3}_{1}  = \cbr{S_3}                         \\
		      }{\aleph^{S_3}  = \varnothing}                        \\
		      \setconc{
		      \aleph^{S_4}_{0}  = \cbr{S_4}                         \\
		      \aleph^{S_4}_{1}  = \cbr{S_4}                         \\
		      }{\aleph^{S_4}  = \varnothing}
		      \setconc{
		      \aleph^{S_5}_{0}  = \cbr{S_5}                         \\
		      \aleph^{S_5}_{1}  = \cbr{S_5}                         \\
		      }{\aleph^{S_5}  = \varnothing}                        \\
		      \setconc{
		      \aleph^{S_6}_{0}  = \cbr{S_6}                         \\
		      \aleph^{S_6}_{1}  = \cbr{S_6}                         \\
		      }{\aleph^{S_6}  = \varnothing}
		      \setconc{
		      \aleph^{S_7}_{0}  = \cbr{S_7}                         \\
		      \aleph^{S_7}_{1}  = \cbr{S_7}                         \\
		      }{\aleph^{S_7}  = \varnothing}                        \\
		      \setconc{
		      \aleph^{S_8}_{0}  = \cbr{S_8}                         \\
		      \aleph^{S_8}_{1}  = \cbr{S_8}                         \\
		      }{\aleph^{S_8}  = \varnothing}                        \\
		      \setconc{
		      \aleph^{S_9}_{0}  = \cbr{S_9}                         \\
		      \aleph^{S_9}_{1}  = \cbr{S_1, S_2, S_9}               \\
		      \aleph^{S_9}_{2}  = \cbr{S_1, S_2, S_9}               \\
		      }{\aleph^{S_9}  = \cbr{S_1, S_2}}                     \\
		      \setconc{
		      \aleph^{S_{10}}_{0}  = \cbr{S_{10}}                   \\
		      \aleph^{S_{10}}_{1}  = \cbr{S_3, S_4, S_{10}}         \\
		      \aleph^{S_{10}}_{2}  = \cbr{S_3, S_4, S_{10}}         \\
		      }{\aleph^{S_{10}}  = \cbr{S_3, S_4}}                  \\
		      \setconc{
		      \aleph^{S_{11}}_{0}  = \cbr{S_{11}}                   \\
		      \aleph^{S_{11}}_{1}  = \cbr{S_5, S_6, S_{11}}         \\
		      \aleph^{S_{11}}_{2}  = \cbr{S_5, S_6, S_{11}}         \\
		      }{\aleph^{S_{11}}  = \cbr{S_5, S_6}}                  \\
		      \setconc{
		      \aleph^{S_{12}}_{0}  = \cbr{S_{12}}                   \\
		      \aleph^{S_{12}}_{1}  = \cbr{S_7, S_8, S_{12}}         \\
		      \aleph^{S_{12}}_{2}  = \cbr{S_7, S_8, S_{12}}         \\
		      }{\aleph^{S_{12}}  = \cbr{S_7, S_8}}                  \\
		      \setconc{
		      \aleph^{S_{15}}_{0}  = \cbr{S_{15}}                   \\
		      \aleph^{S_{15}}_{1}  = \cbr{S_{10}, S_{15}}           \\
		      \aleph^{S_{15}}_{2}  = \cbr{S_3, S_4, S_{10}, S_{15}} \\
		      \aleph^{S_{15}}_{3}  = \cbr{S_3, S_4, S_{10}, S_{15}} \\
		      }{\aleph^{S_{15}}  = \cbr{S_3, S_4, S_{10}}}          \\
		      \setconc{
		      \aleph^{S_{16}}_{0}  = \cbr{S_{16}}                   \\
		      \aleph^{S_{16}}_{1}  = \cbr{S_{12}, S_{16}}           \\
		      \aleph^{S_{16}}_{2}  = \cbr{S_7, S_8, S_{12}, S_{16}} \\
		      \aleph^{S_{16}}_{3}  = \cbr{S_7, S_8, S_{12}, S_{16}} \\
		      }{\aleph^{S_{16}}    = \cbr{S_7, S_8, S_{12}}}        \\
	      \end{align*}
	      Множество правил \(P'_{19}\) содержит все правила грамматики \(G'_{18}\) кроме цепных:
	      \begin{align*}
		      P'_{19} = \cbr{
			      \begin{array}{ll}
				      S_1 \to S_{15}a|a & S_5 \to S_{16}b|S_{15}b|b \\
				      S_2 \to S_{15}b|b & S_6 \to S_{16}c|S_{15}c|c \\
				      S_3 \to S_9a      & S_7 \to S_{11}b           \\
				      S_4 \to S_9b      & S_8 \to S_{11}c           \\
			      \end{array}
		      }
	      \end{align*}
	      С добавлением новых правил, опираясь на соотношение вида
	      \begin{align*}
		      P'_{19} = P'_{19} \cup \cbr{(B \to \alpha) | \forall (A \to \alpha) \in P, A \in \aleph^B},
	      \end{align*}
	      то есть
	      \begin{align*}
		      P'_{19} = P'_{19} \cup \cbr{
			      \begin{array}{ll}
				      S_9 \to S_{15}a|a|S_{15}b|b                    & S_{10} \to S_9a|S_9b       \\
				      S_{11} \to S_{16}b|S_{15}b|S_{16}c|S_{15}c|b|c & S_{12} \to S_{11}b|S_{11}c \\
				      S_{15} \to S_9a|S_9b                           & S_{16} \to S_{11}b|S_{11}c
			      \end{array}
		      }
	      \end{align*}
	      Таким образом, результирующая грамматика \(G'_{19}\) примет следующий вид
	      \begin{align*}
		      G'_{19} = \grammatics{S_9, S_1, S_2, S_{10}, S_3, S_4, S_{15}, S_{11}, S_5, S_6, S_{12}, S_7, S_8, S_{16}}{\Sigma}{
		      S_1 \to S_{15}a|a                              & S_5 \to S_{16}b|S_{15}b|b  \\
		      S_2 \to S_{15}b|b                              & S_6 \to S_{16}c|S_{15}c|c  \\
		      S_3 \to S_9a                                   & S_7 \to S_{11}b            \\
		      S_4 \to S_9b                                   & S_8 \to S_{11}c            \\
		      S_9 \to S_{15}a|a|S_{15}b|b                    & S_{10} \to S_9a|S_9b       \\
		      S_{11} \to S_{16}b|S_{15}b|S_{16}c|S_{15}c|b|c & S_{12} \to S_{11}b|S_{11}c \\
		      S_{15} \to S_9a|S_9b                           & S_{16} \to S_{11}b|S_{11}c
		      }{S_{16}}                                                                   \\
	      \end{align*}
	\item Строим последовательность множеств \(\aleph_i^X\) для праволинейной грамматики \(G''_{17}\)
	      \begin{align*}
		      \setconc{
		      \aleph^{S_0}_{0}     = \cbr{S_0}                                             \\
		      \aleph^{S_0}_{1}     = \cbr{S_0}                                             \\
		      }{\aleph^{S_0}       = \varnothing}
		      \setconc{
		      \aleph^{S_1}_{0}     = \cbr{S_1}                                             \\
		      \aleph^{S_1}_{1}     = \cbr{S_1}                                             \\
		      }{\aleph^{S_1}       = \varnothing}                                          \\
		      \setconc{
		      \aleph^{S_2}_{0}     = \cbr{S_2}                                             \\
		      \aleph^{S_2}_{1}     = \cbr{S_2}                                             \\
		      }{\aleph^{S_2}       = \varnothing}
		      \setconc{
		      \aleph^{S_3}_{0}     = \cbr{S_3}                                             \\
		      \aleph^{S_3}_{1}     = \cbr{S_3}                                             \\
		      }{\aleph^{S_3}       = \varnothing}                                          \\
		      \setconc{
		      \aleph^{S_4}_{0}     = \cbr{S_4}                                             \\
		      \aleph^{S_4}_{1}     = \cbr{S_4}                                             \\
		      }{\aleph^{S_4}       = \varnothing}
		      \setconc{
		      \aleph^{S_5}_{0}     = \cbr{S_5}                                             \\
		      \aleph^{S_5}_{1}     = \cbr{S_5}                                             \\
		      }{\aleph^{S_5}       = \varnothing}                                          \\
		      \setconc{
		      \aleph^{S_6}_{0}     = \cbr{S_6}                                             \\
		      \aleph^{S_6}_{1}     = \cbr{S_6}                                             \\
		      }{\aleph^{S_6}       = \varnothing}
		      \setconc{
		      \aleph^{S_7}_{0}     = \cbr{S_7}                                             \\
		      \aleph^{S_7}_{1}     = \cbr{S_7}                                             \\
		      }{\aleph^{S_7}       = \varnothing}                                          \\
		      \setconc{
		      \aleph^{S_8}_{0}     = \cbr{S_8}                                             \\
		      \aleph^{S_8}_{1}     = \cbr{S_8}                                             \\
		      }{\aleph^{S_8}       = \varnothing}                                          \\
		      \setconc{
		      \aleph^{S_9}_{0}     = \cbr{S_9}                                             \\
		      \aleph^{S_9}_{1}     = \cbr{S_1, S_2, S_9}                                   \\
		      \aleph^{S_9}_{2}     = \cbr{S_1, S_2, S_9}                                   \\
		      }{\aleph^{S_9}       = \cbr{S_1, S_2}}                                       \\
		      \setconc{
		      \aleph^{S_{10}}_{0}  = \cbr{S_{10}}                                          \\
		      \aleph^{S_{10}}_{1}  = \cbr{S_3, S_4, S_{10}}                                \\
		      \aleph^{S_{10}}_{2}  = \cbr{S_3, S_4, S_{10}}                                \\
		      }{\aleph^{S_{10}}    = \cbr{S_3, S_4}}                                       \\
		      \setconc{
		      \aleph^{S_{11}}_{0}  = \cbr{S_{11}}                                          \\
		      \aleph^{S_{11}}_{1}  = \cbr{S_5, S_6, S_{11}}                                \\
		      \aleph^{S_{11}}_{2}  = \cbr{S_5, S_6, S_{11}}                                \\
		      }{\aleph^{S_{11}}    = \cbr{S_5, S_6}}                                       \\
		      \setconc{
		      \aleph^{S_{12}}_{0}  = \cbr{S_{12}}                                          \\
		      \aleph^{S_{12}}_{1}  = \cbr{S_7, S_8, S_{12}}                                \\
		      \aleph^{S_{12}}_{2}  = \cbr{S_7, S_8, S_{12}}                                \\
		      }{\aleph^{S_{12}}    = \cbr{S_7, S_8}}                                       \\
		      \setconc{
		      \aleph^{S_{15}}_{0}  = \cbr{S_{15}}                                          \\
		      \aleph^{S_{15}}_{1}  = \cbr{S_9, S_{15}, S_{16}}                             \\
		      \aleph^{S_{15}}_{2}  = \cbr{S_1, S_2, S_9, S_{11}, S_{15}, S_{16}}           \\
		      \aleph^{S_{15}}_{3}  = \cbr{S_1, S_2, S_5, S_6, S_9, S_{11}, S_{15}, S_{16}} \\
		      \aleph^{S_{15}}_{4}  = \cbr{S_1, S_2, S_5, S_6, S_9, S_{11}, S_{15}, S_{16}} \\
		      }{\aleph^{S_{15}}    = \cbr{S_1, S_2, S_5, S_6, S_9, S_{11}, S_{16}}}        \\
		      \setconc{
		      \aleph^{S_{16}}_{0}  = \cbr{S_{16}}                                          \\
		      \aleph^{S_{16}}_{1}  = \cbr{S_{11}, S_{16}}                                  \\
		      \aleph^{S_{16}}_{2}  = \cbr{S_5, S_6, S_{11}, S_{16}}                        \\
		      \aleph^{S_{16}}_{3}  = \cbr{S_5, S_6, S_{11}, S_{16}}                        \\
		      }{\aleph^{S_{16}}    = \cbr{S_5, S_6, S_{11}}}                               \\
	      \end{align*}
	      Множество правил \(P''_{18}\) содержит все правила грамматики \(G''_{17}\) кроме цепных:
	      \begin{align*}
		      P''_{18} = \cbr{
			      \begin{array}{ll}
				      S_1 \to aS_{10}         & S_5 \to bS_{12}   \\
				      S_2 \to bS_{10}         & S_6 \to cS_{12}   \\
				      S_3 \to aS_{15}|aS_{16} & S_7 \to bS_{16}|b \\
				      S_4 \to bS_{15}|bS_{16} & S_8 \to cS_{16}|c \\
			      \end{array}
		      }
	      \end{align*}
	      С добавлением новых правил, опираясь на соотношение вида
	      \begin{align*}
		      P''_{18} = P''_{18} \cup \cbr{(B \to \alpha) | \forall (A \to \alpha) \in P, A \in \aleph^B},
	      \end{align*}
	      то есть
	      \begin{align*}
		      P''_{18} = P''_{18} \cup \cbr{
			      \begin{array}{ll}
				      S_9 \to aS_{10}|bS_{10}                    & S_{10} \to aS_{15}|aS_{16}|bS_{15}|bS_{16} \\
				      S_{11} \to bS_{12}|cS_{12}                 & S_{12} \to bS_{16}|b|cS_{16}|c             \\
				      S_{15} \to aS_{10}|bS_{10}|bS_{12}|cS_{12} & S_{16} \to bS_{12}|cS_{12}
			      \end{array}
		      }
	      \end{align*}
	      Таким образом, результирующая грамматика \(G''_{18}\) примет следующий вид
	      \begin{align*}
		      G''_{18} = \grammatics{S_9, S_1, S_2, S_{10}, S_3, S_4, S_{15}, S_{11}, S_5, S_6, S_{12}, S_7, S_8, S_{16}}{\Sigma}{
		      S_1 \to aS_{10}                            & S_5 \to bS_{12}                            \\
		      S_2 \to bS_{10}                            & S_6 \to cS_{12}                            \\
		      S_3 \to aS_{15}|aS_{16}                    & S_7 \to bS_{16}|b                          \\
		      S_4 \to bS_{15}|bS_{16}                    & S_8 \to cS_{16}|c                          \\
		      S_9 \to aS_{10}|bS_{10}                    & S_{10} \to aS_{15}|aS_{16}|bS_{15}|bS_{16} \\
		      S_{11} \to bS_{12}|cS_{12}                 & S_{12} \to bS_{16}|b|cS_{16}|c             \\
		      S_{15} \to aS_{10}|bS_{10}|bS_{12}|cS_{12} & S_{16} \to bS_{12}|cS_{12}
		      }{S_{15}}
	      \end{align*}
\end{itemize}
Так как при удалении пустых правил и цепных правил лево- и праволинейной грамматик произошло их изменение, то необходимо повторить удаление бесполезных и недостижимых символов.

	\item Удаление бесполезных символов грамматик \(G'_{18}\) и \(G''_{19}\)
	      \subsubsubsection{ Удаление бесполезных символов грамматик \(G'_{19}\) и \(G''_{18}\)}
\begin{itemize}
	\item Для леволинейной грамматики \(G'_{19}\)
	      \begin{align*}
		      C_0   & = \varnothing                                                                                            \\
		      C_{1} & = \cbr{S_1, S_2, S_5, S_6, S_9, S_{11}, S_{16}} \cup C_0 = \cbr{S_1, S_2, S_5, S_6, S_9, S_{11}, S_{16}} \\
		      C_{2} & = \cbr{S_3, S_4, S_5, S_6, S_7, S_8, S_{10}, S_{11}, S_{12}, S_{15}, S_{16}} \cup C_{1} =                \\
		            & = \cbr{S_1, S_2, S_3, S_4, S_5, S_6, S_7, S_8, S_9, S_{10}, S_{11}, S_{12}, S_{15}, S_{16}}              \\
		      C_{3} & = \cbr{S_1, S_2, S_3, S_4, S_5, S_6, S_7, S_8, S_9, S_{10}, S_{11}, S_{12}, S_{15}, S_{16}} \cup C_{2} = \\
		            & = \cbr{S_1, S_2, S_3, S_4, S_5, S_6, S_7, S_8, S_9, S_{10}, S_{11}, S_{12}, S_{15}, S_{16}} = \aleph
	      \end{align*}
	      Бесполезных символов нет, следовательно, грамматика \(G'_{19}\) не изменилась.
	\item Для праволинейной грамматики \(G''_{18}\)
	      \begin{align*}
		      C_0   & = \varnothing                                                                                                  \\
		      C_{1} & = \cbr{S_7, S_8, S_{12}} \cup C_0 = \cbr{S_7, S_8, S_{12}}                                                     \\
		      C_{2} & = \cbr{S_5, S_6, S_{11}, S_{15}, S_{16}} \cup C_{1} = \cbr{S_5, S_6, S_7, S_8, S_{11}, S_{12}, S_{15}, S_{16}} \\
		      C_{3} & = \cbr{S_3, S_4, S_5, S_6, S_7, S_8, S_{10}, S_{11}, S_{12}, S_{15}, S_{16}} \cup C_{2} =                      \\
		            & = \cbr{S_3, S_4, S_5, S_6, S_7, S_8, S_{10}, S_{11}, S_{12}, S_{15}, S_{16}}                                   \\
		      C_{4} & = \cbr{S_1, S_2, S_3, S_4, S_5, S_6, S_7, S_8, S_9, S_{10}, S_{11}, S_{12}, S_{15}, S_{16}} \cup C_{3} =       \\
		            & = \cbr{S_1, S_2, S_3, S_4, S_5, S_6, S_7, S_8, S_9, S_{10}, S_{11}, S_{12}, S_{15}, S_{16}}                    \\
		      C_{5} & = \cbr{S_1, S_2, S_3, S_4, S_5, S_6, S_7, S_8, S_9, S_{10}, S_{11}, S_{12}, S_{15}, S_{16}} \cup C_{4} =       \\
		            & = \cbr{S_1, S_2, S_3, S_4, S_5, S_6, S_7, S_8, S_9, S_{10}, S_{11}, S_{12}, S_{15}, S_{16}} = \aleph
	      \end{align*}
	      Бесполезных символов нет, следовательно, грамматика \(G''_{18}\) не изменилась.
\end{itemize}

	\item Удаление недостижимых символов грамматик \(G'_{18}\) и \(G''_{19}\)
	      \subsubsubsection{ Удаление недостижимых символов грамматик \(G'_{19}\) и \(G''_{18}\)}
\begin{itemize}
	\item Для леволинейной грамматики \(G'_{19}\)
	      \begin{align*}
		      C_{0} & = \cbr{S_{16}}                                                                                       \\
		      C_{1} & = \cbr{S_9, S_{11}, a, b, c} \cup C_{0} = \cbr{S_9, S_{11}, S_{16}, a, b, c}                         \\
		      C_{2} & = \cbr{S_9, S_{11}, S_{15}, S_{16}, a, b, c} \cup C_{1} = \cbr{S_9, S_{11}, S_{15}, S_{16}, a, b, c} \\
		      C_{3} & = \cbr{S_9, S_{11}, S_{15}, S_{16}, a, b, c} \cup C_{2} = \cbr{S_9, S_{11}, S_{15}, S_{16}, a, b, c}
	      \end{align*}
	      Строим результирующую грамматику \(G'_{20}\) без недостижимых символов
	      \begin{align*}
		      \aleph'_{20} & = \aleph'_{19} \cap C_{3} = \cbr{S_9, S_{11}, S_{15}, S_{16}}                                                                \\
		      \Sigma'_{20} & = \Sigma'_{19} \cap C_{3} = \cbr{a, b, c}                                                                                    \\
		      P'_{20}      & = \cbr{(A \to \alpha) | \forall (A\to\alpha)\in P'_{18},A\in \aleph'_{19}, \alpha \in (\Sigma'_{19} \cup  \aleph'_{19})^*} = \\
		                   & = \cbr{\begin{array}{ll}
				                            S_9 \to S_{15}a|a|S_{15}b|b & S_{11} \to S_{16}b|S_{15}b|S_{16}c|S_{15}c|b|c \\
				                            S_{15} \to S_9a|S_9b        & S_{16} \to S_{11}b|S_{11}c
			                            \end{array}}                                          \\
		      S'_{20}      & \equiv S_{16}
	      \end{align*}
	      Таким образом, результирующая грамматика \(G'_{20}\) примет вид
	      \begin{align*}
		      G'_{20} = \grammatics{S_9, S_{11}, S_{15}, S_{16}}{\cbr{a, b, c}}{%
		      S_9 \to S_{15}a|a|S_{15}b|b & S_{11} \to S_{16}b|S_{15}b|S_{16}c|S_{15}c|b|c \\
		      S_{15} \to S_9a|S_9b        & S_{16} \to S_{11}b|S_{11}c
		      }{S_{16}}
	      \end{align*}
	\item Для праволинейной грамматики \(G''_{18}\)
	      \begin{align*}
		      C_{0} & = \cbr{S_{15}}                                                                                             \\
		      C_{1} & = \cbr{S_{10}, S_{12}, a, b, c} \cup C_{0} = \cbr{S_{10}, S_{12}, S_{15}, a, b, c}                         \\
		      C_{2} & = \cbr{S_{10}, S_{12}, S_{15}, S_{16}, a, b, c} \cup C_{1} = \cbr{S_{10}, S_{12}, S_{15}, S_{16}, a, b, c} \\
		      C_{3} & = \cbr{S_{10}, S_{12}, S_{15}, S_{16}, a, b, c} \cup C_{2} = \cbr{S_{10}, S_{12}, S_{15}, S_{16}, a, b, c}
	      \end{align*}
	      Строим результирующую грамматику \(G''_{19}\) без недостижимых символов
	      \begin{align*}
		      \aleph''_{19} & = \aleph''_{19} \cap C_{4} = \cbr{S_{10}, S_{12}, S_{15}, S_{16}}                                                                \\
		      \Sigma''_{19} & = \Sigma''_{19} \cap C_{4} = \cbr{a, b, c}                                                                                       \\
		      P''_{19}      & = \cbr{(A \to \alpha) | \forall (A\to\alpha)\in P''_{19},A\in \aleph''_{20}, \alpha \in (\Sigma''_{20} \cup  \aleph''_{20})^*} = \\
		                    & = \cbr{\begin{array}{ll}
				                             S_{10} \to aS_{15}|aS_{16}|bS_{15}|bS_{16} & S_{12} \to bS_{16}|b|cS_{16}|c \\
				                             S_{15} \to aS_{10}|bS_{10}|bS_{12}|cS_{12} & S_{16} \to bS_{12}|cS_{12}
			                             \end{array}}                                               \\
		      S''_{19}      & \equiv S_{15}
	      \end{align*}
	      Таким образом, результирующая грамматика \(G''_{19}\) примет вид
	      \begin{align*}
		      G''_{19} = \grammatics{S_{10}, S_{12}, S_{15}, S_{16}}{\cbr{a, b, c}}{%
		      S_{10} \to aS_{15}|aS_{16}|bS_{15}|bS_{16} & S_{12} \to bS_{16}|b|cS_{16}|c \\
		      S_{15} \to aS_{10}|bS_{10}|bS_{12}|cS_{12} & S_{16} \to bS_{12}|cS_{12}
		      }{S_{15}}
	      \end{align*}
\end{itemize}

\end{enumerate}

\subsubsection{Построение конечного автомата для приведенной грамматики}
\begin{enumerate}
	\item Приведение к автоматному виду
	      Все правила в заданной грамматике имеют вид
	      \begin{align*}
		      P'_{19} \subset \cbr{A\to Bx|x \colon A, B \in \aleph, x \in \Sigma}
	      \end{align*}
	      для леволинейной грамматики, а для праволинейной
	      \begin{align*}
		      P''_{20} \subset \cbr{A\to xB|x \colon A, B \in \aleph, x \in \Sigma}
	      \end{align*}
	      А это в свою очередь значит, по построению, что правила данных грамматик \(G'_{19}\) и \(G''_{20}\) удовлетворяют определению автоматной грамматики, а, значит, изменение данных грамматик не производится.
	\item Построение конечных автоматов \(M_1 = (Q_1, \Sigma, \delta_1, q_1, F_1)\) и \(M_2 = (Q_2, \Sigma, \delta_2, q_2, F_2)\) для автоматных грамматик \(G'_{19}\) и \(G''_{20}\).
	      \subsubsubsection{Построение конечных автоматов \(M_1 = (Q_1, \Sigma, \delta_1, q_1, F_1)\) и \(M_2 = (Q_2, \Sigma, \delta_2, q_2, F_2)\) для автоматных грамматик \(G'_{20}\) и \(G''_{19}\).}
\begin{itemize}
	\item Построение автомата \(M_1 = (Q_1, \Sigma, \delta_1, q_1, F_1)\) для леволинейной грамматики производится следующим образом:
	      \begin{itemize}
		      \item Множество состояний состоит из именуемых нетерминалы состояний;
		      \item Добавляется новое состояние --- начальное (на наименование действуют соглашения по наименованию нетерминалов грамматик)
	      \end{itemize}
	      Таким образом
	      \begin{align*}
		      Q_1 = \aleph'_{20} \cup \cbr{H} = \cbr{H, S_9, S_{11}, S_{15}, S_{16}}
	      \end{align*}
	      Начальное состояние:
	      \begin{align*}
		      q_1 \equiv H
	      \end{align*}
	      Множество заключительных состояний содержит целевой символ исходной грамматики
	      \begin{align*}
		      F = \cbr{S_{16}}
	      \end{align*}
	      Множество переходов:
	      \begin{align*}
		      \begin{array}{lll}
			      \delta_1(S_{15}, a) = \cbr{S_9}    & \delta_1(S_{15}, b) = \cbr{S_9, S_{11}}   & \delta_1(S_{15}, c) = \cbr{S_{11}} \\
			      \delta_1(S_{16}, b) = \cbr{S_{11}} & \delta_1(S_{16}, c) = \cbr{S_{11}}                                             \\
			      \delta_1(S_{9}, a) = \cbr{S_{15}}  & \delta_1(S_{9}, b) = \cbr{S_{15}, S_{16}}                                      \\
			      \delta_1(S_{11}, b) = \cbr{S_{16}} & \delta_1(S_{11}, c) = \cbr{S_{16}}                                             \\
			      \delta_1(H, a) = \cbr{S_{9}}       & \delta_1(H, b) = \cbr{S_{9}, S_{11}}      & \delta_1(H, c) = \cbr{S_{16}}
		      \end{array}
	      \end{align*}
	\item Построение автомата \(M_2 = (Q_2, \Sigma, \delta_2, q_2, F_2)\) для леволинейной грамматики производится следующим образом:
	      \begin{itemize}
		      \item Множество состояний состоит из именуемых нетерминалы состояний;
		      \item Добавляется новое состояние --- заключительное (на наименование действуют соглашения по наименованию нетерминалов грамматик)
	      \end{itemize}
	      Таким образом
	      \begin{align*}
		      Q_2 = \aleph''_{20} \cup \cbr{F} = \cbr{F, S_{10}, S_{12}, S_{15}, S_{16}}
	      \end{align*}
	      Начальное состояние --- состояние, соответствующее целевому символу исходной грамматики:
	      \begin{align*}
		      q_2 \equiv S_{15}
	      \end{align*}
	      Множество заключительных состояний будет содержать новое состояние
	      \begin{align*}
		      F_2 = \cbr{F}
	      \end{align*}
	      Множество переходов:
	      \begin{align*}
		      \begin{array}{lll}
			      \delta_2(S_{10}, a) = \cbr{S_{15}, S_{16}} & \delta_2(S_{10}, b) = \cbr{S_{15}, S_{16}}                                      \\
			      \delta_2(S_{12}, b) = \cbr{S_{16}, F}      & \delta_2(S_{12}, c) = \cbr{S_{16}, F}                                           \\
			      \delta_2(S_{15}, a) = \cbr{S_{10}}         & \delta_2(S_{15}, b) = \cbr{S_{10}, S_{12}} & \delta_2(S_{15}, c) = \cbr{S_{12}} \\
			      \delta_2(S_{16}, b) = \cbr{S_{12}}         & \delta_2(S_{16}, c) = \cbr{S_{12}}
		      \end{array}
	      \end{align*}
\end{itemize}
На этом построение конечных автоматов по автоматным грамматикам заканчивается

	\item Построение диаграммы состояний автомата \(M\)
	      \subsubsubsection{Построение диаграммы состояний автомата \(M\)}

Диаграмма состояний конечного автомата --- неупорядоченный ориентированный помеченный граф, вершины которого помечены именами состояний автомата и в котором есть дуга из вершины \(A\) к вершине \(B\) и если есть такой символ \(t\in\Sigma\), для которого существует функция перехода вида \(\delta(A,t)=B\) во множестве \(\delta\) конечного автомата \(M\). Кроме того, эта дуга помечается списком, состоящих из всех \(t\in\Sigma\), для которых есть функция перехода \(\delta(A, t) = B\).
Посторим димграммы состояний для КА \(M_1 = (Q_1, \Sigma, \delta_1, q_1, F_1)\) и \(M_2 = (Q_2, \Sigma, \delta_2, q_2, F_2)\).
\begin{figure}[h!]
	\centering
	\begin{tikzpicture}[
			->,
			>=stealth',
			node distance=2.5cm,
			every state/.style={thick, fill=gray!10},
			initial text={Начало}
		]

		\node[state, initial] (H) {$H$};
		\node[state, above of=H] (S9) {$S_9$};
		\node[state, right of=S9] (S15) {$S_{15}$};
		\node[state, right of=H] (S11) {$S_{11}$};
		\node[state, accepting, right of=S11] (S16) {$S_{16}$};
		\path
		(H) edge node[left] {a,b} (S9)
		(H) edge node[above] {b,c} (S11)

		(S9) edge[bend left=20] node[above] {a,b} (S15)

		(S11) edge[bend left=20] node[above] {b,c} (S16)

		(S15) edge[bend left=20] node[below] {a,b} (S9)
		(S15) edge node[right] {b,c} (S11)

		(S16) edge[bend left=20] node[below] {b,c} (S11)
		;
	\end{tikzpicture}
	\caption{Диаграмма состояний недетерменированного конечного автомата \(M_1\)}
\end{figure}
\begin{figure}[h!]
	\centering
	\begin{tikzpicture}[
			->, % Стрелки от начала к концу
			>=stealth', % Стиль стрелок
			node distance=3cm, % Расстояние между узлами
			every state/.style={thick, fill=gray!10}, % Стиль состояний
			initial text={Начало} % Убираем текст "start" у начального состояния
		]
		\node[state, initial] (S15) {$S_{15}$};
		\node[state, above of=S15] (S10) {$S_{10}$};
		\node[state, right of=S15] (S12) {$S_{12}$};
		\node[state, right of=S10] (S16) {$S_{16}$};
		\node[state, accepting, right of=S12] (F) {$F$};

		\path
		(S15) edge[bend left=20] node[left] {a,b} (S10)
		edge node[below] {b,c} (S12)
		(S10) edge[bend left=20] node[right] {a,b} (S15)
		edge node[above] {a,b} (S16)
		(S12) edge node[below] {b,c} (F)
		edge[bend left=20] node[left] {b,c} (S16)
		(S16) edge[bend left=20] node[right] {b,c} (S12)
		;

	\end{tikzpicture}
	\caption{Диаграмма состояний недетерменированного конечного автомата \(M_2\)}
\end{figure}

На этих диаграммах и далее выделенные состояний являются заключительными.

\end{enumerate}


\subsection{Построение КА по регулярному выражению}
\subsubsection{Построение КА \(M_3\)}
Выполним построение конечных автоматов для выражения \cref{eq:regex}. Очередность построения конечных автоматов будет определяться таки же образом, как и в случае построения грамматик по регулярному выражению \cref{eq:regex-desc}.

Воспользуемся рекурсивным определением регулярного выражения для построения последовательности конечных автоматов для каждого элементарного регулярного выражения, входящих в состав выражения \cref{eq:regex-desc}. Собственно последний КА и будет являться искомым.

Построим КА для указанных выражений. Каждый КА будем нумеровать по номеру выражения, для которого строится данный  КА. Кроме того нумерация состояний КА будет определяться следующим образом: номер каждого состояния будет начинаться с номера конечного автомата.

Для выражения \(a\) конечный автомат примет вид
\begin{align*}
	M_1 = \rbr{\cbr{q_{10}, q_{11}}, \Sigma, \delta_1, q_{10}, \cbr{q_{11}}},
\end{align*}
где множество переходов \(\delta_1\) автомата будет содержать переходы вида
\begin{align*}
	\delta_1(q_{10}, a) = \cbr{q_{11}}
\end{align*}
Граф переходов построенного КА \(M_1\) примет вид
\begin{figure}[h!]
	\centering
	\begin{tikzpicture}[
			->,
			>=stealth',
			node distance=2.5cm,
			every state/.style={thick, fill=gray!10},
			initial text={Начало}
		]

		% Состояния
		\node[state, initial] (q10) {$q_{10}$};
		\node[state, accepting, right of=q10] (q11) {$q_{11}$};

		% Переходы
		\path
		(q10) edge node[above] {a} (q11);
	\end{tikzpicture}
	\caption{Диаграмма состояний НКА \(M_1\)}
\end{figure}

Для выражения \(b\) конечный автомат примет вид
\begin{align*}
	M_2 = \rbr{\cbr{q_{20}, q_{21}}, \Sigma, \delta_2, q_{20}, \cbr{q_{21}}},
\end{align*}
где множество переходов \(\delta_2\) автомата будет содержать переходы вида
\begin{align*}
	\delta_2(q_{20}, b) = \cbr{q_{21}}
\end{align*}
Граф переходов построенного КА \(M_2\) примет вид
\begin{figure}[h!]
	\centering
	\begin{tikzpicture}[
			->,
			>=stealth',
			node distance=2.5cm,
			every state/.style={thick, fill=gray!10},
			initial text={Начало}
		]

		% Состояния
		\node[state, initial] (q20) {$q_{20}$};
		\node[state, accepting, right of=q20] (q21) {$q_{21}$};

		% Переходы
		\path
		(q20) edge node[above] {b} (q21);
	\end{tikzpicture}
	\caption{Диаграмма состояний НКА \(M_2\)}
\end{figure}

Для выражения \(a\) конечный автомат примет вид
\begin{align*}
	M_3 = \rbr{\cbr{q_{30}, q_{31}}, \Sigma, \delta_3, q_{30}, \cbr{q_{31}}},
\end{align*}
где множество переходов \(\delta_3\) автомата будет содержать переходы вида
\begin{align*}
	\delta_3(q_{30}, a) = \cbr{q_{31}}
\end{align*}
Граф переходов построенного КА \(M_3\) примет вид
\begin{figure}[h!]
	\centering
	\begin{tikzpicture}[
			->,
			>=stealth',
			node distance=2.5cm,
			every state/.style={thick, fill=gray!10},
			initial text={Начало}
		]

		% Состояния
		\node[state, initial] (q30) {$q_{30}$};
		\node[state, accepting, right of=q30] (q31) {$q_{31}$};

		% Переходы
		\path
		(q30) edge node[above] {a} (q31);
	\end{tikzpicture}
	\caption{Диаграмма состояний НКА \(M_3\)}
\end{figure}

Для выражения \(b\) конечный автомат примет вид
\begin{align*}
	M_4 = \rbr{\cbr{q_{40}, q_{41}}, \Sigma, \delta_4, q_{40}, \cbr{q_{41}}},
\end{align*}
где множество переходов \(\delta_4\) автомата будет содержать переходы вида
\begin{align*}
	\delta_4(q_{40}, b) = \cbr{q_{41}}
\end{align*}
Граф переходов построенного КА \(M_4\) примет вид
\begin{figure}[h!]
	\centering
	\begin{tikzpicture}[
			->,
			>=stealth',
			node distance=2.5cm,
			every state/.style={thick, fill=gray!10},
			initial text={Начало}
		]

		% Состояния
		\node[state, initial] (q40) {$q_{40}$};
		\node[state, accepting, right of=q40] (q41) {$q_{41}$};

		% Переходы
		\path
		(q40) edge node[above] {b} (q41);
	\end{tikzpicture}
	\caption{Диаграмма состояний НКА \(M_4\)}
\end{figure}

Для выражения \(b\) конечный автомат примет вид
\begin{align*}
	M_5 = \rbr{\cbr{q_{50}, q_{51}}, \Sigma, \delta_5, q_{50}, \cbr{q_{51}}},
\end{align*}
где множество переходов \(\delta_5\) автомата будет содержать переходы вида
\begin{align*}
	\delta_5(q_{50}, b) = \cbr{q_{51}}
\end{align*}
Граф переходов построенного КА \(M_5\) примет вид
\begin{figure}[h!]
	\centering
	\begin{tikzpicture}[
			->,
			>=stealth',
			node distance=2.5cm,
			every state/.style={thick, fill=gray!10},
			initial text={Начало}
		]

		% Состояния
		\node[state, initial] (q50) {$q_{50}$};
		\node[state, accepting, right of=q50] (q51) {$q_{51}$};

		% Переходы
		\path
		(q50) edge node[above] {b} (q51);
	\end{tikzpicture}
	\caption{Диаграмма состояний НКА \(M_5\)}
\end{figure}

Для выражения \(c\) конечный автомат примет вид
\begin{align*}
	M_6 = \rbr{\cbr{q_{60}, q_{61}}, \Sigma, \delta_6, q_{60}, \cbr{q_{61}}},
\end{align*}
где множество переходов \(\delta_5\) автомата будет содержать переходы вида
\begin{align*}
	\delta_6(q_{60}, c) = \cbr{q_{61}}
\end{align*}
Граф переходов построенного КА \(M_6\) примет вид
\begin{figure}[h!]
	\centering
	\begin{tikzpicture}[
			->,
			>=stealth',
			node distance=2.5cm,
			every state/.style={thick, fill=gray!10},
			initial text={Начало}
		]

		% Состояния
		\node[state, initial] (q60) {$q_{60}$};
		\node[state, accepting, right of=q60] (q61) {$q_{61}$};

		% Переходы
		\path
		(q60) edge node[above] {c} (q61);
	\end{tikzpicture}
	\caption{Диаграмма состояний НКА \(M_6\)}
\end{figure}

\newpage

Для выражения \(b\) конечный автомат примет вид
\begin{align*}
	M_7 = \rbr{\cbr{q_{70}, q_{71}}, \Sigma, \delta_7, q_{70}, \cbr{q_{71}}},
\end{align*}
где множество переходов \(\delta_7\) автомата будет содержать переходы вида
\begin{align*}
	\delta_7(q_{70}, b) = \cbr{q_{71}}
\end{align*}
Граф переходов построенного КА \(M_7\) примет вид
\begin{figure}[h!]
	\centering
	\begin{tikzpicture}[
			->,
			>=stealth',
			node distance=2.5cm,
			every state/.style={thick, fill=gray!10},
			initial text={Начало}
		]

		% Состояния
		\node[state, initial] (q70) {$q_{70}$};
		\node[state, accepting, right of=q70] (q71) {$q_{71}$};

		% Переходы
		\path
		(q70) edge node[above] {b} (q71);
	\end{tikzpicture}
	\caption{Диаграмма состояний НКА \(M_7\)}
\end{figure}


Для выражения \(c\) конечный автомат примет вид
\begin{align*}
	M_8 = \rbr{\cbr{q_{80}, q_{81}}, \Sigma, \delta_8, q_{80}, \cbr{q_{81}}},
\end{align*}
где множество переходов \(\delta_5\) автомата будет содержать переходы вида
\begin{align*}
	\delta_8(q_{80}, c) = \cbr{q_{81}}
\end{align*}
Граф переходов построенного КА \(M_8\) примет вид
\begin{figure}[h!]
	\centering
	\begin{tikzpicture}[
			->,
			>=stealth',
			node distance=2.5cm,
			every state/.style={thick, fill=gray!10},
			initial text={Начало}
		]

		% Состояния
		\node[state, initial] (q80) {$q_{80}$};
		\node[state, accepting, right of=q80] (q81) {$q_{81}$};

		% Переходы
		\path
		(q80) edge node[above] {c} (q81);
	\end{tikzpicture}
	\caption{Диаграмма состояний НКА \(M_8\)}
\end{figure}

\newpage
Для выражения \(a+b\) строим КА \(M_9 = (Q_9, \Sigma, \delta_9, q_{90}, F_9)\) следующим образом:
\begin{enumerate}
	\item Множество состояний автомата \(M_9\) получается путем объединений множества состояний автоматов \(M_1\) и \(M_2\) и нового состояния \(q_{90}\)
	      \begin{align*}
		      Q_9 = Q_1 \cup Q_2 \cup \cbr{q_{90}} = \cbr{q_{10}, q_{11}, q_{20}, q_{21}, q_{90}}
	      \end{align*}
	\item \(q_{90}\) --- начальное состояние;
	\item Конечные состояния определяются как объединение конечных состояний \(M_1\) и \(M_2\)
	      \begin{align*}
		      F_{9} = F_{1} \cup F_{2} = \cbr{q_{11}, q_{21}}
	      \end{align*}
	\item Множество переходов \(\delta_9\) строится:
	      \begin{align*}
		      \begin{array}{ll}
			      \delta_9(q_{90}, a) = \cbr{q_{11}} & \delta_9(q_{90}, b) = \cbr{q_{21}} \\
			      \delta_9(q_{10}, a) = \cbr{q_{11}}                                      \\
			      \delta_9(q_{20}, b) = \cbr{q_{21}}
		      \end{array}
	      \end{align*}
	      Граф переходов построенного КА \(M_9\) примет вид
	      \begin{figure}[h!]
		      \centering
		      \begin{tikzpicture}[
				      ->,
				      >=stealth',
				      node distance=2.5cm,
				      every state/.style={thick, fill=gray!10},
				      initial text={Начало}
			      ]

			      % Состояния
			      \node[state, initial] (q90) {$q_{90}$};

			      \node[state, above  of=q90] (q10) {$q_{10}$};
			      \node[state, accepting, right of=q10] (q11) {$q_{11}$};

			      \node[state, below  of=q90] (q20) {$q_{20}$};
			      \node[state, accepting, right of=q20] (q21) {$q_{21}$};

			      % Переходы
			      \path
			      (q10) edge node[above] {a} (q11)
			      (q20) edge node[below] {b} (q21)
			      (q90) edge node[above] {a} (q11)
			      (q90) edge node[below] {b} (q21)
			      ;
		      \end{tikzpicture}
		      \caption{Диаграмма состояний НКА \(M_9\)}
	      \end{figure}
\end{enumerate}
\newpage
Для выражения \(a+b\) строим КА \(M_{10} = (Q_{10}, \Sigma, \delta_{10}, q_{100}, F_{10})\) следующим образом:
\begin{enumerate}
	\item Множество состояний автомата \(M_{10}\) получается путем объединений множества состояний автоматов \(M_1\) и \(M_2\) и нового состояния \(q_{100}\)
	      \begin{align*}
		      Q_{10} = Q_3 \cup Q_4 \cup \cbr{q_{100}} = \cbr{q_{30}, q_{31}, q_{40}, q_{41}, q_{100}}
	      \end{align*}
	\item \(q_{100}\) --- начальное состояние;
	\item Конечные состояния определяются как объединение конечных состояний \(M_3\) и \(M_4\)
	      \begin{align*}
		      F_{10} = F_{3} \cup F_{4} = \cbr{q_{31}, q_{41}}
	      \end{align*}
	\item Множество переходов \(\delta\) строится:
	      \begin{align*}
		      \begin{array}{ll}
			      \delta_{10}(q_{100}, a) = \cbr{q_{31}} & \delta_{10}(q_{100}, b) = \cbr{q_{41}} \\
			      \delta_{10}(q_{30}, a) = \cbr{q_{31}}                                           \\
			      \delta_{10}(q_{40}, b) = \cbr{q_{41}}
		      \end{array}
	      \end{align*}
	      Граф переходов построенного КА \(M_{10}\) примет вид
	      \begin{figure}[h!]
		      \centering
		      \begin{tikzpicture}[
				      ->,
				      >=stealth',
				      node distance=2.5cm,
				      every state/.style={thick, fill=gray!10},
				      initial text={Начало}
			      ]

			      % Состояния
			      \node[state, initial] (q100) {$q_{100}$};

			      \node[state, above  of=q100] (q30) {$q_{30}$};
			      \node[state, accepting, right of=q30] (q31) {$q_{31}$};

			      \node[state, below  of=q100] (q40) {$q_{40}$};
			      \node[state, accepting, right of=q40] (q41) {$q_{41}$};

			      % Переходы
			      \path
			      (q30) edge node[above] {a} (q31)
			      (q40) edge node[below] {b} (q41)
			      (q100) edge node[above] {a} (q31)
			      (q100) edge node[below] {b} (q41)
			      ;
		      \end{tikzpicture}
		      \caption{Диаграмма состояний НКА \(M_{10}\)}
	      \end{figure}
\end{enumerate}
\newpage
Для выражения \(b+c\) строим КА \(M_{11} = (Q_{11}, \Sigma, \delta_{11}, q_{110}, F_{11})\) следующим образом:
\begin{enumerate}
	\item Множество состояний автомата \(M_{11}\) получается путем объединений множества состояний автоматов \(M_1\) и \(M_2\) и нового состояния \(q_{110}\)
	      \begin{align*}
		      Q_{11} = Q_1 \cup Q_2 \cup \cbr{q_{110}} = \cbr{q_{50}, q_{51}, q_{60}, q_{61}, q_{110}}
	      \end{align*}
	\item \(q_{110}\) --- начальное состояние;
	\item Конечные состояния определяются как объединение конечных состояний \(M_5\) и \(M_6\)
	      \begin{align*}
		      F_{11} = F_{5} \cup F_{6} = \cbr{q_{51}, q_{61}}
	      \end{align*}
	\item Множество переходов \(\delta\) строится:
	      \begin{align*}
		      \begin{array}{ll}
			      \delta_{11}(q_{110}, b) = \cbr{q_{51}} & \delta_{11}(q_{110}, c) = \cbr{q_{61}} \\
			      \delta_{11}(q_{50}, b) = \cbr{q_{51}}                                           \\
			      \delta_{11}(q_{60}, c) = \cbr{q_{61}}
		      \end{array}
	      \end{align*}
	      Граф переходов построенного КА \(M_{11}\) примет вид
	      \begin{figure}[h!]
		      \centering
		      \begin{tikzpicture}[
				      ->,
				      >=stealth',
				      node distance=2.5cm,
				      every state/.style={thick, fill=gray!10},
				      initial text={Начало}
			      ]

			      % Состояния
			      \node[state, initial] (q110) {$q_{110}$};

			      \node[state, above  of=q110] (q50) {$q_{50}$};
			      \node[state, accepting, right of=q50] (q51) {$q_{51}$};

			      \node[state, below  of=q110] (q60) {$q_{60}$};
			      \node[state, accepting, right of=q60] (q61) {$q_{61}$};

			      % Переходы
			      \path
			      (q50) edge node[above] {b} (q51)
			      (q60) edge node[below] {c} (q61)
			      (q110) edge node[above] {b} (q51)
			      (q110) edge node[below] {c} (q61)
			      ;
		      \end{tikzpicture}
		      \caption{Диаграмма состояний НКА \(M_{11}\)}
	      \end{figure}
\end{enumerate}

\newpage
Для выражения \(b+c\) строим КА \(M_{12} = (Q_{12}, \Sigma, \delta_{12}, q_{120}, F_{12})\) следующим образом:
\begin{enumerate}
	\item Множество состояний автомата \(M_{12}\) получается путем объединений множества состояний автоматов \(M_1\) и \(M_2\) и нового состояния \(q_{120}\)
	      \begin{align*}
		      Q_{12} = Q_1 \cup Q_2 \cup \cbr{q_{120}} = \cbr{q_{70}, q_{71}, q_{80}, q_{81}, q_{120}}
	      \end{align*}
	\item \(q_{120}\) --- начальное состояние;
	\item Конечные состояния определяются как объединение конечных состояний \(M_7\) и \(M_8\)
	      \begin{align*}
		      F_{12} = F_{7} \cup F_{8} = \cbr{q_{71}, q_{81}}
	      \end{align*}
	\item Множество переходов \(\delta\) строится:
	      \begin{align*}
		      \begin{array}{ll}
			      \delta_{12}(q_{120}, b) = \cbr{q_{71}} & \delta_{12}(q_{120}, c) = \cbr{q_{81}} \\
			      \delta_{12}(q_{70}, b) = \cbr{q_{71}}                                           \\
			      \delta_{12}(q_{80}, c) = \cbr{q_{81}}
		      \end{array}
	      \end{align*}
	      Граф переходов построенного КА \(M_{12}\) примет вид
	      \begin{figure}[h!]
		      \centering
		      \begin{tikzpicture}[
				      ->,
				      >=stealth',
				      node distance=2.5cm,
				      every state/.style={thick, fill=gray!10},
				      initial text={Начало}
			      ]

			      % Состояния
			      \node[state, initial] (q120) {$q_{120}$};

			      \node[state, above  of=q120] (q70) {$q_{70}$};
			      \node[state, accepting, right of=q70] (q71) {$q_{71}$};

			      \node[state, below  of=q120] (q80) {$q_{80}$};
			      \node[state, accepting, right of=q80] (q81) {$q_{81}$};

			      % Переходы
			      \path
			      (q70) edge node[above] {b} (q71)
			      (q80) edge node[below] {c} (q81)
			      (q120) edge node[above] {b} (q71)
			      (q120) edge node[below] {c} (q81)
			      ;
		      \end{tikzpicture}
		      \caption{Диаграмма состояний НКА \(M_{12}\)}
	      \end{figure}
\end{enumerate}

\newpage
Для выражения \((a+b)(a+b)\) строим КА \(M_{13} = (Q_{13}, \Sigma, \delta_{13}, q_{130}, F_{13})\):
\begin{enumerate}
	\item множество состояний автомата \(M_{13}\) получается путём объединения множеств состояний исходных автоматов
	      \begin{align*}
		      Q_{13} = Q_{9} \cup Q_{10} = \cbr{q_{10}, q_{11}, q_{20}, q_{21}, q_{90},q_{30}, q_{31}, q_{40}, q_{41}, q_{100} };
	      \end{align*}
	\item начальным состоянием результирующего автомата \(M_{13}\) будет начальное состояние автомата \(M_9\)
	      \begin{align*}
		      q_{130} \equiv q_{90};
	      \end{align*}
	\item множество заключительных состояний \(F_{13}\) будет содержать только множество заключительных состояний автомата \(M_{10}\)
	      \begin{align*}
		      F_{13} = F_{10} = \cbr{q_{31}, q_{41}}
	      \end{align*}
	\item множество переходов \(\delta_{13}\) автомата \(M_{13}\) будет содержать переходы автомата \(M_{9}\) кроме переходов из заключительных состояний
	      \begin{align*}
		      \begin{array}{ll}
			      \delta_{13}(q_{90}, a)  = \delta_9(q_{90}, a) = \cbr{q_{11}} & \delta_{13}(q_{90}, b)  =  \delta_9(q_{90}, b) = \cbr{q_{21}} \\
			      \delta_{13}(q_{10}, a)  = \delta_9(q_{10}, a) = \cbr{q_{11}} & \delta_{13}(q_{20}, b)  = \delta_9(q_{20}, b) = \cbr{q_{21}},
		      \end{array}
	      \end{align*}
	      а также добавляются переходы из заключительных состояний первого автомата в состояния второго, в которые имеются переходы из начальных состояний второго автомата
	      \begin{align*}
		      \begin{array}{ll}
			      \delta_{13}(q_{11}, a)  = \varnothing \cup \cbr{q_{31}} = \cbr{q_{31}} & \delta_{13}(q_{11}, b)  = \varnothing \cup \cbr{q_{41}} = \cbr{q_{41}}  \\
			      \delta_{13}(q_{21}, a)  = \varnothing \cup \cbr{q_{31}} = \cbr{q_{31}} & \delta_{13}(q_{21}, b)  = \varnothing \cup \cbr{q_{41}} = \cbr{q_{41}}.
		      \end{array}
	      \end{align*}
	      Кроме этого добавляются все состояния автомата \(M_{10}\)
	      \begin{align*}
		      \begin{array}{ll}
			      \delta_{13}(q_{100}, a)=      \delta_{10}(q_{100}, a) = \cbr{q_{31}} & \delta_{13}(q_{100}, b) =\delta_{10}(q_{100}, b) = \cbr{q_{41}} \\
			      \delta_{13}(q_{30}, a) =      \delta_{10}(q_{30}, a) = \cbr{q_{31}}  & \delta_{13}(q_{40}, b)  =\delta_{10}(q_{40}, b) = \cbr{q_{41}}
		      \end{array}
	      \end{align*}
\end{enumerate}
Граф переходов построенного КА \(M_{13}\) примет вид:
\begin{figure}[h!]
	\centering
	\begin{tikzpicture}[
			->,
			>=stealth',
			node distance=2.0cm,
			every state/.style={thick, fill=gray!10},
			initial text={Начало}
		]

		% Состояния
		\node[state, initial] (q90) {$q_{90}$};

		\node[state, above  of=q90] (q10) {$q_{10}$};
		\node[state,  right of=q10] (q11) {$q_{11}$};

		\node[state, below  of=q90] (q20) {$q_{20}$};
		\node[state,  right of=q20] (q21) {$q_{21}$};

		\node[state, accepting, right of=q11] (q31) {$q_{31}$};
		\node[state, accepting, right of=q21] (q41) {$q_{41}$};
		\node[state, right of=q31] (q30) {$q_{30}$};
		\node[state, right of=q41] (q40) {$q_{40}$};
		\node[state, below of=q30] (q100) {$q_{100}$};

		% Переходы
		\path
		(q10) edge node[above] {a} (q11)
		(q20) edge node[below] {b} (q21)
		(q90) edge node[above] {a} (q11)
		(q90) edge node[below] {b} (q21)

		(q30) edge node[above] {a} (q31)
		(q40) edge node[below] {b} (q41)
		(q100) edge node[above] {a} (q31)
		(q100) edge node[below] {b} (q41)

		(q11) edge node[above] {a} (q31)
		(q21) edge[bend left=20] node[above] {a} (q31)

		(q11) edge[bend right=20] node[below] {b} (q41)
		(q21) edge node[below] {b} (q41)
		;
	\end{tikzpicture}
	\caption{Диаграмма состояний НКА \(M_{13}\)}
\end{figure}

\newpage
Для выражения \((b+c)(b+c)\) строим КА \(M_{14} = (Q_{14}, \Sigma, \delta_{14}, q_{140}, F_{14})\):
\begin{enumerate}
	\item множество состояний автомата \(M_{14}\) получается путём объединения множеств состояний исходных автоматов
	      \begin{align*}
		      Q_{14} = Q_{11} \cup Q_{12} = \cbr{q_{50}, q_{51}, q_{60}, q_{61}, q_{110}, q_{70}, q_{71}, q_{80}, q_{81}, q_{120}};
	      \end{align*}
	\item начальным состоянием результирующего автомата \(M_{14}\) будет начальное состояние автомата \(M_{11}\)
	      \begin{align*}
		      q_{140} \equiv q_{110};
	      \end{align*}
	\item множество заключительных состояний \(F_{14}\) будет содержать только множество заключительных состояний автомата \(M_{12}\)
	      \begin{align*}
		      F_{14} = F_{12} = \cbr{q_{71}, q_{81}}
	      \end{align*}
	\item множество переходов \(\delta_{14}\) автомата \(M_{14}\) будет содержать переходы автомата \(M_{11}\) кроме переходов из заключительных состояний
	      \begin{align*}
		      \begin{array}{ll}
			      \delta_{14}(q_{110}, b) = \delta_{11}(q_{110}, b) = \cbr{q_{51}} & \delta_{14}(q_{110}, c) = \delta_{11}(q_{110}, c) = \cbr{q_{61}} \\
			      \delta_{14}(q_{50}, b)  = \delta_{11}(q_{50}, b) = \cbr{q_{51}}  & \delta_{14}(q_{60}, c)  = \delta_{11}(q_{60}, c) = \cbr{q_{61}}
		      \end{array}
	      \end{align*}
	      а также добавляются переходы из заключительных состояний первого автомата в состояния второго, в которые имеются переходы из начальных состояний второго автомата
	      \begin{align*}
		      \begin{array}{ll}
			      \delta_{14}(q_{51}, b)  = \varnothing \cup \cbr{q_{71}} = \cbr{q_{71}} & \delta_{14}(q_{51}, c)  = \varnothing \cup \cbr{q_{81}} = \cbr{q_{81}}  \\
			      \delta_{14}(q_{61}, b)  = \varnothing \cup \cbr{q_{71}} = \cbr{q_{71}} & \delta_{14}(q_{61}, c)  = \varnothing \cup \cbr{q_{81}} = \cbr{q_{81}}.
		      \end{array}
	      \end{align*}
	      Кроме этого добавляются все состояния автомата \(M_{12}\)
	      \begin{align*}
		      \begin{array}{ll}
			      \delta_{14}(q_{120}, b) =\delta_{12}(q_{120}, b) = \cbr{q_{71}} & \delta_{14}(q_{120}, c) =\delta_{12}(q_{120}, c) = \cbr{q_{81}} \\
			      \delta_{14}(q_{70}, b)  =\delta_{12}(q_{70}, b) = \cbr{q_{71}}  & \delta_{14}(q_{80}, c)  =\delta_{12}(q_{80}, c) = \cbr{q_{81}}
		      \end{array}
	      \end{align*}
\end{enumerate}
Граф переходов построенного КА \(M_{14}\) примет вид:
\begin{figure}[h!]
	\centering
	\begin{tikzpicture}[
			->,
			>=stealth',
			node distance=2.0cm,
			every state/.style={thick, fill=gray!10},
			initial text={Начало}
		]

		% Состояния
		\node[state, initial] (q110) {$q_{110}$};

		\node[state, above  of=q110] (q50) {$q_{50}$};
		\node[state,  right of=q50] (q51) {$q_{51}$};

		\node[state, below  of=q110] (q60) {$q_{60}$};
		\node[state,  right of=q60] (q61) {$q_{61}$};

		\node[state, accepting, right of=q51] (q71) {$q_{71}$};
		\node[state, accepting, right of=q61] (q81) {$q_{81}$};
		\node[state, right of=q71] (q70) {$q_{70}$};
		\node[state, right of=q81] (q80) {$q_{80}$};
		\node[state, above of=q80] (q120) {$q_{120}$};

		% Переходы
		\path
		(q50) edge node[above] {b} (q51)
		(q60) edge node[below] {c} (q61)
		(q110) edge node[above] {b} (q11)
		(q110) edge node[below] {c} (q21)

		(q70) edge node[above] {b} (q71)
		(q80) edge node[below] {c} (q81)
		(q120) edge node[above] {b} (q71)
		(q120) edge node[below] {c} (q81)

		(q51) edge node[above] {b} (q71)
		(q61) edge[bend left=20] node[above] {b} (q71)

		(q51) edge[bend right=20] node[below] {c} (q81)
		(q61) edge node[below] {c} (q81)
		;
	\end{tikzpicture}
	\caption{Диаграмма состояний НКА \(M_{14}\)}
\end{figure}

\newpage
Для выражения \(((a+b)(a+b))^*\) строим КА \(M_{15}=(Q_{15}, \Sigma, \delta_{15}, q_{150}, F_{15})\):
\begin{enumerate}
	\item множество состояний конечного автомтата \(M_{13}\) переносится с добавлением нового состояния \(q_{150}\), состояние \(q_{150}\) --- начальное
	      \begin{align*}
		      Q_{15} = Q_{13} \cup \cbr{ q_{150}} = \cbr{q_{10}, q_{11}, q_{20}, q_{21}, q_{90},q_{30}, q_{31}, q_{40}, q_{41}, q_{100}, q_{150} }.
	      \end{align*}
	\item множество результирующих состояний переносится с добавлением нового состояния \(q_{150}\)
	      \begin{align*}
		      F_{15} = F_{13} \cup \cbr{q_{150}} = \cbr{q_{31}, q_{41}, q_{150}}
	      \end{align*}
	\item множество переходов  \(\delta_{15}\) сохраняет все те переходы из незаключительных состояний, что и в автомате \(M_{13}\)
	      \begin{align*}
		      \begin{array}{ll}
			      \delta_{15}(q_{90}, a)  = \delta_{13}(q_{90}, a) = \cbr{q_{11}} & \delta_{15}(q_{90}, b)  =  \delta_{13}(q_{90}, b) = \cbr{q_{21}} \\
			      \delta_{15}(q_{10}, a)  = \delta_{13}(q_{10}, a) = \cbr{q_{11}} & \delta_{15}(q_{20}, b)  = \delta_{13}(q_{20}, b) = \cbr{q_{21}}  \\
			      \delta_{15}(q_{100}, a)= \delta_{13}(q_{100}, a) = \cbr{q_{31}} & \delta_{15}(q_{100}, b) =\delta_{13}(q_{100}, b) = \cbr{q_{41}}  \\
			      \delta_{15}(q_{30}, a) = \delta_{13}(q_{30}, a) = \cbr{q_{31}}  & \delta_{15}(q_{40}, b)  =\delta_{13}(q_{40}, b) = \cbr{q_{41}}   \\
			      \delta_{15}(q_{11}, a) =\delta_{13}(q_{11}, a)  =  \cbr{q_{31}} & \delta_{15}(q_{11}, b)=\delta_{13}(q_{11}, b)  =  \cbr{q_{41}}   \\
			      \delta_{15}(q_{21}, a) =\delta_{13}(q_{21}, a)  =  \cbr{q_{31}} & \delta_{15}(q_{21}, b)=\delta_{13}(q_{21}, b)  =  \cbr{q_{41}},
		      \end{array}
	      \end{align*}
	      добавляются переходы из заключительных состояний автомата в состояния, в которые ведут переходы начального состояния автомата
	      \begin{align*}
		      \begin{array}{ll}
			      \delta_{15}(q_{31}, a) = \varnothing \cup \delta_{13}(q_{90}, a) = \cbr{q_{11}} & \delta_{15}(q_{31}, b) = \varnothing \cup \delta_{13}(q_{90}, b) = \cbr{q_{21}}  \\
			      \delta_{15}(q_{41}, a) = \varnothing \cup \delta_{13}(q_{90}, a) = \cbr{q_{11}} & \delta_{15}(q_{41}, b) = \varnothing \cup \delta_{13}(q_{90}, b) = \cbr{q_{21}},
		      \end{array}
	      \end{align*}
	      для нового начального состояния \(q_{150}\) переносятся все переходы из старого началльного состояния \(q_{90}\)
	      \begin{align*}
		      \begin{array}{ll}
			      \delta_{15}(q_{150}, a)  = \delta_{13}(q_{90}, a) = \cbr{q_{11}} & \delta_{15}(q_{150}, b)  =  \delta_{13}(q_{90}, b) = \cbr{q_{21}}.
		      \end{array}
	      \end{align*}
\end{enumerate}
Граф переходов построенного КА \(M_{15}\) примет вид:
\begin{figure}[h!]
	\centering
	\begin{tikzpicture}[
			->,
			>=stealth',
			node distance=2.0cm,
			every state/.style={thick, fill=gray!10},
			initial text={Начало}
		]

		% Состояния
		\node[state, initial, accepting] (q150) {$q_{150}$};

		\node[state, above  of=q150] (q10) {$q_{10}$};
		\node[state,  right of=q10] (q11) {$q_{11}$};

		\node[state, below  of=q150] (q20) {$q_{20}$};
		\node[state,  right of=q20] (q21) {$q_{21}$};

		\node[state, right of=q150] (q90) {$q_{90}$};

		\node[state, accepting, right= 3cm of q11] (q31) {$q_{31}$};
		\node[state, accepting, right= 3cm of q21] (q41) {$q_{41}$};
		\node[state, right of=q31] (q30) {$q_{30}$};
		\node[state, right of=q41] (q40) {$q_{40}$};
		\node[state, below of=q30] (q100) {$q_{100}$};

		% Переходы
		\path
		(q10) edge node[above] {a} (q11)
		(q20) edge node[below] {b} (q21)

		(q150) edge node[above] {a} (q11)
		(q150) edge node[below] {b} (q21)

		(q90) edge node[left] {a} (q11)
		(q90) edge node[left] {b} (q21)

		(q30) edge node[above] {a} (q31)
		(q40) edge node[below] {b} (q41)
		(q100) edge node[above] {a} (q31)
		(q100) edge node[below] {b} (q41)

		(q11) edge node[above] {a} (q31)
		(q21) edge[bend left=20] node[above] {a} (q31)

		(q11) edge[bend right=20] node[below] {b} (q41)
		(q21) edge node[below] {b} (q41)

		(q31) edge[bend right=20] node[above]{a} (q11)
		(q31) edge[bend left=20] node[below]{b} (q21)

		(q41) edge[bend left=20] node[below]{b} (q21)
		(q41) edge[bend right=20] node[above]{a} (q11)
		;
	\end{tikzpicture}
	\caption{Диаграмма состояний НКА \(M_{15}\)}
\end{figure}

\newpage
Для выражения \(((b+c)(b+c))^+\) строим КА \(M_{16} = (Q_{16}, \Sigma, \delta_{16}, q_{160}, F_{16})\):
\begin{enumerate}
	\item множество состояний конечного автомата \(M_{14}\) переносится с добавлением нового состояния \(q_{160}\), состояние \(q_{160}\) --- начальное
	      \begin{align*}
		      Q_{16} = Q_{14} \cup \cbr{q_{160}} =\cbr{q_{50}, q_{51}, q_{60}, q_{61}, q_{110}, q_{70}, q_{71}, q_{80}, q_{81}, q_{120}, q_{160}}.
	      \end{align*}
	\item множество результирующих состояний автомтата переносится без изменений
	      \begin{align*}
		      F_{16} = F_{14} = \cbr{q_{71}, q_{81}}
	      \end{align*}
	\item множество переходов \(\delta_{16}\) сохраняет все переходы из незаключительных состояний, что и в автомате \(M_{14}\)
	      \begin{align*}
		      \begin{array}{ll}
			      \delta_{16}(q_{110}, b) = \delta_{14}(q_{110}, b) = \cbr{q_{51}} & \delta_{16}(q_{110}, c) = \delta_{14}(q_{110}, c) = \cbr{q_{61}} \\
			      \delta_{16}(q_{50}, b)  = \delta_{14}(q_{50}, b)  = \cbr{q_{51}} & \delta_{16}(q_{60}, c)  = \delta_{14}(q_{60}, c)  = \cbr{q_{61}} \\
			      \delta_{16}(q_{51}, b)  = \delta_{14}(q_{51}, b)  = \cbr{q_{71}} & \delta_{16}(q_{51}, c)  = \delta_{14}(q_{51}, c)  = \cbr{q_{81}} \\
			      \delta_{16}(q_{61}, b)  = \delta_{14}(q_{61}, b)  = \cbr{q_{71}} & \delta_{16}(q_{61}, c)  = \delta_{14}(q_{61}, c)  = \cbr{q_{81}} \\
			      \delta_{16}(q_{120}, b) = \delta_{14}(q_{120}, b) = \cbr{q_{71}} & \delta_{16}(q_{120}, c) = \delta_{14}(q_{120}, c) = \cbr{q_{81}} \\
			      \delta_{16}(q_{70}, b)  = \delta_{14}(q_{70}, b) = \cbr{q_{71}}  & \delta_{16}(q_{80}, c) =  \delta_{14}(q_{80}, c) = \cbr{q_{81}},
		      \end{array}
	      \end{align*}
	      добавляются переходы из заключительных состояний автомата в состояния, в которые ведут переходы начального состояния автомата
	      \begin{align*}
		      \begin{array}{ll}
			      \delta_{16}(q_{71}, b) = \varnothing \cup \delta_{14}(q_{110}, b) = \cbr{q_{51}} & \delta_{16}(q_{71}, c) = \varnothing \cup \delta_{14}(q_{110}, c) = \cbr{q_{61}}  \\
			      \delta_{16}(q_{81}, b) = \varnothing \cup \delta_{14}(q_{110}, b) = \cbr{q_{51}} & \delta_{16}(q_{81}, c) = \varnothing \cup \delta_{14}(q_{110}, c) = \cbr{q_{61}},
		      \end{array}
	      \end{align*}
	      для нового начального состояния \(q_{160}\) переносятся все переходы из старого начального состояния \(q_{110}\)
	      \begin{align*}
		      \begin{array}{ll}
			      \delta_{16}(q_{160}, b) = \delta_{16}(q_{110}, b) = \cbr{q_{51}} & \delta_{16}(q_{160}, c) = \delta_{16}(q_{110}, c) = \cbr{q_{61}}
		      \end{array}
	      \end{align*}
\end{enumerate}
Граф переходов построенного КА \(M_{16}\) примет вид:
\begin{figure}[h!]
	\centering
	\begin{tikzpicture}[
			->,
			>=stealth',
			node distance=2.0cm,
			every state/.style={thick, fill=gray!10},
			initial text={Начало}
		]

		% Состояния
		\node[state, initial] (q160) {$q_{160}$};
		\node[state, right of=q160] (q110) {$q_{110}$};

		\node[state, above  of=q160] (q50) {$q_{50}$};
		\node[state,  right of=q50] (q51) {$q_{51}$};

		\node[state, below  of=q160] (q60) {$q_{60}$};
		\node[state,  right of=q60] (q61) {$q_{61}$};

		\node[state, accepting, right = 3cm of q51] (q71) {$q_{71}$};
		\node[state, accepting, right = 3cm of q61] (q81) {$q_{81}$};
		\node[state, right of=q71] (q70) {$q_{70}$};
		\node[state, right of=q81] (q80) {$q_{80}$};
		\node[state, above of=q80] (q120) {$q_{120}$};

		% Переходы
		\path
		(q110) edge node[left] {b} (q51)
		(q110) edge node[left] {c} (q61)

		(q50) edge node[above] {b} (q51)
		(q60) edge node[below] {c} (q61)
		(q160) edge node[above] {b} (q11)
		(q160) edge node[below] {c} (q21)

		(q70) edge node[above] {b} (q71)
		(q80) edge node[below] {c} (q81)
		(q120) edge node[above] {b} (q71)
		(q120) edge node[below] {c} (q81)

		(q51) edge node[above] {b} (q71)
		(q61) edge[bend left=20] node[above] {b} (q71)

		(q51) edge[bend right=20] node[below] {c} (q81)
		(q61) edge node[below] {c} (q81)

		(q71) edge[bend right=20] node[above] {b} (q51)
		(q81) edge[bend left=20] node[below] {c} (q61)

		(q71) edge[bend left=20] node[below]{c} (q61)
		(q81) edge[bend right=20] node[above]{b} (q51)
		;
	\end{tikzpicture}
	\caption{Диаграмма состояний НКА \(M_{16}\)}
\end{figure}

\newpage
Для выражения \(((a+b)(a+b))^+((b+c)(b+c))^*\) строим КА \(M_{17}=(Q_{17}, \Sigma, \delta_{17}, q_{170}, F_{17})\):
\begin{enumerate}
	\item Множество состояний автомата \(M_{17}\) получается путём объединения множеств состояний исходных автоматов
	      \begin{align*}
		      Q_{17} = Q_{15} \cup Q_{16} =  \cbr{
			      \begin{array}{l}
				      q_{10}, q_{11}, q_{20}, q_{21}, q_{90},q_{30}, q_{31}, q_{40}, q_{41}, q_{100}, q_{150}, \\
				      q_{50}, q_{51}, q_{60}, q_{61}, q_{110}, q_{70}, q_{71}, q_{80}, q_{81}, q_{120}, q_{160}
			      \end{array}}
	      \end{align*}
	\item начальным состоянием результирующего автомата \(M_{17}\) будет начальное состояние автомата
\end{enumerate}


% S -> UUTTTb -> UUTTAb -> UUTAAAb -> UUAATAAb -> UUAAAATAb -> UUAAAAAATb -> UUAAAAAAAb -> 
% -> UAAUAAAAb -> UAAAAUAb -> UAAAAb -> AAUAb -> AAb -> aab

\subsection{Определение детерменированности построенных автоматов \(M_1\), \(M_2\), \(M_3\)}
\begin{enumerate}
	\item Автомат \(M_1\) --- НКА, так как есть переходы
	      \begin{align*}
		      \delta_1(H,b) = \cbr{S_9, S_{11}} \\
		      \delta_1(S_{15},b) = \cbr{S_9, S_{11}}
	      \end{align*}
	\item Автомат \(M_2\) --- НКА, так как есть переходы
	      \begin{align*}
		      \delta_2(S_{15}, b) = \cbr{S_{10}, S_{12}} \\
		      \delta_2(S_{12}, b) = \cbr{S_{16}, F}
	      \end{align*}
	\item Автомат \(M_3 \equiv M_{17}\) --- НКА, так как есть переходы
	      \begin{align*}
		      \delta_3(q_{31}, b) = \cbr{q_{21}, q_{51}} \\
		      \delta_3(q_{41}, b) = \cbr{q_{21}, q_{51}} \\
		      \delta_3(q_{150}, b) = \cbr{q_{21}, q_{51}}
	      \end{align*}
\end{enumerate}
Все три представленных автомата являются недетерменированными.

\subsection{Построение детерменированных конечных автоматов для НКА \(M_1\), \(M_2\), \(M_3\)}

\begin{itemize}
	\item Ножество состояний \(Q'\) результирующего автомата ДКА состояит из всех подмножеств \(Q\) исходного автомата. Каждое состояние \(Q'\) обозначается как \([A_1,\dots,A_n]\), где \(A_i \in Q\). Тогда получаем число различных сочетаний
	      \begin{align*}
		      |Q'| = \sum_{k=1}^n C^k_n = 2^n - 1
	      \end{align*}
	\item Начальное состояние имеет вид (\(H\) --- начальное состояние автомата \(M\))
	      \begin{align*}
		      q'_0  \equiv [H]
	      \end{align*}
	\item Множество конечных состояний (конечные состояния исходного автомата \(F=\cbr{F_1, \dots, F_n}\)) имеет вид
	      \begin{align*}
		      F' = \cbr{[q_1, \dots, q_j,q_{j+1},\dots,q_{j+k} ]}    \\
		      \cbr{q_1, \dots, q_j} \subset \cbr{H, F_1, \dots, F_n} \\
		      \cbr{q_{j+1},\dots,q_j} \subset Q \setminus \cbr{H, F_1, \dots, F_n}
	      \end{align*}
\end{itemize}
\subsubsection{ДКА для \(M_1\)}
Переходы определяются как
\begin{align*}
	\begin{array}{ll}
		\gendelta{S_9}{a}{S_{15}}{\maqa}                               \\
		\gendelta{H}{a}{S_9}{\maqa,S_{15}}                             \\
		\gendelta{S_{15}}{a}{S_9}{\maqa}                               \\
		\gendelta{S_9S_{15}}{a}{S_9S_{15}}{\maqa,H}                    \\
		\gendelta{S_9H}{a}{S_9S_{15}}{\maqa}                           \\
		\gendelta{S_{11}}{c}{S_{16}}{\maqc}                            \\
		\gendelta{S_{15}}{c}{S_{11}}{\maqc, H, S_{16}}                 \\
		\gendelta{H}{c}{S_{11}}{\maqc, S_{16}}                         \\
		\gendelta{S_{16}}{c}{S_{11}}{\maqc}                            \\
		\gendelta{HS_{11}}{c}{S_{11}S_{16}}{\maqc,S_{15},S_{16}}       \\
		\gendelta{S_{15}S_{11}}{c}{S_{11}S_{16}}{\maqc,S_{16}}         \\
		\gendelta{S_{16}S_{11}}{c}{S_{11}S_{16}}{\maqc}                \\
		\gendelta{S_{11}}{b}{S_{16}}{}                                 \\
		\gendelta{S_{9}}{b}{S_{15}}{}                                  \\
		\gendelta{H}{b}{S_9S_{11}}{S_{15},S_{16}}                      \\
		\gendelta{S_{15}}{b}{S_9S_{11}}{S_{16}}                        \\
		\gendelta{S_{16}}{b}{S_{11}}{}                                 \\
		\gendelta{S_9H}{b}{S_9S_{11}S_{15}}{S_{15},S_{16}}             \\
		\gendelta{S_9S_{15}}{b}{S_9S_{11}S_{15}}{S_{16}}               \\
		\gendelta{S_{11}H}{b}{S_9S_{11}S_{16}}{S_{15},S_{16}}          \\
		\gendelta{S_{11}S_{15}}{b}{S_9S_{11}S_{16}}{S_{16}}            \\
		\gendelta{S_9S_{16}}{b}{S_{15}S_{11}}{}                        \\
		\gendelta{S_9S_{11}H}{b}{S_9S_{15}S_{11}S_{16}}{S_{16},S_{15}} \\
		\gendelta{S_9S_{11}S_{15}}{b}{S_9S_{15}S_{11}S_{16}}{S_{16}}   \\
		\gendelta{S_9S_{11}}{b}{S_{15}S_{16}}{}
	\end{array}
\end{align*}

\subsubsection{ДКА для \(M_2\)}
Переходы определяются как
\begin{align*}
	\begin{array}{ll}
		\gendelta{S_{10}}{a}{S_{15}S_{16}}{\mbqa, S_{16}}                     \\
		\gendelta{S_{15}}{a}{S_{10}}{\mbqa, S_{16}}                           \\
		\gendelta{S_{10}S_{15}}{a}{S_{10}S_{15}S_{16}}{\mbqa, S_{16}}         \\
		\gendelta{S_{15}}{c}{S_{12}}{\mbqc, S_{16}, F}                        \\
		\gendelta{S_{16}}{c}{S_{12}}{\mbqc, F}                                \\
		\gendelta{S_{12}}{c}{S_{16}F}{\mbqc, F}                               \\
		\gendelta{S_{12}S_{15}}{c}{S_{12}S_{16}F}{\mbqc, S_{16}, F}           \\
		\gendelta{S_{12}S_{16}}{c}{S_{12}S_{16}F}{\mbqc, F}                   \\
		\gendelta{S_{16}}{b}{S_{12}}{F}                                       \\
		\gendelta{S_{12}}{b}{FS_{16}}{F}                                      \\
		\gendelta{S_{10}}{b}{S_{15}S_{16}}{F}                                 \\
		\gendelta{S_{15}}{b}{S_{10}S_{12}}{S_{16},F}                          \\
		\gendelta{S_{15}S_{12}}{b}{S_{10}S_{12}S_{16}F}{S_{16},F}             \\
		\gendelta{S_{15}S_{10}}{b}{S_{10}S_{12}S_{15}S_{16}}{S_{16},F}        \\
		\gendelta{S_{10}S_{16}}{b}{S_{12}S_{15}S_{16}}{F}                     \\
		\gendelta{S_{12}S_{10}S_{15}}{b}{S_{10}S_{12}S_{15}S_{16}F}{S_{16},F} \\
		\gendelta{S_{12}S_{10}}{b}{S_{15}S_{16}F}{F}
	\end{array}
\end{align*}

\newpage
\subsubsection{ДКА для \(M_3\)}
\subsubsubsection{Удаление недостижимых символов}
Для удобства, сначала удалим недостижимые состояния:
\begin{align*}
	R = \cbr{q_{150}}, P_0 = \cbr{q_{150}}                                                                                                                                      \\
	P_1 = \cbr{q_{11},q_{21}, q_{51}, q_{61}}, R \setminus P_1 \neq \varnothing \Longrightarrow R = \cbr{q_{150}, q_{11},q_{21}, q_{51}, q_{61}}                                \\
	P_2 = \cbr{q_{31},q_{41}, q_{71}, q_{81}}, R \setminus P_2 \neq \varnothing \Longrightarrow R = \cbr{q_{150}, q_{11},q_{21}, q_{31}, q_{41} q_{51}, q_{61}, q_{71}, q_{81}} \\
	P_3 = \cbr{q_{51}, q_{61}}, R \setminus P_3 = \varnothing \Longrightarrow R = \cbr{q_{150}, q_{11},q_{21}, q_{31}, q_{41} q_{51}, q_{61}, q_{71}, q_{81}}                   \\
\end{align*}
Автомат после удаления недостижимых состояний
\begin{align*}
	M_3 = \rbr{\cbr{q_{150}, q_{11},q_{21}, q_{31}, q_{41} q_{51}, q_{61}, q_{71}, q_{81}}, \Sigma, \delta_3, q_{150}, \cbr{q_{71}, q_{81}}} \\
	\begin{array}{lll}
		\delta_{3}(q_{11}, a) =  \cbr{q_{31}} & \delta_{3}(q_{11}, b)=  \cbr{q_{41}}                                                  \\
		\delta_{3}(q_{21}, a) =  \cbr{q_{31}} & \delta_{3}(q_{21}, b)=  \cbr{q_{41}}                                                  \\
		\delta_{3}(q_{31}, a) = \cbr{q_{11}}  & \delta_{3}(q_{31}, b) = \cbr{q_{21}, q_{51}}  & \delta_{3}(q_{31}, c) = \cbr{q_{61}}  \\
		\delta_{3}(q_{41}, a) = \cbr{q_{11}}  & \delta_{3}(q_{41}, b) = \cbr{q_{21}, q_{51}}  & \delta_{3}(q_{41}, c) = \cbr{q_{61}}  \\
		\delta_{3}(q_{150}, a) = \cbr{q_{11}} & \delta_{3}(q_{150}, b) = \cbr{q_{21}, q_{51}} & \delta_{3}(q_{150}, c) = \cbr{q_{61}} \\
		\delta_{3}(q_{51}, b)  = \cbr{q_{71}} & \delta_{3}(q_{51}, c)  = \cbr{q_{81}}                                                 \\
		\delta_{3}(q_{61}, b)  = \cbr{q_{71}} & \delta_{3}(q_{61}, c)  = \cbr{q_{81}}                                                 \\
		\delta_{3}(q_{71}, b) = \cbr{q_{51}}  & \delta_{3}(q_{71}, c) =\cbr{q_{61}}                                                   \\
		\delta_{3}(q_{81}, b) = \cbr{q_{51}}  & \delta_{3}(q_{81}, c) =\cbr{q_{61}}                                                   \\
	\end{array}
\end{align*}

\begin{figure}[h!]
	\centering
	\begin{tikzpicture}[
			->,
			>=stealth',
			node distance=2.0cm,
			every state/.style={thick, fill=gray!10},
			initial text={Начало}
		]

		% Состояния
		\node[state, initial] (q150) {$q_{150}$};
		\node[state,  right of=q10] (q11) {$q_{11}$};
		\node[state,  right of=q20] (q21) {$q_{21}$};
		\node[state, right= of q11] (q31) {$q_{31}$};
		\node[state, right= of q21] (q41) {$q_{41}$};
		% Переходы
		\path
		(q150) edge node[above] {a} (q11)
		(q150) edge node[below] {b} (q21)
		(q11) edge node[below] {a} (q31)
		(q21) edge[bend left=20] node[above] {a} (q31)
		(q11) edge[bend right=20] node[below] {b} (q41)
		(q21) edge node[above] {b} (q41)
		(q31) edge[bend right=20] node[above]{a} (q11)
		(q31) edge[bend left=20] node[below]{b} (q21)
		(q41) edge[bend left=20] node[below]{b} (q21)
		(q41) edge[bend right=20] node[above]{a} (q11)
		;
		\node[state, above of=q110] (q51) {$q_{51}$};
		\node[state, below of=q110] (q61) {$q_{61}$};
		\node[state, accepting, right = of q51] (q71) {$q_{71}$};
		\node[state, accepting, right = of q61] (q81) {$q_{81}$};
		% Переходы
		\path
		(q51) edge node[below] {b} (q71)
		(q61) edge[bend left=20] node[above] {b} (q71)
		(q51) edge[bend right=20] node[below] {c} (q81)
		(q61) edge node[above] {c} (q81)
		(q71) edge[bend right=20] node[above] {b} (q51)
		(q81) edge[bend left=20] node[below] {c} (q61)
		(q71) edge[bend left=20] node[below]{c} (q61)
		(q81) edge[bend right=20] node[above]{b} (q51)
		;

		\path
		(q41) edge[bend right=30] node[below]{c} (q61)
		(q31) edge[bend left=30] node[above]{b} (q51)
		(q41) edge[bend left=20] node[above]{b} (q51)
		(q31) edge[bend right=20] node[below]{c} (q61)
		(q150) edge[bend right=60] node[below]{c} (q61)
		(q150) edge[bend left=60] node[above]{b} (q51)
		;

	\end{tikzpicture}
	\caption{Диаграмма состояний НКА \(M_{17}\)}
\end{figure}

\newpage
\subsubsubsection{Построение ДКА}
Переходы определяются как
\begin{align*}
	\begin{array}{ll}
		\gendelta{q_{150}}{a}{q_{11}}{\mcqa, q_{41}, q_{31}}                                \\
		\gendelta{q_{41}}{a}{q_{11}}{\mcqa, q_{31}}                                         \\
		\gendelta{q_{31}}{a}{q_{11}}{\mcqa}                                                 \\
		\gendelta{q_{11}}{a}{q_{31}}{\mcqa, q_{21}}                                         \\
		\gendelta{q_{21}}{a}{q_{31}}{\mcqa}                                                 \\
		\gendelta{q_{11}q_{150}}{a}{q_{11}q_{31}}{\mcqa,q_{31}, q_{41},q_{21}}              \\
		\gendelta{q_{11}q_{31}}{a}{q_{11}q_{31}}{\mcqa, q_{41},q_{21}}                      \\
		\gendelta{q_{11}q_{41}}{a}{q_{11}q_{31}}{\mcqa, q_{21}}                             \\
		\gendelta{q_{21}q_{150}}{a}{q_{11}q_{31}}{\mcqa,q_{31}, q_{41}}                     \\
		\gendelta{q_{21}q_{31}}{a}{q_{11}q_{31}}{\mcqa, q_{41}}                             \\
		\gendelta{q_{21}q_{41}}{a}{q_{11}q_{31}}{\mcqa}                                     \\
		\gendelta{q_{150}}{c}{q_{61}}{\mcqc,q_{31},q_{41},q_{71},q_{81}}                    \\
		\gendelta{q_{31}}{c}{q_{61}}{\mcqc,q_{41},q_{71},q_{81}}                            \\
		\gendelta{q_{41}}{c}{q_{61}}{\mcqc,q_{71},q_{81}}                                   \\
		\gendelta{q_{71}}{c}{q_{61}}{\mcqc,q_{81}}                                          \\
		\gendelta{q_{81}}{c}{q_{61}}{\mcqc}                                                 \\
		\gendelta{q_{51}}{c}{q_{81}}{\mcqc, q_{61}}                                         \\
		\gendelta{q_{61}}{c}{q_{81}}{\mcqc}                                                 \\
		\gendelta{q_{51}q_{150}}{c}{q_{61}q_{81}}{\mcqc,q_{31},q_{41},q_{71},q_{81},q_{61}} \\
		\gendelta{q_{51}q_{31}}{c}{q_{61}q_{81}}{\mcqc,q_{41},q_{71},q_{81},q_{61}}         \\
		\gendelta{q_{51}q_{41}}{c}{q_{61}q_{81}}{\mcqc,q_{71},q_{81},q_{61}}                \\
		\gendelta{q_{51}q_{71}}{c}{q_{61}q_{81}}{\mcqc,q_{81},q_{61}}                       \\
		\gendelta{q_{51}q_{81}}{c}{q_{61}q_{81}}{\mcqc,q_{61}}                              \\
		\gendelta{q_{61}q_{150}}{c}{q_{61}q_{81}}{\mcqc,q_{31},q_{41},q_{71},q_{81}}        \\
		\gendelta{q_{61}q_{31}}{c}{q_{61}q_{81}}{\mcqc,q_{41},q_{71},q_{81}}                \\
		\gendelta{q_{61}q_{41}}{c}{q_{61}q_{81}}{\mcqc,q_{71},q_{81}}                       \\
		\gendelta{q_{61}q_{71}}{c}{q_{61}q_{81}}{\mcqc,q_{81}}                              \\
		\gendelta{q_{61}q_{81}}{c}{q_{61}q_{81}}{\mcqc}                                     \\
		\gendelta{q_{11}}{b}{q_{41}}{q_{21}}                                                \\
		\gendelta{q_{21}}{b}{q_{41}}{}                                                      \\
		\gendelta{q_{51}}{b}{q_{71}}{q_{61}}                                                \\
		\gendelta{q_{61}}{b}{q_{71}}{}                                                      \\
		\gendelta{q_{150}}{b}{q_{21}q_{51}}{q_{31},q_{41},q_{81}, q_{71}}                   \\
		\gendelta{q_{31}}{b}{q_{21}q_{51}}{q_{41},q_{81}, q_{71}}                           \\
		\gendelta{q_{41}}{b}{q_{21}q_{51}}{q_{81}, q_{71}}                                  \\
		\gendelta{q_{81}}{b}{q_{51}}{q_{71}}                                                \\
		\gendelta{q_{71}}{b}{q_{51}}{}                                                      \\
	\end{array}
\end{align*}

\begin{align*}
	\begin{array}{ll}
		\gendelta{q_{71}q_{51}}{b}{q_{51}q_{71}}{q_{81},q_{61}}                                                \\
		\gendelta{q_{81}q_{51}}{b}{q_{51}q_{71}}{q_{61}}                                                       \\
		\gendelta{q_{71}q_{61}}{b}{q_{51}q_{71}}{q_{81}}                                                       \\
		\gendelta{q_{81}q_{61}}{b}{q_{51}q_{71}}{}                                                             \\
		\gendelta{q_{51}q_{11}}{b}{q_{41}q_{71}}{q_{61},q_{21}}                                                \\
		\gendelta{q_{61}q_{11}}{b}{q_{41}q_{71}}{q_{21}}                                                       \\
		\gendelta{q_{51}q_{21}}{b}{q_{41}q_{71}}{q_{61}}                                                       \\
		\gendelta{q_{61}q_{21}}{b}{q_{41}q_{71}}{}                                                             \\
		\gendelta{q_{71}q_{11}}{b}{q_{51}q_{41}}{ q_{81}, q_{21}}                                              \\
		\gendelta{q_{81}q_{11}}{b}{q_{51}q_{41}}{ q_{21}}                                                      \\
		\gendelta{q_{71}q_{21}}{b}{q_{51}q_{41}}{ q_{81}}                                                      \\
		\gendelta{q_{81}q_{21}}{b}{q_{51}q_{41}}{}                                                             \\
		\gendelta{q_{150}q_{11}}{b}{q_{21}q_{51}q_{41}}{q_{31},q_{41},q_{81},q_{71},q_{21}}                    \\
		\gendelta{q_{31}q_{11}}{b}{q_{21}q_{51}q_{41}}{q_{41},q_{81},q_{71},q_{21}}                            \\
		\gendelta{q_{41}q_{11}}{b}{q_{21}q_{51}q_{41}}{q_{81},q_{71},q_{21}}                                   \\
		\gendelta{q_{150}q_{21}}{b}{q_{21}q_{51}q_{41}}{q_{31},q_{41},q_{81},q_{71}}                           \\
		\gendelta{q_{31}q_{21}}{b}{q_{21}q_{51}q_{41}}{q_{41},q_{81},q_{71}}                                   \\
		\gendelta{q_{41}q_{21}}{b}{q_{21}q_{51}q_{41}}{q_{81},q_{71}}                                          \\
		\gendelta{q_{150}q_{51}}{b}{q_{21}q_{51}q_{71}}{q_{31},q_{41},q_{81},q_{71},q_{61}}                    \\
		\gendelta{q_{31}q_{51}}{b}{q_{21}q_{51}q_{71}}{q_{41},q_{81},q_{71},q_{61}}                            \\
		\gendelta{q_{41}q_{51}}{b}{q_{21}q_{51}q_{71}}{q_{81},q_{71},q_{61}}                                   \\
		\gendelta{q_{150}q_{61}}{b}{q_{21}q_{51}q_{71}}{q_{31},q_{41},q_{81},q_{71}}                           \\
		\gendelta{q_{31}q_{61}}{b}{q_{21}q_{51}q_{71}}{q_{41},q_{81},q_{71}}                                   \\
		\gendelta{q_{41}q_{61}}{b}{q_{21}q_{51}q_{71}}{q_{81},q_{71}}                                          \\
		\gendelta{q_{11}q_{71}q_{51}}{b}{q_{71}q_{51}q_{41}}{q_{21},q_{81},q_{61}}                             \\
		\gendelta{q_{21}q_{71}q_{51}}{b}{q_{71}q_{51}q_{41}}{q_{81},q_{61}}                                    \\
		\gendelta{q_{11}q_{81}q_{51}}{b}{q_{71}q_{51}q_{41}}{q_{21},q_{61}}                                    \\
		\gendelta{q_{21}q_{81}q_{51}}{b}{q_{71}q_{51}q_{41}}{q_{61}}                                           \\
		\gendelta{q_{11}q_{81}q_{61}}{b}{q_{71}q_{51}q_{41}}{q_{21}}                                           \\
		\gendelta{q_{21}q_{81}q_{61}}{b}{q_{71}q_{51}q_{41}}{}                                                 \\
		\gendelta{q_{150}q_{11}q_{51}}{b}{q_{21}q_{51}q_{41}q_{71}}{q_{31},q_{41},q_{81},q_{71},q_{61},q_{21}} \\
		\gendelta{q_{31}q_{11}q_{51}}{b}{q_{21}q_{51}q_{41}q_{71}}{q_{41},q_{81},q_{71},q_{61},q_{21}}         \\
		\gendelta{q_{41}q_{11}q_{51}}{b}{q_{21}q_{51}q_{41}q_{71}}{q_{81},q_{71},q_{61},q_{21}}                \\
		\gendelta{q_{150}q_{21}q_{51}}{b}{q_{21}q_{51}q_{41}q_{71}}{q_{31},q_{41},q_{81},q_{71},q_{61}}        \\
		\gendelta{q_{31}q_{21}q_{51}}{b}{q_{21}q_{51}q_{41}q_{71}}{q_{41},q_{81},q_{71},q_{61}}                \\
		\gendelta{q_{41}q_{21}q_{51}}{b}{q_{21}q_{51}q_{41}q_{71}}{q_{81},q_{71},q_{61}}                       \\
		\gendelta{q_{150}q_{11}q_{61}}{b}{q_{21}q_{51}q_{41}q_{71}}{q_{31},q_{41},q_{81},q_{71},q_{21}}        \\
		\gendelta{q_{31}q_{11}q_{61}}{b}{q_{21}q_{51}q_{41}q_{71}}{q_{41},q_{81},q_{71},q_{21}}                \\
		\gendelta{q_{41}q_{11}q_{61}}{b}{q_{21}q_{51}q_{41}q_{71}}{q_{81},q_{71},q_{21}}                       \\
		\gendelta{q_{150}q_{21}q_{61}}{b}{q_{21}q_{51}q_{41}q_{71}}{q_{31},q_{41},q_{81},q_{71}}               \\
		\gendelta{q_{31}q_{21}q_{61}}{b}{q_{21}q_{51}q_{41}q_{71}}{q_{41},q_{81},q_{71}}                       \\
		\gendelta{q_{41}q_{21}q_{61}}{b}{q_{21}q_{51}q_{41}q_{71}}{q_{81},q_{71}}                              \\
	\end{array}
\end{align*}



\subsection{Определение детерменированности построенных автоматов \(M_1\), \(M_2\), \(M_3\)}
\begin{enumerate}
	\item Автомат \(M_1\) --- НКА, так как есть переходы
	      \begin{align*}
		      \delta_1(H,b) = \cbr{S_9, S_{11}} \\
		      \delta_1(S_{15},b) = \cbr{S_9, S_{11}}
	      \end{align*}
	\item Автомат \(M_2\) --- НКА, так как есть переходы
	      \begin{align*}
		      \delta_2(S_{15}, b) = \cbr{S_{10}, S_{12}} \\
		      \delta_2(S_{12}, b) = \cbr{S_{16}, F}
	      \end{align*}
	\item Автомат \(M_3 \equiv M_{17}\) --- НКА, так как есть переходы
	      \begin{align*}
		      \delta_3(q_{31}, b) = \cbr{q_{21}, q_{51}} \\
		      \delta_3(q_{41}, b) = \cbr{q_{21}, q_{51}} \\
		      \delta_3(q_{150}, b) = \cbr{q_{21}, q_{51}}
	      \end{align*}
\end{enumerate}
Все три представленных автомата являются недетерменированными.

\subsection{Построение детерменированных конечных автоматов для НКА \(M_1\), \(M_2\), \(M_3\)}

\begin{itemize}
	\item Ножество состояний \(Q'\) результирующего автомата ДКА состояит из всех подмножеств \(Q\) исходного автомата. Каждое состояние \(Q'\) обозначается как \([A_1,\dots,A_n]\), где \(A_i \in Q\). Тогда получаем число различных сочетаний
	      \begin{align*}
		      |Q'| = \sum_{k=1}^n C^k_n = 2^n - 1
	      \end{align*}
	\item Начальное состояние имеет вид (\(H\) --- начальное состояние автомата \(M\))
	      \begin{align*}
		      q'_0  \equiv [H]
	      \end{align*}
	\item Множество конечных состояний (конечные состояния исходного автомата \(F=\cbr{F_1, \dots, F_n}\)) имеет вид
	      \begin{align*}
		      F' = \cbr{[q_1, \dots, q_j,q_{j+1},\dots,q_{j+k} ]}    \\
		      \cbr{q_1, \dots, q_j} \subset \cbr{H, F_1, \dots, F_n} \\
		      \cbr{q_{j+1},\dots,q_j} \subset Q \setminus \cbr{H, F_1, \dots, F_n}
	      \end{align*}
\end{itemize}
\subsubsection{ДКА для \(M_1\)}
Переходы определяются как
\begin{align*}
	\begin{array}{ll}
		\gendelta{S_9}{a}{S_{15}}{\maqa}                               \\
		\gendelta{H}{a}{S_9}{\maqa,S_{15}}                             \\
		\gendelta{S_{15}}{a}{S_9}{\maqa}                               \\
		\gendelta{S_9S_{15}}{a}{S_9S_{15}}{\maqa,H}                    \\
		\gendelta{S_9H}{a}{S_9S_{15}}{\maqa}                           \\
		\gendelta{S_{11}}{c}{S_{16}}{\maqc}                            \\
		\gendelta{S_{15}}{c}{S_{11}}{\maqc, H, S_{16}}                 \\
		\gendelta{H}{c}{S_{11}}{\maqc, S_{16}}                         \\
		\gendelta{S_{16}}{c}{S_{11}}{\maqc}                            \\
		\gendelta{HS_{11}}{c}{S_{11}S_{16}}{\maqc,S_{15},S_{16}}       \\
		\gendelta{S_{15}S_{11}}{c}{S_{11}S_{16}}{\maqc,S_{16}}         \\
		\gendelta{S_{16}S_{11}}{c}{S_{11}S_{16}}{\maqc}                \\
		\gendelta{S_{11}}{b}{S_{16}}{}                                 \\
		\gendelta{S_{9}}{b}{S_{15}}{}                                  \\
		\gendelta{H}{b}{S_9S_{11}}{S_{15},S_{16}}                      \\
		\gendelta{S_{15}}{b}{S_9S_{11}}{S_{16}}                        \\
		\gendelta{S_{16}}{b}{S_{11}}{}                                 \\
		\gendelta{S_9H}{b}{S_9S_{11}S_{15}}{S_{15},S_{16}}             \\
		\gendelta{S_9S_{15}}{b}{S_9S_{11}S_{15}}{S_{16}}               \\
		\gendelta{S_{11}H}{b}{S_9S_{11}S_{16}}{S_{15},S_{16}}          \\
		\gendelta{S_{11}S_{15}}{b}{S_9S_{11}S_{16}}{S_{16}}            \\
		\gendelta{S_9S_{16}}{b}{S_{15}S_{11}}{}                        \\
		\gendelta{S_9S_{11}H}{b}{S_9S_{15}S_{11}S_{16}}{S_{16},S_{15}} \\
		\gendelta{S_9S_{11}S_{15}}{b}{S_9S_{15}S_{11}S_{16}}{S_{16}}   \\
		\gendelta{S_9S_{11}}{b}{S_{15}S_{16}}{}
	\end{array}
\end{align*}

\subsubsection{ДКА для \(M_2\)}
Переходы определяются как
\begin{align*}
	\begin{array}{ll}
		\gendelta{S_{10}}{a}{S_{15}S_{16}}{\mbqa, S_{16}}                     \\
		\gendelta{S_{15}}{a}{S_{10}}{\mbqa, S_{16}}                           \\
		\gendelta{S_{10}S_{15}}{a}{S_{10}S_{15}S_{16}}{\mbqa, S_{16}}         \\
		\gendelta{S_{15}}{c}{S_{12}}{\mbqc, S_{16}, F}                        \\
		\gendelta{S_{16}}{c}{S_{12}}{\mbqc, F}                                \\
		\gendelta{S_{12}}{c}{S_{16}F}{\mbqc, F}                               \\
		\gendelta{S_{12}S_{15}}{c}{S_{12}S_{16}F}{\mbqc, S_{16}, F}           \\
		\gendelta{S_{12}S_{16}}{c}{S_{12}S_{16}F}{\mbqc, F}                   \\
		\gendelta{S_{16}}{b}{S_{12}}{F}                                       \\
		\gendelta{S_{12}}{b}{FS_{16}}{F}                                      \\
		\gendelta{S_{10}}{b}{S_{15}S_{16}}{F}                                 \\
		\gendelta{S_{15}}{b}{S_{10}S_{12}}{S_{16},F}                          \\
		\gendelta{S_{15}S_{12}}{b}{S_{10}S_{12}S_{16}F}{S_{16},F}             \\
		\gendelta{S_{15}S_{10}}{b}{S_{10}S_{12}S_{15}S_{16}}{S_{16},F}        \\
		\gendelta{S_{10}S_{16}}{b}{S_{12}S_{15}S_{16}}{F}                     \\
		\gendelta{S_{12}S_{10}S_{15}}{b}{S_{10}S_{12}S_{15}S_{16}F}{S_{16},F} \\
		\gendelta{S_{12}S_{10}}{b}{S_{15}S_{16}F}{F}
	\end{array}
\end{align*}

\newpage
\subsubsection{ДКА для \(M_3\)}
\subsubsubsection{Удаление недостижимых символов}
Для удобства, сначала удалим недостижимые состояния:
\begin{align*}
	R = \cbr{q_{150}}, P_0 = \cbr{q_{150}}                                                                                                                                      \\
	P_1 = \cbr{q_{11},q_{21}, q_{51}, q_{61}}, R \setminus P_1 \neq \varnothing \Longrightarrow R = \cbr{q_{150}, q_{11},q_{21}, q_{51}, q_{61}}                                \\
	P_2 = \cbr{q_{31},q_{41}, q_{71}, q_{81}}, R \setminus P_2 \neq \varnothing \Longrightarrow R = \cbr{q_{150}, q_{11},q_{21}, q_{31}, q_{41} q_{51}, q_{61}, q_{71}, q_{81}} \\
	P_3 = \cbr{q_{51}, q_{61}}, R \setminus P_3 = \varnothing \Longrightarrow R = \cbr{q_{150}, q_{11},q_{21}, q_{31}, q_{41} q_{51}, q_{61}, q_{71}, q_{81}}                   \\
\end{align*}
Автомат после удаления недостижимых состояний
\begin{align*}
	M_3 = \rbr{\cbr{q_{150}, q_{11},q_{21}, q_{31}, q_{41} q_{51}, q_{61}, q_{71}, q_{81}}, \Sigma, \delta_3, q_{150}, \cbr{q_{71}, q_{81}}} \\
	\begin{array}{lll}
		\delta_{3}(q_{11}, a) =  \cbr{q_{31}} & \delta_{3}(q_{11}, b)=  \cbr{q_{41}}                                                  \\
		\delta_{3}(q_{21}, a) =  \cbr{q_{31}} & \delta_{3}(q_{21}, b)=  \cbr{q_{41}}                                                  \\
		\delta_{3}(q_{31}, a) = \cbr{q_{11}}  & \delta_{3}(q_{31}, b) = \cbr{q_{21}, q_{51}}  & \delta_{3}(q_{31}, c) = \cbr{q_{61}}  \\
		\delta_{3}(q_{41}, a) = \cbr{q_{11}}  & \delta_{3}(q_{41}, b) = \cbr{q_{21}, q_{51}}  & \delta_{3}(q_{41}, c) = \cbr{q_{61}}  \\
		\delta_{3}(q_{150}, a) = \cbr{q_{11}} & \delta_{3}(q_{150}, b) = \cbr{q_{21}, q_{51}} & \delta_{3}(q_{150}, c) = \cbr{q_{61}} \\
		\delta_{3}(q_{51}, b)  = \cbr{q_{71}} & \delta_{3}(q_{51}, c)  = \cbr{q_{81}}                                                 \\
		\delta_{3}(q_{61}, b)  = \cbr{q_{71}} & \delta_{3}(q_{61}, c)  = \cbr{q_{81}}                                                 \\
		\delta_{3}(q_{71}, b) = \cbr{q_{51}}  & \delta_{3}(q_{71}, c) =\cbr{q_{61}}                                                   \\
		\delta_{3}(q_{81}, b) = \cbr{q_{51}}  & \delta_{3}(q_{81}, c) =\cbr{q_{61}}                                                   \\
	\end{array}
\end{align*}

\begin{figure}[h!]
	\centering
	\begin{tikzpicture}[
			->,
			>=stealth',
			node distance=2.0cm,
			every state/.style={thick, fill=gray!10},
			initial text={Начало}
		]

		% Состояния
		\node[state, initial] (q150) {$q_{150}$};
		\node[state,  right of=q10] (q11) {$q_{11}$};
		\node[state,  right of=q20] (q21) {$q_{21}$};
		\node[state, right= of q11] (q31) {$q_{31}$};
		\node[state, right= of q21] (q41) {$q_{41}$};
		% Переходы
		\path
		(q150) edge node[above] {a} (q11)
		(q150) edge node[below] {b} (q21)
		(q11) edge node[below] {a} (q31)
		(q21) edge[bend left=20] node[above] {a} (q31)
		(q11) edge[bend right=20] node[below] {b} (q41)
		(q21) edge node[above] {b} (q41)
		(q31) edge[bend right=20] node[above]{a} (q11)
		(q31) edge[bend left=20] node[below]{b} (q21)
		(q41) edge[bend left=20] node[below]{b} (q21)
		(q41) edge[bend right=20] node[above]{a} (q11)
		;
		\node[state, above of=q110] (q51) {$q_{51}$};
		\node[state, below of=q110] (q61) {$q_{61}$};
		\node[state, accepting, right = of q51] (q71) {$q_{71}$};
		\node[state, accepting, right = of q61] (q81) {$q_{81}$};
		% Переходы
		\path
		(q51) edge node[below] {b} (q71)
		(q61) edge[bend left=20] node[above] {b} (q71)
		(q51) edge[bend right=20] node[below] {c} (q81)
		(q61) edge node[above] {c} (q81)
		(q71) edge[bend right=20] node[above] {b} (q51)
		(q81) edge[bend left=20] node[below] {c} (q61)
		(q71) edge[bend left=20] node[below]{c} (q61)
		(q81) edge[bend right=20] node[above]{b} (q51)
		;

		\path
		(q41) edge[bend right=30] node[below]{c} (q61)
		(q31) edge[bend left=30] node[above]{b} (q51)
		(q41) edge[bend left=20] node[above]{b} (q51)
		(q31) edge[bend right=20] node[below]{c} (q61)
		(q150) edge[bend right=60] node[below]{c} (q61)
		(q150) edge[bend left=60] node[above]{b} (q51)
		;

	\end{tikzpicture}
	\caption{Диаграмма состояний НКА \(M_{17}\)}
\end{figure}

\newpage
\subsubsubsection{Построение ДКА}
Переходы определяются как
\begin{align*}
	\begin{array}{ll}
		\gendelta{q_{150}}{a}{q_{11}}{\mcqa, q_{41}, q_{31}}                                \\
		\gendelta{q_{41}}{a}{q_{11}}{\mcqa, q_{31}}                                         \\
		\gendelta{q_{31}}{a}{q_{11}}{\mcqa}                                                 \\
		\gendelta{q_{11}}{a}{q_{31}}{\mcqa, q_{21}}                                         \\
		\gendelta{q_{21}}{a}{q_{31}}{\mcqa}                                                 \\
		\gendelta{q_{11}q_{150}}{a}{q_{11}q_{31}}{\mcqa,q_{31}, q_{41},q_{21}}              \\
		\gendelta{q_{11}q_{31}}{a}{q_{11}q_{31}}{\mcqa, q_{41},q_{21}}                      \\
		\gendelta{q_{11}q_{41}}{a}{q_{11}q_{31}}{\mcqa, q_{21}}                             \\
		\gendelta{q_{21}q_{150}}{a}{q_{11}q_{31}}{\mcqa,q_{31}, q_{41}}                     \\
		\gendelta{q_{21}q_{31}}{a}{q_{11}q_{31}}{\mcqa, q_{41}}                             \\
		\gendelta{q_{21}q_{41}}{a}{q_{11}q_{31}}{\mcqa}                                     \\
		\gendelta{q_{150}}{c}{q_{61}}{\mcqc,q_{31},q_{41},q_{71},q_{81}}                    \\
		\gendelta{q_{31}}{c}{q_{61}}{\mcqc,q_{41},q_{71},q_{81}}                            \\
		\gendelta{q_{41}}{c}{q_{61}}{\mcqc,q_{71},q_{81}}                                   \\
		\gendelta{q_{71}}{c}{q_{61}}{\mcqc,q_{81}}                                          \\
		\gendelta{q_{81}}{c}{q_{61}}{\mcqc}                                                 \\
		\gendelta{q_{51}}{c}{q_{81}}{\mcqc, q_{61}}                                         \\
		\gendelta{q_{61}}{c}{q_{81}}{\mcqc}                                                 \\
		\gendelta{q_{51}q_{150}}{c}{q_{61}q_{81}}{\mcqc,q_{31},q_{41},q_{71},q_{81},q_{61}} \\
		\gendelta{q_{51}q_{31}}{c}{q_{61}q_{81}}{\mcqc,q_{41},q_{71},q_{81},q_{61}}         \\
		\gendelta{q_{51}q_{41}}{c}{q_{61}q_{81}}{\mcqc,q_{71},q_{81},q_{61}}                \\
		\gendelta{q_{51}q_{71}}{c}{q_{61}q_{81}}{\mcqc,q_{81},q_{61}}                       \\
		\gendelta{q_{51}q_{81}}{c}{q_{61}q_{81}}{\mcqc,q_{61}}                              \\
		\gendelta{q_{61}q_{150}}{c}{q_{61}q_{81}}{\mcqc,q_{31},q_{41},q_{71},q_{81}}        \\
		\gendelta{q_{61}q_{31}}{c}{q_{61}q_{81}}{\mcqc,q_{41},q_{71},q_{81}}                \\
		\gendelta{q_{61}q_{41}}{c}{q_{61}q_{81}}{\mcqc,q_{71},q_{81}}                       \\
		\gendelta{q_{61}q_{71}}{c}{q_{61}q_{81}}{\mcqc,q_{81}}                              \\
		\gendelta{q_{61}q_{81}}{c}{q_{61}q_{81}}{\mcqc}                                     \\
		\gendelta{q_{11}}{b}{q_{41}}{q_{21}}                                                \\
		\gendelta{q_{21}}{b}{q_{41}}{}                                                      \\
		\gendelta{q_{51}}{b}{q_{71}}{q_{61}}                                                \\
		\gendelta{q_{61}}{b}{q_{71}}{}                                                      \\
		\gendelta{q_{150}}{b}{q_{21}q_{51}}{q_{31},q_{41},q_{81}, q_{71}}                   \\
		\gendelta{q_{31}}{b}{q_{21}q_{51}}{q_{41},q_{81}, q_{71}}                           \\
		\gendelta{q_{41}}{b}{q_{21}q_{51}}{q_{81}, q_{71}}                                  \\
		\gendelta{q_{81}}{b}{q_{51}}{q_{71}}                                                \\
		\gendelta{q_{71}}{b}{q_{51}}{}                                                      \\
	\end{array}
\end{align*}

\begin{align*}
	\begin{array}{ll}
		\gendelta{q_{71}q_{51}}{b}{q_{51}q_{71}}{q_{81},q_{61}}                                                \\
		\gendelta{q_{81}q_{51}}{b}{q_{51}q_{71}}{q_{61}}                                                       \\
		\gendelta{q_{71}q_{61}}{b}{q_{51}q_{71}}{q_{81}}                                                       \\
		\gendelta{q_{81}q_{61}}{b}{q_{51}q_{71}}{}                                                             \\
		\gendelta{q_{51}q_{11}}{b}{q_{41}q_{71}}{q_{61},q_{21}}                                                \\
		\gendelta{q_{61}q_{11}}{b}{q_{41}q_{71}}{q_{21}}                                                       \\
		\gendelta{q_{51}q_{21}}{b}{q_{41}q_{71}}{q_{61}}                                                       \\
		\gendelta{q_{61}q_{21}}{b}{q_{41}q_{71}}{}                                                             \\
		\gendelta{q_{71}q_{11}}{b}{q_{51}q_{41}}{ q_{81}, q_{21}}                                              \\
		\gendelta{q_{81}q_{11}}{b}{q_{51}q_{41}}{ q_{21}}                                                      \\
		\gendelta{q_{71}q_{21}}{b}{q_{51}q_{41}}{ q_{81}}                                                      \\
		\gendelta{q_{81}q_{21}}{b}{q_{51}q_{41}}{}                                                             \\
		\gendelta{q_{150}q_{11}}{b}{q_{21}q_{51}q_{41}}{q_{31},q_{41},q_{81},q_{71},q_{21}}                    \\
		\gendelta{q_{31}q_{11}}{b}{q_{21}q_{51}q_{41}}{q_{41},q_{81},q_{71},q_{21}}                            \\
		\gendelta{q_{41}q_{11}}{b}{q_{21}q_{51}q_{41}}{q_{81},q_{71},q_{21}}                                   \\
		\gendelta{q_{150}q_{21}}{b}{q_{21}q_{51}q_{41}}{q_{31},q_{41},q_{81},q_{71}}                           \\
		\gendelta{q_{31}q_{21}}{b}{q_{21}q_{51}q_{41}}{q_{41},q_{81},q_{71}}                                   \\
		\gendelta{q_{41}q_{21}}{b}{q_{21}q_{51}q_{41}}{q_{81},q_{71}}                                          \\
		\gendelta{q_{150}q_{51}}{b}{q_{21}q_{51}q_{71}}{q_{31},q_{41},q_{81},q_{71},q_{61}}                    \\
		\gendelta{q_{31}q_{51}}{b}{q_{21}q_{51}q_{71}}{q_{41},q_{81},q_{71},q_{61}}                            \\
		\gendelta{q_{41}q_{51}}{b}{q_{21}q_{51}q_{71}}{q_{81},q_{71},q_{61}}                                   \\
		\gendelta{q_{150}q_{61}}{b}{q_{21}q_{51}q_{71}}{q_{31},q_{41},q_{81},q_{71}}                           \\
		\gendelta{q_{31}q_{61}}{b}{q_{21}q_{51}q_{71}}{q_{41},q_{81},q_{71}}                                   \\
		\gendelta{q_{41}q_{61}}{b}{q_{21}q_{51}q_{71}}{q_{81},q_{71}}                                          \\
		\gendelta{q_{11}q_{71}q_{51}}{b}{q_{71}q_{51}q_{41}}{q_{21},q_{81},q_{61}}                             \\
		\gendelta{q_{21}q_{71}q_{51}}{b}{q_{71}q_{51}q_{41}}{q_{81},q_{61}}                                    \\
		\gendelta{q_{11}q_{81}q_{51}}{b}{q_{71}q_{51}q_{41}}{q_{21},q_{61}}                                    \\
		\gendelta{q_{21}q_{81}q_{51}}{b}{q_{71}q_{51}q_{41}}{q_{61}}                                           \\
		\gendelta{q_{11}q_{81}q_{61}}{b}{q_{71}q_{51}q_{41}}{q_{21}}                                           \\
		\gendelta{q_{21}q_{81}q_{61}}{b}{q_{71}q_{51}q_{41}}{}                                                 \\
		\gendelta{q_{150}q_{11}q_{51}}{b}{q_{21}q_{51}q_{41}q_{71}}{q_{31},q_{41},q_{81},q_{71},q_{61},q_{21}} \\
		\gendelta{q_{31}q_{11}q_{51}}{b}{q_{21}q_{51}q_{41}q_{71}}{q_{41},q_{81},q_{71},q_{61},q_{21}}         \\
		\gendelta{q_{41}q_{11}q_{51}}{b}{q_{21}q_{51}q_{41}q_{71}}{q_{81},q_{71},q_{61},q_{21}}                \\
		\gendelta{q_{150}q_{21}q_{51}}{b}{q_{21}q_{51}q_{41}q_{71}}{q_{31},q_{41},q_{81},q_{71},q_{61}}        \\
		\gendelta{q_{31}q_{21}q_{51}}{b}{q_{21}q_{51}q_{41}q_{71}}{q_{41},q_{81},q_{71},q_{61}}                \\
		\gendelta{q_{41}q_{21}q_{51}}{b}{q_{21}q_{51}q_{41}q_{71}}{q_{81},q_{71},q_{61}}                       \\
		\gendelta{q_{150}q_{11}q_{61}}{b}{q_{21}q_{51}q_{41}q_{71}}{q_{31},q_{41},q_{81},q_{71},q_{21}}        \\
		\gendelta{q_{31}q_{11}q_{61}}{b}{q_{21}q_{51}q_{41}q_{71}}{q_{41},q_{81},q_{71},q_{21}}                \\
		\gendelta{q_{41}q_{11}q_{61}}{b}{q_{21}q_{51}q_{41}q_{71}}{q_{81},q_{71},q_{21}}                       \\
		\gendelta{q_{150}q_{21}q_{61}}{b}{q_{21}q_{51}q_{41}q_{71}}{q_{31},q_{41},q_{81},q_{71}}               \\
		\gendelta{q_{31}q_{21}q_{61}}{b}{q_{21}q_{51}q_{41}q_{71}}{q_{41},q_{81},q_{71}}                       \\
		\gendelta{q_{41}q_{21}q_{61}}{b}{q_{21}q_{51}q_{41}q_{71}}{q_{81},q_{71}}                              \\
	\end{array}
\end{align*}



\subsection{Определение детерменированности построенных автоматов \(M_1\), \(M_2\), \(M_3\)}
\begin{enumerate}
	\item Автомат \(M_1\) --- НКА, так как есть переходы
	      \begin{align*}
		      \delta_1(H,b) = \cbr{S_9, S_{11}} \\
		      \delta_1(S_{15},b) = \cbr{S_9, S_{11}}
	      \end{align*}
	\item Автомат \(M_2\) --- НКА, так как есть переходы
	      \begin{align*}
		      \delta_2(S_{15}, b) = \cbr{S_{10}, S_{12}} \\
		      \delta_2(S_{12}, b) = \cbr{S_{16}, F}
	      \end{align*}
	\item Автомат \(M_3 \equiv M_{17}\) --- НКА, так как есть переходы
	      \begin{align*}
		      \delta_3(q_{31}, b) = \cbr{q_{21}, q_{51}} \\
		      \delta_3(q_{41}, b) = \cbr{q_{21}, q_{51}} \\
		      \delta_3(q_{150}, b) = \cbr{q_{21}, q_{51}}
	      \end{align*}
\end{enumerate}
Все три представленных автомата являются недетерменированными.

\subsection{Построение детерменированных конечных автоматов для НКА \(M_1\), \(M_2\), \(M_3\)}

\begin{itemize}
	\item Ножество состояний \(Q'\) результирующего автомата ДКА состояит из всех подмножеств \(Q\) исходного автомата. Каждое состояние \(Q'\) обозначается как \([A_1,\dots,A_n]\), где \(A_i \in Q\). Тогда получаем число различных сочетаний
	      \begin{align*}
		      |Q'| = \sum_{k=1}^n C^k_n = 2^n - 1
	      \end{align*}
	\item Начальное состояние имеет вид (\(H\) --- начальное состояние автомата \(M\))
	      \begin{align*}
		      q'_0  \equiv [H]
	      \end{align*}
	\item Множество конечных состояний (конечные состояния исходного автомата \(F=\cbr{F_1, \dots, F_n}\)) имеет вид
	      \begin{align*}
		      F' = \cbr{[q_1, \dots, q_j,q_{j+1},\dots,q_{j+k} ]}    \\
		      \cbr{q_1, \dots, q_j} \subset \cbr{H, F_1, \dots, F_n} \\
		      \cbr{q_{j+1},\dots,q_j} \subset Q \setminus \cbr{H, F_1, \dots, F_n}
	      \end{align*}
\end{itemize}
\subsubsection{ДКА для \(M_1\)}
Переходы определяются как
\begin{align*}
	\begin{array}{ll}
		\gendelta{S_9}{a}{S_{15}}{\maqa}                               \\
		\gendelta{H}{a}{S_9}{\maqa,S_{15}}                             \\
		\gendelta{S_{15}}{a}{S_9}{\maqa}                               \\
		\gendelta{S_9S_{15}}{a}{S_9S_{15}}{\maqa,H}                    \\
		\gendelta{S_9H}{a}{S_9S_{15}}{\maqa}                           \\
		\gendelta{S_{11}}{c}{S_{16}}{\maqc}                            \\
		\gendelta{S_{15}}{c}{S_{11}}{\maqc, H, S_{16}}                 \\
		\gendelta{H}{c}{S_{11}}{\maqc, S_{16}}                         \\
		\gendelta{S_{16}}{c}{S_{11}}{\maqc}                            \\
		\gendelta{HS_{11}}{c}{S_{11}S_{16}}{\maqc,S_{15},S_{16}}       \\
		\gendelta{S_{15}S_{11}}{c}{S_{11}S_{16}}{\maqc,S_{16}}         \\
		\gendelta{S_{16}S_{11}}{c}{S_{11}S_{16}}{\maqc}                \\
		\gendelta{S_{11}}{b}{S_{16}}{}                                 \\
		\gendelta{S_{9}}{b}{S_{15}}{}                                  \\
		\gendelta{H}{b}{S_9S_{11}}{S_{15},S_{16}}                      \\
		\gendelta{S_{15}}{b}{S_9S_{11}}{S_{16}}                        \\
		\gendelta{S_{16}}{b}{S_{11}}{}                                 \\
		\gendelta{S_9H}{b}{S_9S_{11}S_{15}}{S_{15},S_{16}}             \\
		\gendelta{S_9S_{15}}{b}{S_9S_{11}S_{15}}{S_{16}}               \\
		\gendelta{S_{11}H}{b}{S_9S_{11}S_{16}}{S_{15},S_{16}}          \\
		\gendelta{S_{11}S_{15}}{b}{S_9S_{11}S_{16}}{S_{16}}            \\
		\gendelta{S_9S_{16}}{b}{S_{15}S_{11}}{}                        \\
		\gendelta{S_9S_{11}H}{b}{S_9S_{15}S_{11}S_{16}}{S_{16},S_{15}} \\
		\gendelta{S_9S_{11}S_{15}}{b}{S_9S_{15}S_{11}S_{16}}{S_{16}}   \\
		\gendelta{S_9S_{11}}{b}{S_{15}S_{16}}{}
	\end{array}
\end{align*}

\subsubsection{ДКА для \(M_2\)}
Переходы определяются как
\begin{align*}
	\begin{array}{ll}
		\gendelta{S_{10}}{a}{S_{15}S_{16}}{\mbqa, S_{16}}                     \\
		\gendelta{S_{15}}{a}{S_{10}}{\mbqa, S_{16}}                           \\
		\gendelta{S_{10}S_{15}}{a}{S_{10}S_{15}S_{16}}{\mbqa, S_{16}}         \\
		\gendelta{S_{15}}{c}{S_{12}}{\mbqc, S_{16}, F}                        \\
		\gendelta{S_{16}}{c}{S_{12}}{\mbqc, F}                                \\
		\gendelta{S_{12}}{c}{S_{16}F}{\mbqc, F}                               \\
		\gendelta{S_{12}S_{15}}{c}{S_{12}S_{16}F}{\mbqc, S_{16}, F}           \\
		\gendelta{S_{12}S_{16}}{c}{S_{12}S_{16}F}{\mbqc, F}                   \\
		\gendelta{S_{16}}{b}{S_{12}}{F}                                       \\
		\gendelta{S_{12}}{b}{FS_{16}}{F}                                      \\
		\gendelta{S_{10}}{b}{S_{15}S_{16}}{F}                                 \\
		\gendelta{S_{15}}{b}{S_{10}S_{12}}{S_{16},F}                          \\
		\gendelta{S_{15}S_{12}}{b}{S_{10}S_{12}S_{16}F}{S_{16},F}             \\
		\gendelta{S_{15}S_{10}}{b}{S_{10}S_{12}S_{15}S_{16}}{S_{16},F}        \\
		\gendelta{S_{10}S_{16}}{b}{S_{12}S_{15}S_{16}}{F}                     \\
		\gendelta{S_{12}S_{10}S_{15}}{b}{S_{10}S_{12}S_{15}S_{16}F}{S_{16},F} \\
		\gendelta{S_{12}S_{10}}{b}{S_{15}S_{16}F}{F}
	\end{array}
\end{align*}

\newpage
\subsubsection{ДКА для \(M_3\)}
\subsubsubsection{Удаление недостижимых символов}
Для удобства, сначала удалим недостижимые состояния:
\begin{align*}
	R = \cbr{q_{150}}, P_0 = \cbr{q_{150}}                                                                                                                                      \\
	P_1 = \cbr{q_{11},q_{21}, q_{51}, q_{61}}, R \setminus P_1 \neq \varnothing \Longrightarrow R = \cbr{q_{150}, q_{11},q_{21}, q_{51}, q_{61}}                                \\
	P_2 = \cbr{q_{31},q_{41}, q_{71}, q_{81}}, R \setminus P_2 \neq \varnothing \Longrightarrow R = \cbr{q_{150}, q_{11},q_{21}, q_{31}, q_{41} q_{51}, q_{61}, q_{71}, q_{81}} \\
	P_3 = \cbr{q_{51}, q_{61}}, R \setminus P_3 = \varnothing \Longrightarrow R = \cbr{q_{150}, q_{11},q_{21}, q_{31}, q_{41} q_{51}, q_{61}, q_{71}, q_{81}}                   \\
\end{align*}
Автомат после удаления недостижимых состояний
\begin{align*}
	M_3 = \rbr{\cbr{q_{150}, q_{11},q_{21}, q_{31}, q_{41} q_{51}, q_{61}, q_{71}, q_{81}}, \Sigma, \delta_3, q_{150}, \cbr{q_{71}, q_{81}}} \\
	\begin{array}{lll}
		\delta_{3}(q_{11}, a) =  \cbr{q_{31}} & \delta_{3}(q_{11}, b)=  \cbr{q_{41}}                                                  \\
		\delta_{3}(q_{21}, a) =  \cbr{q_{31}} & \delta_{3}(q_{21}, b)=  \cbr{q_{41}}                                                  \\
		\delta_{3}(q_{31}, a) = \cbr{q_{11}}  & \delta_{3}(q_{31}, b) = \cbr{q_{21}, q_{51}}  & \delta_{3}(q_{31}, c) = \cbr{q_{61}}  \\
		\delta_{3}(q_{41}, a) = \cbr{q_{11}}  & \delta_{3}(q_{41}, b) = \cbr{q_{21}, q_{51}}  & \delta_{3}(q_{41}, c) = \cbr{q_{61}}  \\
		\delta_{3}(q_{150}, a) = \cbr{q_{11}} & \delta_{3}(q_{150}, b) = \cbr{q_{21}, q_{51}} & \delta_{3}(q_{150}, c) = \cbr{q_{61}} \\
		\delta_{3}(q_{51}, b)  = \cbr{q_{71}} & \delta_{3}(q_{51}, c)  = \cbr{q_{81}}                                                 \\
		\delta_{3}(q_{61}, b)  = \cbr{q_{71}} & \delta_{3}(q_{61}, c)  = \cbr{q_{81}}                                                 \\
		\delta_{3}(q_{71}, b) = \cbr{q_{51}}  & \delta_{3}(q_{71}, c) =\cbr{q_{61}}                                                   \\
		\delta_{3}(q_{81}, b) = \cbr{q_{51}}  & \delta_{3}(q_{81}, c) =\cbr{q_{61}}                                                   \\
	\end{array}
\end{align*}

\begin{figure}[h!]
	\centering
	\begin{tikzpicture}[
			->,
			>=stealth',
			node distance=2.0cm,
			every state/.style={thick, fill=gray!10},
			initial text={Начало}
		]

		% Состояния
		\node[state, initial] (q150) {$q_{150}$};
		\node[state,  right of=q10] (q11) {$q_{11}$};
		\node[state,  right of=q20] (q21) {$q_{21}$};
		\node[state, right= of q11] (q31) {$q_{31}$};
		\node[state, right= of q21] (q41) {$q_{41}$};
		% Переходы
		\path
		(q150) edge node[above] {a} (q11)
		(q150) edge node[below] {b} (q21)
		(q11) edge node[below] {a} (q31)
		(q21) edge[bend left=20] node[above] {a} (q31)
		(q11) edge[bend right=20] node[below] {b} (q41)
		(q21) edge node[above] {b} (q41)
		(q31) edge[bend right=20] node[above]{a} (q11)
		(q31) edge[bend left=20] node[below]{b} (q21)
		(q41) edge[bend left=20] node[below]{b} (q21)
		(q41) edge[bend right=20] node[above]{a} (q11)
		;
		\node[state, above of=q110] (q51) {$q_{51}$};
		\node[state, below of=q110] (q61) {$q_{61}$};
		\node[state, accepting, right = of q51] (q71) {$q_{71}$};
		\node[state, accepting, right = of q61] (q81) {$q_{81}$};
		% Переходы
		\path
		(q51) edge node[below] {b} (q71)
		(q61) edge[bend left=20] node[above] {b} (q71)
		(q51) edge[bend right=20] node[below] {c} (q81)
		(q61) edge node[above] {c} (q81)
		(q71) edge[bend right=20] node[above] {b} (q51)
		(q81) edge[bend left=20] node[below] {c} (q61)
		(q71) edge[bend left=20] node[below]{c} (q61)
		(q81) edge[bend right=20] node[above]{b} (q51)
		;

		\path
		(q41) edge[bend right=30] node[below]{c} (q61)
		(q31) edge[bend left=30] node[above]{b} (q51)
		(q41) edge[bend left=20] node[above]{b} (q51)
		(q31) edge[bend right=20] node[below]{c} (q61)
		(q150) edge[bend right=60] node[below]{c} (q61)
		(q150) edge[bend left=60] node[above]{b} (q51)
		;

	\end{tikzpicture}
	\caption{Диаграмма состояний НКА \(M_{17}\)}
\end{figure}

\newpage
\subsubsubsection{Построение ДКА}
Переходы определяются как
\begin{align*}
	\begin{array}{ll}
		\gendelta{q_{150}}{a}{q_{11}}{\mcqa, q_{41}, q_{31}}                                \\
		\gendelta{q_{41}}{a}{q_{11}}{\mcqa, q_{31}}                                         \\
		\gendelta{q_{31}}{a}{q_{11}}{\mcqa}                                                 \\
		\gendelta{q_{11}}{a}{q_{31}}{\mcqa, q_{21}}                                         \\
		\gendelta{q_{21}}{a}{q_{31}}{\mcqa}                                                 \\
		\gendelta{q_{11}q_{150}}{a}{q_{11}q_{31}}{\mcqa,q_{31}, q_{41},q_{21}}              \\
		\gendelta{q_{11}q_{31}}{a}{q_{11}q_{31}}{\mcqa, q_{41},q_{21}}                      \\
		\gendelta{q_{11}q_{41}}{a}{q_{11}q_{31}}{\mcqa, q_{21}}                             \\
		\gendelta{q_{21}q_{150}}{a}{q_{11}q_{31}}{\mcqa,q_{31}, q_{41}}                     \\
		\gendelta{q_{21}q_{31}}{a}{q_{11}q_{31}}{\mcqa, q_{41}}                             \\
		\gendelta{q_{21}q_{41}}{a}{q_{11}q_{31}}{\mcqa}                                     \\
		\gendelta{q_{150}}{c}{q_{61}}{\mcqc,q_{31},q_{41},q_{71},q_{81}}                    \\
		\gendelta{q_{31}}{c}{q_{61}}{\mcqc,q_{41},q_{71},q_{81}}                            \\
		\gendelta{q_{41}}{c}{q_{61}}{\mcqc,q_{71},q_{81}}                                   \\
		\gendelta{q_{71}}{c}{q_{61}}{\mcqc,q_{81}}                                          \\
		\gendelta{q_{81}}{c}{q_{61}}{\mcqc}                                                 \\
		\gendelta{q_{51}}{c}{q_{81}}{\mcqc, q_{61}}                                         \\
		\gendelta{q_{61}}{c}{q_{81}}{\mcqc}                                                 \\
		\gendelta{q_{51}q_{150}}{c}{q_{61}q_{81}}{\mcqc,q_{31},q_{41},q_{71},q_{81},q_{61}} \\
		\gendelta{q_{51}q_{31}}{c}{q_{61}q_{81}}{\mcqc,q_{41},q_{71},q_{81},q_{61}}         \\
		\gendelta{q_{51}q_{41}}{c}{q_{61}q_{81}}{\mcqc,q_{71},q_{81},q_{61}}                \\
		\gendelta{q_{51}q_{71}}{c}{q_{61}q_{81}}{\mcqc,q_{81},q_{61}}                       \\
		\gendelta{q_{51}q_{81}}{c}{q_{61}q_{81}}{\mcqc,q_{61}}                              \\
		\gendelta{q_{61}q_{150}}{c}{q_{61}q_{81}}{\mcqc,q_{31},q_{41},q_{71},q_{81}}        \\
		\gendelta{q_{61}q_{31}}{c}{q_{61}q_{81}}{\mcqc,q_{41},q_{71},q_{81}}                \\
		\gendelta{q_{61}q_{41}}{c}{q_{61}q_{81}}{\mcqc,q_{71},q_{81}}                       \\
		\gendelta{q_{61}q_{71}}{c}{q_{61}q_{81}}{\mcqc,q_{81}}                              \\
		\gendelta{q_{61}q_{81}}{c}{q_{61}q_{81}}{\mcqc}                                     \\
		\gendelta{q_{11}}{b}{q_{41}}{q_{21}}                                                \\
		\gendelta{q_{21}}{b}{q_{41}}{}                                                      \\
		\gendelta{q_{51}}{b}{q_{71}}{q_{61}}                                                \\
		\gendelta{q_{61}}{b}{q_{71}}{}                                                      \\
		\gendelta{q_{150}}{b}{q_{21}q_{51}}{q_{31},q_{41},q_{81}, q_{71}}                   \\
		\gendelta{q_{31}}{b}{q_{21}q_{51}}{q_{41},q_{81}, q_{71}}                           \\
		\gendelta{q_{41}}{b}{q_{21}q_{51}}{q_{81}, q_{71}}                                  \\
		\gendelta{q_{81}}{b}{q_{51}}{q_{71}}                                                \\
		\gendelta{q_{71}}{b}{q_{51}}{}                                                      \\
	\end{array}
\end{align*}

\begin{align*}
	\begin{array}{ll}
		\gendelta{q_{71}q_{51}}{b}{q_{51}q_{71}}{q_{81},q_{61}}                                                \\
		\gendelta{q_{81}q_{51}}{b}{q_{51}q_{71}}{q_{61}}                                                       \\
		\gendelta{q_{71}q_{61}}{b}{q_{51}q_{71}}{q_{81}}                                                       \\
		\gendelta{q_{81}q_{61}}{b}{q_{51}q_{71}}{}                                                             \\
		\gendelta{q_{51}q_{11}}{b}{q_{41}q_{71}}{q_{61},q_{21}}                                                \\
		\gendelta{q_{61}q_{11}}{b}{q_{41}q_{71}}{q_{21}}                                                       \\
		\gendelta{q_{51}q_{21}}{b}{q_{41}q_{71}}{q_{61}}                                                       \\
		\gendelta{q_{61}q_{21}}{b}{q_{41}q_{71}}{}                                                             \\
		\gendelta{q_{71}q_{11}}{b}{q_{51}q_{41}}{ q_{81}, q_{21}}                                              \\
		\gendelta{q_{81}q_{11}}{b}{q_{51}q_{41}}{ q_{21}}                                                      \\
		\gendelta{q_{71}q_{21}}{b}{q_{51}q_{41}}{ q_{81}}                                                      \\
		\gendelta{q_{81}q_{21}}{b}{q_{51}q_{41}}{}                                                             \\
		\gendelta{q_{150}q_{11}}{b}{q_{21}q_{51}q_{41}}{q_{31},q_{41},q_{81},q_{71},q_{21}}                    \\
		\gendelta{q_{31}q_{11}}{b}{q_{21}q_{51}q_{41}}{q_{41},q_{81},q_{71},q_{21}}                            \\
		\gendelta{q_{41}q_{11}}{b}{q_{21}q_{51}q_{41}}{q_{81},q_{71},q_{21}}                                   \\
		\gendelta{q_{150}q_{21}}{b}{q_{21}q_{51}q_{41}}{q_{31},q_{41},q_{81},q_{71}}                           \\
		\gendelta{q_{31}q_{21}}{b}{q_{21}q_{51}q_{41}}{q_{41},q_{81},q_{71}}                                   \\
		\gendelta{q_{41}q_{21}}{b}{q_{21}q_{51}q_{41}}{q_{81},q_{71}}                                          \\
		\gendelta{q_{150}q_{51}}{b}{q_{21}q_{51}q_{71}}{q_{31},q_{41},q_{81},q_{71},q_{61}}                    \\
		\gendelta{q_{31}q_{51}}{b}{q_{21}q_{51}q_{71}}{q_{41},q_{81},q_{71},q_{61}}                            \\
		\gendelta{q_{41}q_{51}}{b}{q_{21}q_{51}q_{71}}{q_{81},q_{71},q_{61}}                                   \\
		\gendelta{q_{150}q_{61}}{b}{q_{21}q_{51}q_{71}}{q_{31},q_{41},q_{81},q_{71}}                           \\
		\gendelta{q_{31}q_{61}}{b}{q_{21}q_{51}q_{71}}{q_{41},q_{81},q_{71}}                                   \\
		\gendelta{q_{41}q_{61}}{b}{q_{21}q_{51}q_{71}}{q_{81},q_{71}}                                          \\
		\gendelta{q_{11}q_{71}q_{51}}{b}{q_{71}q_{51}q_{41}}{q_{21},q_{81},q_{61}}                             \\
		\gendelta{q_{21}q_{71}q_{51}}{b}{q_{71}q_{51}q_{41}}{q_{81},q_{61}}                                    \\
		\gendelta{q_{11}q_{81}q_{51}}{b}{q_{71}q_{51}q_{41}}{q_{21},q_{61}}                                    \\
		\gendelta{q_{21}q_{81}q_{51}}{b}{q_{71}q_{51}q_{41}}{q_{61}}                                           \\
		\gendelta{q_{11}q_{81}q_{61}}{b}{q_{71}q_{51}q_{41}}{q_{21}}                                           \\
		\gendelta{q_{21}q_{81}q_{61}}{b}{q_{71}q_{51}q_{41}}{}                                                 \\
		\gendelta{q_{150}q_{11}q_{51}}{b}{q_{21}q_{51}q_{41}q_{71}}{q_{31},q_{41},q_{81},q_{71},q_{61},q_{21}} \\
		\gendelta{q_{31}q_{11}q_{51}}{b}{q_{21}q_{51}q_{41}q_{71}}{q_{41},q_{81},q_{71},q_{61},q_{21}}         \\
		\gendelta{q_{41}q_{11}q_{51}}{b}{q_{21}q_{51}q_{41}q_{71}}{q_{81},q_{71},q_{61},q_{21}}                \\
		\gendelta{q_{150}q_{21}q_{51}}{b}{q_{21}q_{51}q_{41}q_{71}}{q_{31},q_{41},q_{81},q_{71},q_{61}}        \\
		\gendelta{q_{31}q_{21}q_{51}}{b}{q_{21}q_{51}q_{41}q_{71}}{q_{41},q_{81},q_{71},q_{61}}                \\
		\gendelta{q_{41}q_{21}q_{51}}{b}{q_{21}q_{51}q_{41}q_{71}}{q_{81},q_{71},q_{61}}                       \\
		\gendelta{q_{150}q_{11}q_{61}}{b}{q_{21}q_{51}q_{41}q_{71}}{q_{31},q_{41},q_{81},q_{71},q_{21}}        \\
		\gendelta{q_{31}q_{11}q_{61}}{b}{q_{21}q_{51}q_{41}q_{71}}{q_{41},q_{81},q_{71},q_{21}}                \\
		\gendelta{q_{41}q_{11}q_{61}}{b}{q_{21}q_{51}q_{41}q_{71}}{q_{81},q_{71},q_{21}}                       \\
		\gendelta{q_{150}q_{21}q_{61}}{b}{q_{21}q_{51}q_{41}q_{71}}{q_{31},q_{41},q_{81},q_{71}}               \\
		\gendelta{q_{31}q_{21}q_{61}}{b}{q_{21}q_{51}q_{41}q_{71}}{q_{41},q_{81},q_{71}}                       \\
		\gendelta{q_{41}q_{21}q_{61}}{b}{q_{21}q_{51}q_{41}q_{71}}{q_{81},q_{71}}                              \\
	\end{array}
\end{align*}



\subsection{Определение детерменированности построенных автоматов \(M_1\), \(M_2\), \(M_3\)}
\begin{enumerate}
	\item Автомат \(M_1\) --- НКА, так как есть переходы
	      \begin{align*}
		      \delta_1(H,b) = \cbr{S_9, S_{11}} \\
		      \delta_1(S_{15},b) = \cbr{S_9, S_{11}}
	      \end{align*}
	\item Автомат \(M_2\) --- НКА, так как есть переходы
	      \begin{align*}
		      \delta_2(S_{15}, b) = \cbr{S_{10}, S_{12}} \\
		      \delta_2(S_{12}, b) = \cbr{S_{16}, F}
	      \end{align*}
	\item Автомат \(M_3 \equiv M_{17}\) --- НКА, так как есть переходы
	      \begin{align*}
		      \delta_3(q_{31}, b) = \cbr{q_{21}, q_{51}} \\
		      \delta_3(q_{41}, b) = \cbr{q_{21}, q_{51}} \\
		      \delta_3(q_{150}, b) = \cbr{q_{21}, q_{51}}
	      \end{align*}
\end{enumerate}
Все три представленных автомата являются недетерменированными.

\subsection{Построение детерменированных конечных автоматов для НКА \(M_1\), \(M_2\), \(M_3\)}

\begin{itemize}
	\item Ножество состояний \(Q'\) результирующего автомата ДКА состояит из всех подмножеств \(Q\) исходного автомата. Каждое состояние \(Q'\) обозначается как \([A_1,\dots,A_n]\), где \(A_i \in Q\). Тогда получаем число различных сочетаний
	      \begin{align*}
		      |Q'| = \sum_{k=1}^n C^k_n = 2^n - 1
	      \end{align*}
	\item Начальное состояние имеет вид (\(H\) --- начальное состояние автомата \(M\))
	      \begin{align*}
		      q'_0  \equiv [H]
	      \end{align*}
	\item Множество конечных состояний (конечные состояния исходного автомата \(F=\cbr{F_1, \dots, F_n}\)) имеет вид
	      \begin{align*}
		      F' = \cbr{[q_1, \dots, q_j,q_{j+1},\dots,q_{j+k} ]}    \\
		      \cbr{q_1, \dots, q_j} \subset \cbr{H, F_1, \dots, F_n} \\
		      \cbr{q_{j+1},\dots,q_j} \subset Q \setminus \cbr{H, F_1, \dots, F_n}
	      \end{align*}
\end{itemize}
\subsubsection{ДКА для \(M_1\)}
Переходы определяются как
\begin{align*}
	\begin{array}{ll}
		\gendelta{S_9}{a}{S_{15}}{\maqa}                               \\
		\gendelta{H}{a}{S_9}{\maqa,S_{15}}                             \\
		\gendelta{S_{15}}{a}{S_9}{\maqa}                               \\
		\gendelta{S_9S_{15}}{a}{S_9S_{15}}{\maqa,H}                    \\
		\gendelta{S_9H}{a}{S_9S_{15}}{\maqa}                           \\
		\gendelta{S_{11}}{c}{S_{16}}{\maqc}                            \\
		\gendelta{S_{15}}{c}{S_{11}}{\maqc, H, S_{16}}                 \\
		\gendelta{H}{c}{S_{11}}{\maqc, S_{16}}                         \\
		\gendelta{S_{16}}{c}{S_{11}}{\maqc}                            \\
		\gendelta{HS_{11}}{c}{S_{11}S_{16}}{\maqc,S_{15},S_{16}}       \\
		\gendelta{S_{15}S_{11}}{c}{S_{11}S_{16}}{\maqc,S_{16}}         \\
		\gendelta{S_{16}S_{11}}{c}{S_{11}S_{16}}{\maqc}                \\
		\gendelta{S_{11}}{b}{S_{16}}{}                                 \\
		\gendelta{S_{9}}{b}{S_{15}}{}                                  \\
		\gendelta{H}{b}{S_9S_{11}}{S_{15},S_{16}}                      \\
		\gendelta{S_{15}}{b}{S_9S_{11}}{S_{16}}                        \\
		\gendelta{S_{16}}{b}{S_{11}}{}                                 \\
		\gendelta{S_9H}{b}{S_9S_{11}S_{15}}{S_{15},S_{16}}             \\
		\gendelta{S_9S_{15}}{b}{S_9S_{11}S_{15}}{S_{16}}               \\
		\gendelta{S_{11}H}{b}{S_9S_{11}S_{16}}{S_{15},S_{16}}          \\
		\gendelta{S_{11}S_{15}}{b}{S_9S_{11}S_{16}}{S_{16}}            \\
		\gendelta{S_9S_{16}}{b}{S_{15}S_{11}}{}                        \\
		\gendelta{S_9S_{11}H}{b}{S_9S_{15}S_{11}S_{16}}{S_{16},S_{15}} \\
		\gendelta{S_9S_{11}S_{15}}{b}{S_9S_{15}S_{11}S_{16}}{S_{16}}   \\
		\gendelta{S_9S_{11}}{b}{S_{15}S_{16}}{}
	\end{array}
\end{align*}

\subsubsection{ДКА для \(M_2\)}
Переходы определяются как
\begin{align*}
	\begin{array}{ll}
		\gendelta{S_{10}}{a}{S_{15}S_{16}}{\mbqa, S_{16}}                     \\
		\gendelta{S_{15}}{a}{S_{10}}{\mbqa, S_{16}}                           \\
		\gendelta{S_{10}S_{15}}{a}{S_{10}S_{15}S_{16}}{\mbqa, S_{16}}         \\
		\gendelta{S_{15}}{c}{S_{12}}{\mbqc, S_{16}, F}                        \\
		\gendelta{S_{16}}{c}{S_{12}}{\mbqc, F}                                \\
		\gendelta{S_{12}}{c}{S_{16}F}{\mbqc, F}                               \\
		\gendelta{S_{12}S_{15}}{c}{S_{12}S_{16}F}{\mbqc, S_{16}, F}           \\
		\gendelta{S_{12}S_{16}}{c}{S_{12}S_{16}F}{\mbqc, F}                   \\
		\gendelta{S_{16}}{b}{S_{12}}{F}                                       \\
		\gendelta{S_{12}}{b}{FS_{16}}{F}                                      \\
		\gendelta{S_{10}}{b}{S_{15}S_{16}}{F}                                 \\
		\gendelta{S_{15}}{b}{S_{10}S_{12}}{S_{16},F}                          \\
		\gendelta{S_{15}S_{12}}{b}{S_{10}S_{12}S_{16}F}{S_{16},F}             \\
		\gendelta{S_{15}S_{10}}{b}{S_{10}S_{12}S_{15}S_{16}}{S_{16},F}        \\
		\gendelta{S_{10}S_{16}}{b}{S_{12}S_{15}S_{16}}{F}                     \\
		\gendelta{S_{12}S_{10}S_{15}}{b}{S_{10}S_{12}S_{15}S_{16}F}{S_{16},F} \\
		\gendelta{S_{12}S_{10}}{b}{S_{15}S_{16}F}{F}
	\end{array}
\end{align*}

\newpage
\subsubsection{ДКА для \(M_3\)}
\subsubsubsection{Удаление недостижимых символов}
Для удобства, сначала удалим недостижимые состояния:
\begin{align*}
	R = \cbr{q_{150}}, P_0 = \cbr{q_{150}}                                                                                                                                      \\
	P_1 = \cbr{q_{11},q_{21}, q_{51}, q_{61}}, R \setminus P_1 \neq \varnothing \Longrightarrow R = \cbr{q_{150}, q_{11},q_{21}, q_{51}, q_{61}}                                \\
	P_2 = \cbr{q_{31},q_{41}, q_{71}, q_{81}}, R \setminus P_2 \neq \varnothing \Longrightarrow R = \cbr{q_{150}, q_{11},q_{21}, q_{31}, q_{41} q_{51}, q_{61}, q_{71}, q_{81}} \\
	P_3 = \cbr{q_{51}, q_{61}}, R \setminus P_3 = \varnothing \Longrightarrow R = \cbr{q_{150}, q_{11},q_{21}, q_{31}, q_{41} q_{51}, q_{61}, q_{71}, q_{81}}                   \\
\end{align*}
Автомат после удаления недостижимых состояний
\begin{align*}
	M_3 = \rbr{\cbr{q_{150}, q_{11},q_{21}, q_{31}, q_{41} q_{51}, q_{61}, q_{71}, q_{81}}, \Sigma, \delta_3, q_{150}, \cbr{q_{71}, q_{81}}} \\
	\begin{array}{lll}
		\delta_{3}(q_{11}, a) =  \cbr{q_{31}} & \delta_{3}(q_{11}, b)=  \cbr{q_{41}}                                                  \\
		\delta_{3}(q_{21}, a) =  \cbr{q_{31}} & \delta_{3}(q_{21}, b)=  \cbr{q_{41}}                                                  \\
		\delta_{3}(q_{31}, a) = \cbr{q_{11}}  & \delta_{3}(q_{31}, b) = \cbr{q_{21}, q_{51}}  & \delta_{3}(q_{31}, c) = \cbr{q_{61}}  \\
		\delta_{3}(q_{41}, a) = \cbr{q_{11}}  & \delta_{3}(q_{41}, b) = \cbr{q_{21}, q_{51}}  & \delta_{3}(q_{41}, c) = \cbr{q_{61}}  \\
		\delta_{3}(q_{150}, a) = \cbr{q_{11}} & \delta_{3}(q_{150}, b) = \cbr{q_{21}, q_{51}} & \delta_{3}(q_{150}, c) = \cbr{q_{61}} \\
		\delta_{3}(q_{51}, b)  = \cbr{q_{71}} & \delta_{3}(q_{51}, c)  = \cbr{q_{81}}                                                 \\
		\delta_{3}(q_{61}, b)  = \cbr{q_{71}} & \delta_{3}(q_{61}, c)  = \cbr{q_{81}}                                                 \\
		\delta_{3}(q_{71}, b) = \cbr{q_{51}}  & \delta_{3}(q_{71}, c) =\cbr{q_{61}}                                                   \\
		\delta_{3}(q_{81}, b) = \cbr{q_{51}}  & \delta_{3}(q_{81}, c) =\cbr{q_{61}}                                                   \\
	\end{array}
\end{align*}

\begin{figure}[h!]
	\centering
	\begin{tikzpicture}[
			->,
			>=stealth',
			node distance=2.0cm,
			every state/.style={thick, fill=gray!10},
			initial text={Начало}
		]

		% Состояния
		\node[state, initial] (q150) {$q_{150}$};
		\node[state,  right of=q10] (q11) {$q_{11}$};
		\node[state,  right of=q20] (q21) {$q_{21}$};
		\node[state, right= of q11] (q31) {$q_{31}$};
		\node[state, right= of q21] (q41) {$q_{41}$};
		% Переходы
		\path
		(q150) edge node[above] {a} (q11)
		(q150) edge node[below] {b} (q21)
		(q11) edge node[below] {a} (q31)
		(q21) edge[bend left=20] node[above] {a} (q31)
		(q11) edge[bend right=20] node[below] {b} (q41)
		(q21) edge node[above] {b} (q41)
		(q31) edge[bend right=20] node[above]{a} (q11)
		(q31) edge[bend left=20] node[below]{b} (q21)
		(q41) edge[bend left=20] node[below]{b} (q21)
		(q41) edge[bend right=20] node[above]{a} (q11)
		;
		\node[state, above of=q110] (q51) {$q_{51}$};
		\node[state, below of=q110] (q61) {$q_{61}$};
		\node[state, accepting, right = of q51] (q71) {$q_{71}$};
		\node[state, accepting, right = of q61] (q81) {$q_{81}$};
		% Переходы
		\path
		(q51) edge node[below] {b} (q71)
		(q61) edge[bend left=20] node[above] {b} (q71)
		(q51) edge[bend right=20] node[below] {c} (q81)
		(q61) edge node[above] {c} (q81)
		(q71) edge[bend right=20] node[above] {b} (q51)
		(q81) edge[bend left=20] node[below] {c} (q61)
		(q71) edge[bend left=20] node[below]{c} (q61)
		(q81) edge[bend right=20] node[above]{b} (q51)
		;

		\path
		(q41) edge[bend right=30] node[below]{c} (q61)
		(q31) edge[bend left=30] node[above]{b} (q51)
		(q41) edge[bend left=20] node[above]{b} (q51)
		(q31) edge[bend right=20] node[below]{c} (q61)
		(q150) edge[bend right=60] node[below]{c} (q61)
		(q150) edge[bend left=60] node[above]{b} (q51)
		;

	\end{tikzpicture}
	\caption{Диаграмма состояний НКА \(M_{17}\)}
\end{figure}

\newpage
\subsubsubsection{Построение ДКА}
Переходы определяются как
\begin{align*}
	\begin{array}{ll}
		\gendelta{q_{150}}{a}{q_{11}}{\mcqa, q_{41}, q_{31}}                                \\
		\gendelta{q_{41}}{a}{q_{11}}{\mcqa, q_{31}}                                         \\
		\gendelta{q_{31}}{a}{q_{11}}{\mcqa}                                                 \\
		\gendelta{q_{11}}{a}{q_{31}}{\mcqa, q_{21}}                                         \\
		\gendelta{q_{21}}{a}{q_{31}}{\mcqa}                                                 \\
		\gendelta{q_{11}q_{150}}{a}{q_{11}q_{31}}{\mcqa,q_{31}, q_{41},q_{21}}              \\
		\gendelta{q_{11}q_{31}}{a}{q_{11}q_{31}}{\mcqa, q_{41},q_{21}}                      \\
		\gendelta{q_{11}q_{41}}{a}{q_{11}q_{31}}{\mcqa, q_{21}}                             \\
		\gendelta{q_{21}q_{150}}{a}{q_{11}q_{31}}{\mcqa,q_{31}, q_{41}}                     \\
		\gendelta{q_{21}q_{31}}{a}{q_{11}q_{31}}{\mcqa, q_{41}}                             \\
		\gendelta{q_{21}q_{41}}{a}{q_{11}q_{31}}{\mcqa}                                     \\
		\gendelta{q_{150}}{c}{q_{61}}{\mcqc,q_{31},q_{41},q_{71},q_{81}}                    \\
		\gendelta{q_{31}}{c}{q_{61}}{\mcqc,q_{41},q_{71},q_{81}}                            \\
		\gendelta{q_{41}}{c}{q_{61}}{\mcqc,q_{71},q_{81}}                                   \\
		\gendelta{q_{71}}{c}{q_{61}}{\mcqc,q_{81}}                                          \\
		\gendelta{q_{81}}{c}{q_{61}}{\mcqc}                                                 \\
		\gendelta{q_{51}}{c}{q_{81}}{\mcqc, q_{61}}                                         \\
		\gendelta{q_{61}}{c}{q_{81}}{\mcqc}                                                 \\
		\gendelta{q_{51}q_{150}}{c}{q_{61}q_{81}}{\mcqc,q_{31},q_{41},q_{71},q_{81},q_{61}} \\
		\gendelta{q_{51}q_{31}}{c}{q_{61}q_{81}}{\mcqc,q_{41},q_{71},q_{81},q_{61}}         \\
		\gendelta{q_{51}q_{41}}{c}{q_{61}q_{81}}{\mcqc,q_{71},q_{81},q_{61}}                \\
		\gendelta{q_{51}q_{71}}{c}{q_{61}q_{81}}{\mcqc,q_{81},q_{61}}                       \\
		\gendelta{q_{51}q_{81}}{c}{q_{61}q_{81}}{\mcqc,q_{61}}                              \\
		\gendelta{q_{61}q_{150}}{c}{q_{61}q_{81}}{\mcqc,q_{31},q_{41},q_{71},q_{81}}        \\
		\gendelta{q_{61}q_{31}}{c}{q_{61}q_{81}}{\mcqc,q_{41},q_{71},q_{81}}                \\
		\gendelta{q_{61}q_{41}}{c}{q_{61}q_{81}}{\mcqc,q_{71},q_{81}}                       \\
		\gendelta{q_{61}q_{71}}{c}{q_{61}q_{81}}{\mcqc,q_{81}}                              \\
		\gendelta{q_{61}q_{81}}{c}{q_{61}q_{81}}{\mcqc}                                     \\
		\gendelta{q_{11}}{b}{q_{41}}{q_{21}}                                                \\
		\gendelta{q_{21}}{b}{q_{41}}{}                                                      \\
		\gendelta{q_{51}}{b}{q_{71}}{q_{61}}                                                \\
		\gendelta{q_{61}}{b}{q_{71}}{}                                                      \\
		\gendelta{q_{150}}{b}{q_{21}q_{51}}{q_{31},q_{41},q_{81}, q_{71}}                   \\
		\gendelta{q_{31}}{b}{q_{21}q_{51}}{q_{41},q_{81}, q_{71}}                           \\
		\gendelta{q_{41}}{b}{q_{21}q_{51}}{q_{81}, q_{71}}                                  \\
		\gendelta{q_{81}}{b}{q_{51}}{q_{71}}                                                \\
		\gendelta{q_{71}}{b}{q_{51}}{}                                                      \\
	\end{array}
\end{align*}

\begin{align*}
	\begin{array}{ll}
		\gendelta{q_{71}q_{51}}{b}{q_{51}q_{71}}{q_{81},q_{61}}                                                \\
		\gendelta{q_{81}q_{51}}{b}{q_{51}q_{71}}{q_{61}}                                                       \\
		\gendelta{q_{71}q_{61}}{b}{q_{51}q_{71}}{q_{81}}                                                       \\
		\gendelta{q_{81}q_{61}}{b}{q_{51}q_{71}}{}                                                             \\
		\gendelta{q_{51}q_{11}}{b}{q_{41}q_{71}}{q_{61},q_{21}}                                                \\
		\gendelta{q_{61}q_{11}}{b}{q_{41}q_{71}}{q_{21}}                                                       \\
		\gendelta{q_{51}q_{21}}{b}{q_{41}q_{71}}{q_{61}}                                                       \\
		\gendelta{q_{61}q_{21}}{b}{q_{41}q_{71}}{}                                                             \\
		\gendelta{q_{71}q_{11}}{b}{q_{51}q_{41}}{ q_{81}, q_{21}}                                              \\
		\gendelta{q_{81}q_{11}}{b}{q_{51}q_{41}}{ q_{21}}                                                      \\
		\gendelta{q_{71}q_{21}}{b}{q_{51}q_{41}}{ q_{81}}                                                      \\
		\gendelta{q_{81}q_{21}}{b}{q_{51}q_{41}}{}                                                             \\
		\gendelta{q_{150}q_{11}}{b}{q_{21}q_{51}q_{41}}{q_{31},q_{41},q_{81},q_{71},q_{21}}                    \\
		\gendelta{q_{31}q_{11}}{b}{q_{21}q_{51}q_{41}}{q_{41},q_{81},q_{71},q_{21}}                            \\
		\gendelta{q_{41}q_{11}}{b}{q_{21}q_{51}q_{41}}{q_{81},q_{71},q_{21}}                                   \\
		\gendelta{q_{150}q_{21}}{b}{q_{21}q_{51}q_{41}}{q_{31},q_{41},q_{81},q_{71}}                           \\
		\gendelta{q_{31}q_{21}}{b}{q_{21}q_{51}q_{41}}{q_{41},q_{81},q_{71}}                                   \\
		\gendelta{q_{41}q_{21}}{b}{q_{21}q_{51}q_{41}}{q_{81},q_{71}}                                          \\
		\gendelta{q_{150}q_{51}}{b}{q_{21}q_{51}q_{71}}{q_{31},q_{41},q_{81},q_{71},q_{61}}                    \\
		\gendelta{q_{31}q_{51}}{b}{q_{21}q_{51}q_{71}}{q_{41},q_{81},q_{71},q_{61}}                            \\
		\gendelta{q_{41}q_{51}}{b}{q_{21}q_{51}q_{71}}{q_{81},q_{71},q_{61}}                                   \\
		\gendelta{q_{150}q_{61}}{b}{q_{21}q_{51}q_{71}}{q_{31},q_{41},q_{81},q_{71}}                           \\
		\gendelta{q_{31}q_{61}}{b}{q_{21}q_{51}q_{71}}{q_{41},q_{81},q_{71}}                                   \\
		\gendelta{q_{41}q_{61}}{b}{q_{21}q_{51}q_{71}}{q_{81},q_{71}}                                          \\
		\gendelta{q_{11}q_{71}q_{51}}{b}{q_{71}q_{51}q_{41}}{q_{21},q_{81},q_{61}}                             \\
		\gendelta{q_{21}q_{71}q_{51}}{b}{q_{71}q_{51}q_{41}}{q_{81},q_{61}}                                    \\
		\gendelta{q_{11}q_{81}q_{51}}{b}{q_{71}q_{51}q_{41}}{q_{21},q_{61}}                                    \\
		\gendelta{q_{21}q_{81}q_{51}}{b}{q_{71}q_{51}q_{41}}{q_{61}}                                           \\
		\gendelta{q_{11}q_{81}q_{61}}{b}{q_{71}q_{51}q_{41}}{q_{21}}                                           \\
		\gendelta{q_{21}q_{81}q_{61}}{b}{q_{71}q_{51}q_{41}}{}                                                 \\
		\gendelta{q_{150}q_{11}q_{51}}{b}{q_{21}q_{51}q_{41}q_{71}}{q_{31},q_{41},q_{81},q_{71},q_{61},q_{21}} \\
		\gendelta{q_{31}q_{11}q_{51}}{b}{q_{21}q_{51}q_{41}q_{71}}{q_{41},q_{81},q_{71},q_{61},q_{21}}         \\
		\gendelta{q_{41}q_{11}q_{51}}{b}{q_{21}q_{51}q_{41}q_{71}}{q_{81},q_{71},q_{61},q_{21}}                \\
		\gendelta{q_{150}q_{21}q_{51}}{b}{q_{21}q_{51}q_{41}q_{71}}{q_{31},q_{41},q_{81},q_{71},q_{61}}        \\
		\gendelta{q_{31}q_{21}q_{51}}{b}{q_{21}q_{51}q_{41}q_{71}}{q_{41},q_{81},q_{71},q_{61}}                \\
		\gendelta{q_{41}q_{21}q_{51}}{b}{q_{21}q_{51}q_{41}q_{71}}{q_{81},q_{71},q_{61}}                       \\
		\gendelta{q_{150}q_{11}q_{61}}{b}{q_{21}q_{51}q_{41}q_{71}}{q_{31},q_{41},q_{81},q_{71},q_{21}}        \\
		\gendelta{q_{31}q_{11}q_{61}}{b}{q_{21}q_{51}q_{41}q_{71}}{q_{41},q_{81},q_{71},q_{21}}                \\
		\gendelta{q_{41}q_{11}q_{61}}{b}{q_{21}q_{51}q_{41}q_{71}}{q_{81},q_{71},q_{21}}                       \\
		\gendelta{q_{150}q_{21}q_{61}}{b}{q_{21}q_{51}q_{41}q_{71}}{q_{31},q_{41},q_{81},q_{71}}               \\
		\gendelta{q_{31}q_{21}q_{61}}{b}{q_{21}q_{51}q_{41}q_{71}}{q_{41},q_{81},q_{71}}                       \\
		\gendelta{q_{41}q_{21}q_{61}}{b}{q_{21}q_{51}q_{41}q_{71}}{q_{81},q_{71}}                              \\
	\end{array}
\end{align*}



\end{document}

% \begin{document}
% \end{document}
%
